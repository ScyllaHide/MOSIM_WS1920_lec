% !TeX spellcheck = en_US
\section{Floating point numbers}
\subsection*{28.10.2019}
A floating point number is 
\begin{align*}
	x = \begin{cases}
	= (-1)^{S_x}\cdot m_xb^{e_x}=(S_x,m_x,e_x) &\quad \\
	0 \quad
	\end{cases}
\end{align*}
with sign bit (Vorzeichen-bit) $S_x\in \{0,1\}$, mantissa $m_x=0.m_1m_2..m_l$ and exponent $e_x\in \{\emin,e+1,\dots,\emax\} \with \emin \approx -\emax$. The mantissa digits are $m_i\in S-\{0,1,..,b\}$ with $b:=b-1$ and mantissa length $l\in N^+$ ($b =$ base, $\emin =$ emin, $\emax =$ emax and $\N^+ = \N\setminus \set{0,1}$).
\begin{itemize}
    \item  Floating point format $R=R(b,l,\emin,\emax,\denorm)$
    with
    \begin{align*}
    	\denorm = \begin{cases}
    		\true &\quad\text{if denormalized FPN is allowed}\\
    		\false &\quad\\
    	\end{cases}
    \end{align*}
  \item normalized FPN $x\neq 1$: has as first mantissa digit $m_1\neq 0$ $(b=2 \implies m_1=1!)$
  \item unnormalized FPN $x\neq0$ has $m_1=0$
  \item Normalization of a FPN $x \neq 0$: mantissa digits to push for $k$ digits to the left ($k =$  number of leading 0-digits) and subtracting $k$ from $e_x$. $e_{x_neu}=e_x-k$.
   Normalization is only possible if $e_{xneu} = e_x-k \ge e$
  \item denormalized FPN: $e_x = e$ and mantissa not normalized or can not be normalized
  \item number line is symmetrical to zero! Here its supposed to be the line.
  \item \begriff{\person{Wilkinson}-epsilon}
  \begin{align*}
  	\epsilon:=b^{1-l}=\frac{1}{b^{l-1}}
  \end{align*}
  is the biggest relative gap between two neighbouring normalized FPN $\sim$ the biggest relative error in solving 
  %TODO add example!
  \item \begriff{unit in last place} ulp:
  \begin{align*}
    \ulp(x) := \begin{cases}
    	b^{e_x-l} &\quad\text{if x is normal}\\
    	b^{e-l} &\quad\text{if x is denormalized or }0
    \end{cases}
  \end{align*} 
 ``unit in the last place'' (at the mantissa of $x$. 
  \begin{align*}
  	x=0.m_1 m_2...m_l\cdot b^{e_x} \with m_l = b^{e_x - l}
  \end{align*}
  \item \begriff{successor} ($k \in \Z$)
  \begin{align*}
  	\success(x) := \begin{cases}
  		x+\ulp(x) &\quad \text{if } x\neq -b^k\\
  		x+(\frac{\ulp(x)}{b}) &\quad \text{if } x=-b^k
  	\end{cases}
  \end{align*}
  \item \begriff{predecessor} ($k\in \Z$)
  \begin{align*}
  	\pred(x) := \begin{cases}
  		x-\ulp(x)& \quad\text{if } x\neq b^k\\
  		x-(\frac{\ulp(x)}{b})& \quad\text{if } x=b^k	
  	\end{cases}
  \end{align*}
  \begin{tabular}{l|r} %{\textwidth}
  	%\hline
  	\textbf{$x$}   & $\digits_b(x)$\\ \hline
  	0 & 1\\
  	$b^0 = 1$ & $\vdots$ \\
  	$\vdots$ & $\vdots$ \\
  	$b-1$ & 1\\
  	$b^1 = [10]_b$ & 2\\
  	$\vdots$ & $\vdots$\\
  	$b^2 -1$ & 2\\
  	$b^2 = [100]_b$ & 3\\
  \end{tabular}
	where we have base $b \in \N\setminus \set{0,1}$ and $\digits_b(x) = \floor{ \log_b(x)} +1$
\end{itemize}
\subsection*{Interval arithmetics (Wrap-around) in two-part complement}
	\begin{itemize}
		\item exact result: $z^* =x\frac{\pm}{\ast} y$, $x,y\in I_n$ (Integer with $n$ bits inclusive sign)
		\item generated result: 
		\begin{align*}
			z:=((z^*+z^{n-1})\mod2^n)-2^{n-1}\in I_n
		\end{align*}
		\item for normal mantissa $m_x$ it holds: $\frac{1}{b}\le m_x<1$
		\item  for normal mantissa $m_x$ it holds: $0\le m_x<\frac{1}{b}$
	\end{itemize}

\begin{itemize}
    \item 2 equivalent display possibilities for FPN $x\neq 0:$
    \begin{align*}
    	x=(-1)^{S_x}\cdot m_x\cdot b^{e_x}=(-1)^{S_x}\cdot M_x\cdot b^{e_x-l}\\
        M_x=m_x\cdot b^l \in \N \with b^{l-1}\le M_x<b^l
    \end{align*}
    Optionally one could also add $M_x \implies \abs{M_x}<b^l$, $M_x\in \Z$\\
    $\implies$ Every FPN $x$ is an integer multiple of its $\ulp$`s.
    \item $\ulp(1)=\epsilon$, since $1=0,10\dots0\cdot b^1$ and $\epsilon=b^{1-l}=\ulp(1)$ $x\not =0, x\in R \leftarrow$ grid: 
    \begin{itemize}
        \item $\success(x)\le x(1+\epsilon)=x(1+\ulp(1= = 1\cdot \succ(1)$
        \item $\success(x)\ge x(1-\epsilon)=x(1-\ulp(1)=x\cdot \pred(1)$
    \end{itemize}
    \item relative distance between 2 neighboring FPN:
    \todo[inline]{TODO add graph} 
   \end{itemize}
    \begin{*definition}[Rounding]
    	Rounding is a map $o:\R\to R $ FP-grid with the following properties
    	\begin{defenum}
    		\item ($R1$) $ox=x, \forall x\in R$ (projection) \label{def_rounding_1}
    		\item ($R2$) $x,y \in R: x\le y \implies ox\le oy$ (monotonicity) \label{def_rounding_2}
    		\item ($R3$) $o(-x)=-ox, \forall x\in R$ \label{def_rounding_3}
    	\end{defenum}
    \end{*definition}
	\todo[inline]{TODO add graph!}
	\begin{*remark}
		An antisymmetric rounding i guess additionally satisfies also \ref{def_rounding_3}.
	\end{*remark}
	\begin{*example}
		Established rounding possibilities are:
		\begin{itemize}
			\item $\square x$: \i to nearest floating point number $\to$
			\begin{itemize}
				\item error $\le \frac{1}{2}\ulp$
				\item (stochastically calculated: ``round to even'' $=$ to next FPN with even mantissa, i.e. end digit = 0 (binary)
				\item \emph{If} the value which is to be rounded lays exactly in the middle between 2 FPN
				\item satisfies \ref{def_rounding_3}): $\square (-x)=-\square x$
			\end{itemize}
		
			\item $\sqcup x$: truncation, rounding $\to 0$
			\begin{itemize}
				\item $\abs{\sqcup x} \le \le\abs{\sqcup x}$
				\item maximal error $< 1 \ulp$
				\item satisfies \ref{def_rounding_3}
			\end{itemize}
			\item $\sqcap x$: augmentation($=$ the, in absolute value, biggest FPN), 
			\begin{itemize}
				\item $\abs{x} \le \abs{\sqcap x}$
				\item maximal error $<1 \ulp$
				\item satisfies \ref{def_rounding_3}
			\end{itemize}
			\item $\rndup x$: upward rounding $\to$ $+\infty$
			\begin{itemize}
				\item $x \le \rndup x$
				\item maximal error $< 1 \ulp$
				\item satisfies \emph{not} \ref{def_rounding_3} 
			\end{itemize}
			\item $\rnddown x$: downwards rounding $\to -\infty$
			\begin{itemize}
				\item $\nabla x\le x$
				\item maximum error $< 1 \ulp$
				\item satisfies \emph{not} \ref{def_rounding_3}
				\item instead of \ref{def_rounding_3} the antisymmetry holds:
				\begin{align*}
					\rnddown(-x)=-\rndup x \Leftrightarrow \rndup(-x) = - \rnddown x
				\end{align*}
			\end{itemize}
		\end{itemize}
	\end{*example}
\subsection*{Intervals}
	\begin{itemize}
		\item $I\R \set{X=[\minval{x},\maxval{x}] \mid \minval{x}\le \maxval{x}\nd \minval{x},\maxval{x} \in \R}$ , i.e. set of the bounded, closed, real intervals
		\item Floating-point intervals: $IR =  \set{X=[\minval{x},\maxval{x}] \mid \minval{x}\le \maxval{x}; \minval{x},\maxval{x} \in R}$ intervals with floating point boundaries
	\end{itemize}
\begin{*definition}
	Interval rounding as map 
	\begin{align*}
		\rndinterval \colon \R \to R \with X=[\rndup x, \rnddown x]\subseteq X=[\minval{x},\maxval{x}]
	\end{align*}
	with properties
	\begin{defenum}
		\item $R1$ $\rndinterval X=X \quad\forall X\in \R$ \label{def_interval_rounding_1}
		\item $R2$ $X,Y\in R: X\subseteq Y \implies \rndinterval X\le \rndinterval Y$ (Inclusion isotonic) \label{def_interval_rounding_2}
		\item $R3$ $\rndinterval(-X)=-\rndinterval X$, hence $\rndup(-x)=-\rnddown(x)$, $\rnddown(-x)=-\rndup x \in R$ \label{def_interval_rounding_3}
	\end{defenum}
	Then holds:
	\begin{align*}
	\begin{matrix}
	X + Y &= [\minval{x} + \minval{y}, \maxval{x} + \maxval{y}]\\
	X - Y &= [\minval{x} - \maxval{y}, \maxval{x} - \minval{y}]
	\end{matrix} \quad \diam(X\pm Y) = \diam X + \diam Y
	\end{align*}
\end{*definition}
\subsection*{11.11.19}
\begin{itemize}
	\item Interval arithmetics $(A,B,C,X\in IR)$
	\begin{itemize}
		\item $A+B:=[\minval{a}+\minval{b},\maxval{a}+\maxval{b}]$
		\item $-B:=[-\maxval{b},-\minval{b}]$
		\item $A-B:=[\minval{a}-\maxval{b},\maxval{a}-\minval{b}]=A+(-B)$
		\item $A*B:=[\min S_x,\max S_x]$
		\item $\sfrac{\one}{B}:=[\sfrac{\one}{b},\sfrac{\one}{b}] \with \zero\notin B$ ($\one \nd \zero$ are the neutral elements for $\cdot \nd +$)
		\item $\sfrac{A}{B}:=[\min S_x, \max S_x]$ for $0\not \in B, \sfrac{A}{B}=\sfrac{A \cdot \one}{B}$
		\item For $0\in B=\int B: \sfrac{\one}{B}:=\sfrac{1}{[b,0)}\cup \sfrac{1}{(0,b]}$, inverse value mapping yields again $B$!
		\item General division $\sfrac{A}{B}$ with $\zero\in B$ in \emph{exercise} 3.
	\end{itemize}
	\item Transition to floating-point intervals $A,B \in IR$
	\begin{itemize}
		\item $A\rndintervalin{+}B=[\minval{a} \rndupin{+} \minval{b},\maxval{a}\rnddownin{+} \maxval{b}]$, this means $- \maxval{a} \rndupin{+} - \maxval{b}$, but in general, this is bas, since continuous changing between rounding modes necessary
		\item $A\rndintervalin{-}B=[\minval{a}\rndupin{-} \maxval{b},\maxval{a}\rnddownin{-} \minval{b}]$
		\item $A\rndintervalin{\cdot}B=[\min S_{\rndupin{+}}, \max S_{\rnddownin{+}}]$ with $S_{\rnddownin{\cdot}}$ and $S_{\rndupin{\cdot}}$ are 8 real directed rounded multiplications
		\item $A\rndintervalin{\setminus}B=[\min S_{\rndupin{\setminus}}, \max S_{\rnddownin{\setminus}}]\subseteq A\rndintervalin{\cdot}(\one \rndintervalin{\setminus}B)$
		\item Inclusion property:
		\begin{itemize}
			\item $A\subseteq A', B\subseteq B' \implies A\cdot B\le A' \cdot B'$
			\item $f(A)\le f(A')$, $g(A,B)\subseteq g(A', B')$
		\end{itemize}
		\item Overestimation:
		\begin{itemize}
			\item $X-X=[x-x,x-x]=[-\diam(X),+\diam(X)] \with \diam X=\maxval{x}-\minval{x}$ (diameter of $X$)
			\item $\sin^2 x + cos^2 x=[\minval{f},\maxval{f}]\ni \one$
		\end{itemize} 
		\item Implementation of interval functions, Monotonicity is eventually piecewise necessary\\
		some examples: $e^X=[e^{\minval{x}},e^{\maxval{x}}]$, $\log_b x=[\log_b \minval{x}, \log_b \maxval{x}]$, $\square(x)=[\square(\minval{x}),\square(\maxval{x})]$ and other (e.g. trigonometric functions) functions: local minimums /maximums
		\todo[inline]{TODO graph}
	\end{itemize}
\end{itemize}
\subsection*{12.11.19}
Our goal is a reliable arithmetic (and also standard-function) for usual space of numerical computation.
\begin{itemize}
	\item $\Z\supset \Z_8\supset \Z_4 \supset \Z_2 \gets$ formally with 2 with operations $+,-, \cdot, /$ and division with reminder
	\item $\R \supset R_10 \supset R_8 \supset R_4 \supset R_2$ with operations $+,-, \cdot, /$
	\item $V\R$ (vectors) $\supset VR$ with operations $+,-$, scalar multiplication/ division, scalar product ($V\R \times V\R \to V\R$)
	\item $M\R$ (matrices) $\supset  MR$ with operations $+,-$, scalar multiplication/ division, matrix product ($V \cdot M$)
	\item $\powerset(\R) \supset I\R \supset IR \dots$
	\item $\powerset(V\R) \supset  IV\R \supset IVR \dots$
	\item $\powerset(M\R) \supset  IM\R \supset IMR \dots$
	\item and analogous: $\C, V\C, M\C, \powerset(\C) \supset I\C, P(VC) \supset IV\C, \powerset(M\C) \supset  IM\C$
	\item \person{(Ulrich) Kuhlisch}: non traditional definition of arithmetic (i.e. all with help of real basic operations in $\R$)\\
	But rather: horizontal definition of arithmetic, separately for each plane/ layer (with optimal rounding), this is called \begriff{semimorphic arithmetic}, for each unit operation $O \in \set{+,-, \cdot,/,\scaProd{\cdot}{\cdot}}$, where $\scaProd{\cdot}{\cdot}$ is a scalar product.\\
	The result should be calculated in the following way: as if the mathematical exact result was calculated, and then (with the actual rounding mode ) was rounded. ($O \in \set{\square, \sqcap, \sqcup, \rndup, \rnddown, \dots}$).\\
	$\forall x,y\in N$ (where $N$ is the \begriff{nummerical space}) it has to hold $x\compop{\cdot} y:= O(x \bullet y)$, where $O$ is an operation rounding (computer operation) and $\bullet$ is an exact operation.
	\todo[inline]{TODO find out whats the error here?!}
\end{itemize}
\subsection*{Interval spaces}
	\begin{itemize}
		\item $I\R:=\set{A=[\minval{a},\maxval{a}]\mid \minval{a}\le \maxval{a}; \minval{a},\maxval{a} \in \R}$ with $A = [\minval{a},\maxval{a}] = \set{a \in \R \mid \minval{a} \le a \le \maxval{a}}$
		\item $I\C:=\set{C=[\minval{c},\maxval{c}]\mid \minval{c}\le \maxval{c}; \minval{c},\maxval{c}\in \C}$ where the order relation $\le$ on $\C$ is defined by $a\le b \iff \Re(a)\le \Re(b) \land \Im(a)\le \Im(b)$
		\item subset properties:
		\begin{itemize}
			\item in $I\R\colon A\subseteq B \iff \minval{b}\le \minval{a} \land \maxval{a}\le \maxval{b}$
			\item in $I\C: $ for $\Re$ and $\Im$ simultaneous
		\end{itemize}
	\item representation of complex intervals:
	\begin{itemize}
		\item $+$ either: rectangles in $C$:
		$C=[\minval{c},\maxval{c}]:=\{c\in \C\mid \Re(\minval{c})\le \Re(c)\le \Re(\maxval{c}) \land \Im(\minval{c})\le \Im(c)\le \Im(\maxval{c}) \with \minval{c},\maxval{c}\in C\}, \minval{c}\le \maxval{c}$
		\begin{align*}
					[\minval{c},\maxval{c}]&=[(\minval{a},\minval{b}),(\maxval{a},\maxval{b})]=\text{ pair complex intervals }\\
			&=([\minval{a},\maxval{a}],[\minval{a},\maxval{b}])=\text{ pair real intervals }
		\end{align*}
		\todo[inline]{graph here to do!!!}
		\item $+$ as complex "circles" or even ellipse
		\item In general the result of every operation (or function)
		\begin{align*}
			O: I\R\times I\R \to I\R\with O\in\set{+,-,\cdot,/}
		\end{align*}
		(or with $\C$ instead of $\R$) must be defined in the following way:\\ 
		$A,B\in I\R$ (or $I\C$): $A O B:=$ interval hull ($S_O$), with $S_O:=\set{a O b \mid a\in A, b\in B}$ and in $I\R$ we define $A O B=[\min S_O, \max S_O]$ and it suffices to consider the set $S_O:=\set{\minval{a} O \maxval{b}, \minval{a} O \maxval{b},\maxval{a} O \minval{b},\maxval{a} O \maxval{b}}$
	\end{itemize}
	\item Algebraic properties (in $I\R$)
	\begin{itemize}
		\item $+,\cdot$ are commutative, associative and there exist neutral elements: $\zero=[0,0], \one=[1,1]$. but in \emph{general not inverse elements} such that holds:
		\begin{itemize}
			\item $A+X=0, A+X=1$ not solvable, if $\diam(A)>0$.
			\item Note: $A-A \ni \zero$, but $A-A \neq \zero$ if $\diam(A)>0$ and $A\setminus A\ni 1$, but $A \setminus A\neq \one$ if $\diam(A)>0$\\
			Distributive law does not hold, but the sub-distributivity: $(A\cdot(B+C))\subseteq A\cdot B+A\cdot C$
			\begin{*example}
				\begin{align*}
					[0,1]\cdot([1,1]+[1,1])&=[0,0]=0\\
					[0,1]\cdot \one+[0,1]\cdot(-\one)&=[0,1]+[-1,0]=[-1,+1]\quad \zero\nsubseteq [-1,+1]
					\intertext{but}
					a\cdot(B+C)&=a\cdot B+a\cdot C\quad \forall a\in \R\\
					A\cdot(B+C)&=A\cdot C+B\cdot C \text{ if } b\cdot c\ge \zero, \forall b\in B \nd c\in C
				\end{align*}
			\end{*example}
		\end{itemize}
	\end{itemize}
	\item Algebraic properties in $I\C$:
	\begin{itemize}
		\item Associative law does not hold!
		\begin{*example} Let us set $A=[2,4]+\ii[0,0] \nd B=[1,1]+\ii[1,1]=C$, then
			\begin{align*}
				(A\cdot B)\cdot C =([2,4]+\ii[2,4])\cdot C=[-2,2]+\ii[4,8]\\ 
				A\cdot (B\cdot C)=A\cdot ([0,0]+\ii[2,2])=[0,0]+\ii[4,8]
			\end{align*}
			Then we get $[-2,2]+\ii[4,8]\neq[0,0]+\ii[4,8]$.
		\end{*example}
		\item Inclusion property:
		\begin{itemize}
			\item $A\subseteq A' , B\subseteq B' \implies A O B\subseteq A' O B'$
			\item for functions: $A\subseteq A' \implies f(A)\subseteq f(A')$ or $A_i\subseteq A'_i \forall i \implies f(A_1,..,A_n)\subseteq f(A'_1,\dots,A'_n)$
		\end{itemize}
	\item In composition of operations and functions to mathematical formulas and calculation rules, this property will remain valid. Since only inclusions for the solution are calculated.
	\begin{*example} Calculate $(X-2)^2-4$ for $X=[1,4]$
		\begin{align*} 
			(X-2)^2-4=[-1,2]^2-4=[0,4]-4=[-4,0]\\
			(X-2) O (X-2)-4=[-1,2]\cdot[-1,2]-4\\
			=[-2,3]-4=[-6,0]\\
			X*(X-4)=[1,4]\cdot[-3,0]=[-12,0]\\
			X*X-X*4=[1,16]-[4,16]=[-15,12]\\
		\end{align*}
		Then we get $[-4,0]\subseteq [-6,0] \subseteq [-12,0] \subseteq [-15,12]$ \todo[inline]{check which operations have which symbols for u!}
	\end{*example}
	\end{itemize}
\end{itemize} 
\subsection*{18.11.19}
\begin{itemize}
	\item $A\circ B:=$ intervals $\{a\circ b| a\in A, b\in B\}=[$minSo, maxSo$]$, with So$=\{a\circ b,a\circ b,a\circ b,a\circ b\}$
	\item In general $\not \exists $inverse:
	 If $\diam(A)=a-a>0, $i.e. $a<a$:
	\begin{enumerate}
		\item $A+X=[0,0]=0 \iff a+x=0 \land a+x=0 \implies x=-a>-a=x$, i.e. $x>x$ is prohibited
		\item $A*X=[1,1]=1 \iff a\cdot x=1 \land a\cdot x=1 \implies x=\frac{1}{a}>\frac{1}{a}=x$ \todo[inline]{Here the sign at the end is missing}
	\end{enumerate}
	\item cancelation law %Kürzungsregel (in english?)
	\begin{enumerate}
		\item $A+X=B+X \iff A=B$
		\item $A*X=B*X \iff A=B$ it holds $\forall X\in IR$ with $0\not \in X$
	\end{enumerate}
	\begin{*example}
		Let $X=[-1,+1]: A=-1,B=+1: A*X=X=B*X=[-1,a]*X=[b,+1]*X, \forall a\in X, b\in X$, then we get $X=[0,1]: [0,2]*X=[1,2]*X=[2,2]*X=[0,2]$
	\end{*example}
	\begin{*example}
		\begin{enumerate}
			\item A Calculation example: \todo[inline]{this needs better idea, its a whole calculation, but right now in itemize :(}
			\begin{itemize}
				\item $A=[-3,1], B=[2,4]$
				\item $A+B=[-1,5]$ (and $\diam(A+B)=$$\diam(A)+$$\diam(B)$)
				\item $A-B=[-7,-1]$ (and $\diam(A-B)=$$\diam(A)-$$\diam(B)$)
				\item $B-A=[1,7]=-(A-B)$
				\item $A-A=[-4,+4]$
				\item $B-B=[-2,+2]$
				\item $A*B=[-12,+4]$, S$_*=\{-6,-12,2,4\}$
				\item $A*A=[-3,+9]$, S$_*=\{+9,1,-3,-3\}$
				\item $B*B=B^2=[4,16]$
				\item $A^2=[0,+9]$
				\item $A/B=[-3/2,+1/2]$, S$_/=\set{-3/2,-3/4,1/2,1/4}$ \todo[inline]{What is $S$ here?}
				\item $A*1/B=[-3,1]*[1/b,1/b]=[-3/2,1/2]$
				\item $B/A=$ prohibited, since $0\in A$
				\item $B/B=[2/4,4/2]*[1/2,2]$, S$_/=\{1,2/4,4/2,1\}$
				\item The value range of a function over an interval $X=[x,x]$ $W_f(X)=\{f(x)|x\in X\} \to $ interval hulls take hull(W$_f(X)$)$\in IR$
			\end{itemize}
		\item An example: $X=[0,\pi/6]$
			\begin{enumerate}
				\item $f(x)=\sin(x)$, $\sin([0,\pi/6])=[0,\frac{1}{2}]$
				\item $f(x)=\cos(x)$, $\cos([0,\pi/6])=[\sqrt{3}/2,1]$
				\item $f(x)=\tan(x)$, $\tan([0,\pi/6])=[0,1/sqrt{3}] \neq\frac{\sin(x)}{\cos(x)}$, here "in general" is missing 
				\item $f(x)=\cot(x)$, $\cot([0,\pi/6])=[\sqrt{3}, \infty]$
				\item $f(y)=e^y$, $e^y=[e^y, e^y]$ \todo[inline]{ here the line above/below $y$ is missing!}
			\end{enumerate}
		\end{enumerate}
	\end{*example}
	 
	\item Interval evaluation (Interval-extension) of a function $f$ over an interval $X\in IR$.\\
	Is obtained by substituting $X$ (as a set of real numbers) in each occurrence of the variable $x$ in the calculation rule.\\
	$f(X)\supseteq $W$_f(X)$, usually with overestimation of the range (WB).\\
	Especially if $x$ occurs multiple times in the calculation rule
	\begin{*example}
		Here is an example, which illustrates:
		 $+$ $f(x)=x-x \iff g(x)=0$
		 $f(X)=[-$$\diam(X),+$$\diam(X)], q(X)=[0,0]$
		 $+$ $f(x)=x*x \iff g(x)=x^2$
		 $F(X)$ contains also negative values, if $x<0$ and $x>0$, 
		 $g(X)=[-,($max$\{|x|,|x|\})^2]$
		 $+$ $f(x)=sin^2x+cos^2x \iff g(x)=1$
		 $f([0,\frac{\pi}{6}])=[0,\frac{1}{4}]+[\frac{3}{4}.1]=[\frac{3}{4},\frac{5}{4}]\ni \one$, $g([0,\frac{\pi}{6}])=[1,1]$
	\end{*example}
	\item Solving the equality $A*X=B$ with $A,B,X\in IR, A\neq[0,0]$
	 Help-function: 
	 \begin{align*}
	 		h(A):= \begin{cases}
	 			a/a, & \text{if} \le\abs{a}\\
	 			a/a, & \text{else }
	 		\end{cases}
	 \end{align*}
	\begin{proposition}[\person{Ratschek}]
		\begin{enumerate}
			\item $AX=B$ has a (algebraic) solution $X\in IR$ iff $h(A)\ge h(B)$.
			\item If $h(A)=h(B)\le 0$, then the solution is not unique.
		\end{enumerate}
	\end{proposition}
	\begin{*example}
			\begin{enumerate}
				\item $A=[1,2], B=[-1,3]$, so: $[1,2]\ast[x,x]=[-1,3] \implies X_{\text{general}}=[-1/2,3/2]$ is unique, since $h(A)=1/2, h(B)=-1/3 \implies h(A)\ast h(B) \to \exists$ uniqueness
				\todo{Here the sign between $h(A)$ and $h(B)$ is missing}
				BUT: the set of all point solutions for the equations $ax=b$ with $a\in A, b\in B $ is:
				\todo{fix brackets or small fractions}
				\begin{align*}
					X_{set}=\{x=\frac{a}{b}| a\in A, b\in B\}=\frac{B}{A}=\frac{[-1,+3]}{[1,2]}=[-1,3]\supseteq X_{general}=[-\frac{1}{2},\frac{3}{2}]
				\end{align*}
				 $\square$ In general, it holds: the set of all point-solutions $X_{\text{set}}=\frac{B}{A}\supseteq X_{general}$ (algebraic solutions)
				 If $AX=B $ with $0\not \in A$ has the algebraic solution $X\in IR$, then $X_{general}\subseteq B/A$, since $\forall x\in X_{general}$ it holds $\forall a\in A: a\cdot x=b \in B$, while $X_{set}=B/A:=\{x=b/a| b\in B, a\in A\}$.
				\item $A=[-1/3,1], B=[-1,2]$ so $[-1/3,1]\cdot X=[-1,2] \implies x=[-1,2]$ is a unique solution\\
				$h(A)=-\frac{1}{3}, h(B)=-\frac{1}{2} \implies h(A)(?)h(B) \implies \exists$ unique solution! \todo{Here also its missing the sign between $h(A)$ and $h(B)$}
				 BUT:
				 The set of the point-solutions yields $R$ since $\frac{B}{A}$ is a division through null-interval. 
			\end{enumerate}
		
	\end{*example}
\end{itemize} 

\begin{*definition}
	For $A\in IR$ the diameter is defined by $\diam(A):=a-a= \max_{x,y\in A}\abs{x-y}$ \todo[inline]{max and min a's in definition?}
	\begin{itemize}
		\item is the radius $\rad(A):=\frac{a-a}{2}=\frac{\diam(A)}{2}$
		\item is the midpoint: $\mid(A)=a:=\frac{a+a}{2}=a+\rad(A)=a-\rad(A)$
	\end{itemize}
\end{*definition}
%
\subsection*{25.11.2019}
\begin{*definition}
	For $A\in IR$ the diameter is $diam(A):=a-a=\max\limits_{x,y\in A}\abs{x-y}$
	\begin{itemize}
		\item the radius is: $\rad(A):=\frac{a-a}{2}=\frac{diam(A)}{2}$
		\item the midpoint is: $mid(A)=a:=\frac{a+a}{2}=a+\rad(A)=a-\rad(A)$
		\item  is the maximum of the absolute value 
		\begin{align*}
			|A|:=\max\limits_{a\in A}|a|=\max\set{|a|,|a|}
		\end{align*}
		\item is the minimum of the absolute value:
		\begin{align*}
			\langle=\min\limits_{a\in A}|a|=
			\begin{cases}
			\min{|a|,|a|}, & \text{if } 0\not \in A\\
			0, & \text{if } 0\in A \\
			\end{cases}
		\end{align*}
	\end{itemize}
(An alternative definition of the absolute value of an interval: $\abs{A}^{\ast} [\langle A \rangle,\abs{A}]$).
\end{*definition}
\begin{*definition}
	The distance function takes the distance between two intervals $A,B\in IR$, and is defined by
	\begin{align*}
		d(A,B):=\max\set{\abs{a-b},\abs{a-b}}
	\end{align*}
	\todo[inline]{is this alright or typo?!}
\end{*definition}
\begin{*remark}
	For point-intervals $A=[a,a], B=[b,b]$ this is a real distance $d(A,B)=d(a,b)=\abs{a-b}$.
\end{*remark}
\begin{*definition}[Alternative definition]
	$d(A,B):=\max\{\sup\limits_{a\in A}\{d(a,B)\}, \sup\limits_{b\in B}\{d(b,A)\}\}$, with $\sup\limits_{a\in A}\{d(a,B)\}:=\sup\limits_{a\in A}\inf\limits_{b\in B}\}$
\end{*definition}
\begin{definition}[alternative definition]
	\begin{itemize}
		\item $d(A,B):=\abs{mid(A)-mid(B)}+\abs{\rad(A)-\rad(B)}$
		\item $d(A,B):=\inf\set{q\in R_0^+\mid A\subseteq B+[-q,+q] \land B\subseteq A+[-q,+q]}$
\end{itemize}
	In every definition $d(A,B)$ is a metric.
\end{definition}
\begin{proof}\todo[inline]{is this really a proof? But of which claim?!}
	We use the first definition \todo[inline]{Which definition?} 
	$\cdot d(A,B)\ge 0$
	$d(A,B)= 0 \iff a=b \land a=b \iff A=B$\\
	$d(A,B)\le d(A,C)+d(B,C) \to (*)$ \\
	$d(A,B)=d(B,A)$ \\
	(*)\todo{set reference}: $d(A,C)+d(B,C)= \max\{|a-c|,|a-c|\}+\max\{|b-c|,|b-c|\} \ge \max\{|a-c|+|b-c|,|a-c|,|b-c|\} \ge \max\{|a-b|,|a-b|\}= d(A,B)$.
\end{proof}
$IR$ is with this metric a complete metric space, i.e. every cauchy-sequence $(x_n)$ in $IR$ converges towards an interval $X\in IR$. \\
$\forall \epsilon>0 \exists N_{\epsilon}\in N: d(X_n,X_m)<\epsilon \forall n,m \ge N_{\epsilon}$, then it holds \\ $\lim_{n\to\infty} x_n=x$ for a $X\in IR$ with $X=[x,x]$ \\
$\lim_{n\to\infty} x_n=x \iff \lim_{n\to\infty} d(x_n,x)=0 \iff (\lim_{n\to\infty} x_n=x \land \lim_{n\to\infty} x_n=x$). \\
Important special cases for Cauchy sequences (nested intervals): \\
Every sequence of interleaved intervals $(A_k)$ with $A_0\supseteq A_1\supseteq A_2 \supseteq ... $ converges towards $A=\bigcap_k^\infty A_{k} $\\
Consequence: all (?) basic operations (and usual standard functions) are (piece-wise) continuous in $IR$. \todo[inline]{define a good symbol as palceholder for all operations?}
To conduct the proof on lower-/upper boundaries of the intervals.
Consequences
\begin{itemize}
	\item $\abs{A}=d(A,[0,0])=\max{\abs{a},\abs{a}}=max\{\sup\limits_{a\in A}\{d(a,0)\}=\max\{\sup\limits_{a\in A}\{\abs{a}\}$ \todo[inline]{again min and max for $a$?}
	\item $A\subseteq B \implies \abs{A}\le \abs{B}$
\end{itemize}
\begin{proposition}
	Properties of the distance:
	\begin{enumerate}
		\item $d(A+B,A+C)=d(B,C)$ 
		\item $d(A+B,C+D)\le d(A,C)+ d(B,D)$
		\item $d(a\cdot B, a\cdot C)=\abs{A}\cdot d(B,C)$
		\item $d(AB,AC)\le \abs{A}\cdot d(B,C)$
	\end{enumerate}
\end{proposition}
\begin{proposition}
	Properties of the diameter:
	\begin{enumerate}
		\item $\diam(A)=a-a\ge 0$
		\item defined point interval $X$ through $\diam(X)=0 \iff x=x$
		\item $A\subseteq B \implies \diam(A)\le \diam(B)$
		\item $\diam(A\pm B)=\diam(A)+\diam(B)$ 
		\todo{Here the signs +, - should be above each other}
		\item $\diam(AB)\le \diam(A)\cdot\abs{B}+\diam(B)\cdot\abs{A}$
		\item $\diam(AB)\ge \diam(A)\cdot\abs{B}+\diam(B)\cdot\abs{A}$
		\item $\diam(a\cdot B)=\abs{a}\cdot \diam(B)$
		\item \begin{align*}
			\diam(A^n)\le n\cdot\diam(A)\cdot \abs{A}^{n-1}, (n\in N), (A^n:=A\cdot A\cdot ...) \tag{$\ast$}\label{eq_prop_diam}
		\end{align*}
		\item
		\begin{align*}
			\diam((A-x)^n)\le(\diam(X))^n, \forall x\in X
			\tag{$\#$}\label{eq_prop_diam2}	
		\end{align*}
		\item $\abs{C}\le\diam(C)\le 2\cdot , \with O\in C$ \todo[inline]{Is this really a $O$ here?}
	\end{enumerate}
\end{proposition}
\begin{proof}
	\begin{itemize}
		\item We prove \eqref{eq_prop_diam2} and have $A,B\in IR$:\\
		$n=1 \implies \diam(A^1)=\diam(A)\le 1\cdot \abs{A}^0\cdot\diam(A)$ \\
		Induction hypothesis: The claim holds for $n, n>1$ \\
		Induction step: $\diam(A^{n+1})=\diam(A^n\cdot A)\le \diam(A^n)\cdot\abs{A}\cdot\diam(A)\cdot\abs{A^n}$ \\
		(?)$\le n\cdot$ \todo[inline]{What is missing here for the?} $\abs{A}^{n-1}\cdot\diam(A)\cdot|A|+|A^n|\cdot\diam(A)$ \\
		$=(n+1)\cdot|A^n|\cdot\diam(A)$ \\
		\item $\diam(A)=\abs{A-A}$
		\item $A\subseteq B \implies \frac{1}{2}\cdot \diam(B)\cdot\diam(A))\le d(A,B)\le diam(B)\cdot diam(A)$
		\item  $A \subseteq B\iff b\le a\le a\le b$ \\
		$\implies d(A,B)=\max\{|a-b|,|a-b|\}=\max\{a-b,b-a\}$ \\
		$\le b-a+a-b=b-b-(a-a)=diam(B)-diam(A)$ \\
		$\implies d(A,B)=\max\{a-b,b-a\}\ge \frac{1}{2}(a-b+b-a)$ \\
		$=\frac{1}{2}\cdot(diam(B)-diam(A))$
	\end{itemize}
\end{proof}
Further operations in $IR$ \\
\begin{*definition}[set-intersection]
	The Precondition is: $A,B\in IR$ are \emph{not disjoint}. Then
	\begin{align*}
		A\cup B:=\{c \mid c\in A \land c\in B\}=[\max{a,b},\min{a,b}]
	\end{align*}
\end{*definition}
\begin{proposition}[inclusion monotonicity property]
	We have 
	\begin{align*}
		A\subseteq C, B\subseteq D \implies (A\cup B)\subseteq(C\cup D)
	\end{align*}
\end{proposition}
\begin{proof}
	Without proof.
\end{proof}
\begin{definition}[interval hull]
	(in $IR$ convex hull)
	\begin{align*}
		A\cup B:=[\min\{a,b\},\max\{a,b\}]
	\end{align*}
	(?) \todo[inline]{Here i dont know how to write the sign between A and B}
\end{definition}

\subsection*{26.11.19}
Proof for ($\#$) \todo{Fix references}
\begin{align*}
	c\le 0\le c \implies \diam(C)=c-c=|c|+|c|\ge \max\set{c,c}=|c|
	\intertext{(?)\todo{?} Im not sure if here its big or small C and}
	\diam(C)=|c|+|c|\le 2\cdot \max\set{|c|,|c|}=2|c|
\end{align*}
Properties with a symmetric interval $A=-A=[-a,a]$
\begin{itemize}
	\item $A\cdot B=|B|\cdot A$
	\item $diam(A,B)=|B|\cdot diam(A)$ 
\end{itemize}
\begin{itemize}
\item Subdistributivity (distributivity) \\
In $IR$ the subdistributivity holds $A*(B+C)\le A*B+A*C$ \\
The distributivity holds, if either a) or b) hold: \\
\begin{enumerate}
	\item $A=[a,a]$ is a point interval\\
	\item $B*C\ge 0$, i.e. either: $(B\ge 0 \land C\ge 0)$ or $(B\le 0 \land C\le 0)$
\end{enumerate}
\begin{proof}
	\begin{enumerate}
		\item $a\cdot (B+C):=\{a*(b+c)| b\in B, c\in C\}$\\
		$a\cdot B+ac\dot C:=\{ab| b\in B\} \cup \{ac|c\in C\}=\{ab+ac| b\in B,c\in C\}$ \\
		$\implies a\cdot (B+C)=a\cdot B+ac\dot C$
		\item If $B\ge 0 \land C\ge 0: B+C\ge 0, B*C\ge 0$ (products are $\ge 0$)
		\begin{itemize}
			\item 1st case: $A\ge 0 \iff a\ge 0$: \\
			$A*(B+C)=[a\cdot(b+c),a\cdot(b+c)]$ \\
			$A*B+A*C=[a\cdot b+a\cdot c,a\cdot b+a\cdot c]$ \\
			$\implies A*(B+C)=A*B+A*C$
			\item 2nd case: $A\le 0 \iff a\le 0 $ (all products $\le 0$) \\
			$A*(B+C)=[a\cdot(b+c),a\cdot(b+c)]$ \\
			$A*B+A*C=[a\cdot b+a\cdot c,a\cdot b+a\cdot c]$ \\
			$\implies A*(B+C)=A*B+A*C$
			\item 3rd case: $a\cdot a<0 \iff a<0<a \iff 0\in A \to \int(A)$ \\
			$A*(B+C)=[a\cdot(b+c),a\cdot(b+c)]$ \\
			$A*B+A*C=[a\cdot b+a\cdot c,a\cdot b+a\cdot c]$ \\
			$\implies A*(B+C)=A*B+A*C$ \\
			If $B\le 0 \land C\le 0: B+C\le 0, B*C\ge 0$
			\begin{itemize}
				\item 1st case: $A\ge 0 \iff a\ge 0$: \\
				analogously to the 1st case from above, with switching all boundaries \\
				$A*(B+C)=[a\cdot(b+c),a\cdot(b+c)]$, (products $\le 0$) 
				\item 2nd case: $A\le 0 \iff a\le 0$
				analgously to the 1st case from above with switching the boundaries
				$A*(B+C)=[a\cdot(b+c),a\cdot(b+c)]$, (products $\ge 0$)
				\item 3rd case: $A\ni 0:$ analogously to case 3 from above with switching the boundaries (ALL) \\
				$A*(B+C)=[a\cdot(b+c),a\cdot(b+c)]$
			\end{itemize}
		\end{itemize} 
	\end{enumerate}
\end{proof}

\item Interval vectors (interval boxes) \\
$IR^n =$ set of vectors with $n$ components in $IR$ ($IC^n$ analogously)
\begin{align*}
V\in IR^n: V=
	\begin{bmatrix}
		v_1 \\
		. \\
		. \\
		.\\
		v_n
	\end{bmatrix}= 
	\begin{bmatrix}
		[v_1,v_1] \\
		. \\
		. \\
		.\\
		[v_n,v_n]
	\end{bmatrix} \quad \text{i.e. } V_i=[v_i,v_i] \in IR.
\end{align*}
alternative portrayal as pair of point vectors with components being all lower boundaries, i.e. all upper boundaries
\begin{align*}
	(V_1,V_2 \in R^n \\
	V=[V,V]=[V_1,V_2]=\sqbrackets{
	\begin{bmatrix}
		v_1 \\
		 . \\
		  . \\
		   .\\
		    v_n
	 \end{bmatrix}-
	 \begin{bmatrix}
	 v_1 \\
	  . \\
	   . \\
	    .\\
	     v_n
	 \end{bmatrix}}
\end{align*}
Geometrically we also work with achsis (?)\todo{?} quaders with $Z^n$ edge points $\in \{v=(v_1,..,v_n)^t\in R^n$ with $V_i\in\{v_i,v_i\}, \forall i=1,..,n\}$ \\

Important properties of the interval arithmetics 
\begin{enumerate}
\item Inclusion property
\begin{enumerate}
	\item arithmetic operations
		\begin{align*}
			\circ \in\{+,-,*,/\}: A\subseteq A', B\subseteq B' \implies A\circ B\subseteq A' \circ B'
		\end{align*}
	\item elementary standard functions
		\begin{align*}
			i \in \{\sin, \cos, \tan, \exp, \log, \sqrt{},..\}:A\subseteq A' \implies f(A)\subseteq f(A')
		\end{align*}
	\item also for arbitrary functions which are given with calculation rules (formulas) ($\to$ composition of the basic operations and functions) $\to$ computational graph; \\
	Input: parameter values $\to$ output: function values \\
	Graph: (Here comes the graph) \\
	$\implies$ the inclusion property is through everz computational graph propagandized \\
	$\implies$ inclusion property $\implies$ guaranteed inclusion of the solution(set). 
\end{enumerate}
\item Overestimation of the solution set: occurs mainly as a consequence  of the fact that the same variable   occurs repeatedly in a \todo{?}Ber(?)-graph because everytime through the set based definition of the interval arithmetics, the elements are seen decoupled from one another in a variable interval.
\begin{enumerate}
	\item An example: $X-X=[-diam(X),+diam(X)]\to$ strech as much as possible, and replace through O?
	$\implies$ Recommendation: To keep the number of (?)\todo{?} of each variable in the calculation rule as much as possible. For example, by factoring out common subexpression (example: distributive law).
	\item further countermeasures: interval subdivision of parameter intervalls and individual evaluation of the individual sub-intervals, to take interval hull of this enclosure
	\item use of smoothness characteristic of the calculation rule, ie use of derivatives (mind $f'$) and estimation of the value range (?)\todo{(?)}
\end{enumerate}
\item Wrapping-effect (in multidimensional, $IR^n$) \\
An example: $f:R^2 \to R^2$, rotation around the angle $\phi$ \\
$f(X)=A\phi\cdot X$ with $A=\begin{bmatrix}\cos(\phi) \sin(\phi)\\ -\sin(\phi) \cos(\phi) \end{bmatrix}$ \\
repeated use: $X_{i+1}:= f(X_i)=A\phi\cdot X_i=A\phi\cdot x_i$ \\
$=A\phi\cdot A\phi\cdot X_{i-1} = \to = A\phi^{(i+1)}\cdot x_0$ \\
$=\begin{bmatrix}\cos(\ii \phi) &\sin(\ii \phi)\\ -\sin(\ii \phi) &\cos(\ii \phi) \end{bmatrix} \cdot x_0$ \\
In $IR: $ square $X_0:=\begin{bmatrix}[-\epsilon, +\epsilon]\\ [-\epsilon, +\epsilon] \end{bmatrix}, \epsilon>0 $ \\
$X_1=
\begin{bmatrix}[-\epsilon, +\epsilon]\cdot \cos(\phi) &[-\epsilon, +\epsilon]\cdot \sin(\phi)\\ 
[-\epsilon, +\epsilon]\cdot (-\sin(\phi))& [-\epsilon, +\epsilon]\cdot \cos(\phi) \end{bmatrix}$ \\
$=(|\sin(\phi)|+|\cos(\phi)|)\cdot ([-\epsilon,+\epsilon],[-\epsilon,+\epsilon])^T$ \\
\\
$X_n= (|\sin(\phi)|+|\cos(\phi)|)^n \cdot X_0$ the diameter grows exponentially!
\todo{Here comes the graph}
\end{enumerate}
\end{itemize}

\subsection*{2.12.19} 
\begin{itemize}
\item Extension of the number field $R$ \\
$\Omega:=\{-\infty, +\infty\}$, $R^*:=R \cup \Omega$ \\
$IR:=\{[x,y]|x\le y, x,y\in R\}$ \\
$IR^*:=IR\cup \{(-\infty,y]|y\in R\}\cup \{[x,+\infty)|x\in R\} \cup \{(-\infty, +\infty)\}$, \\
where \\
$(-\infty,y]:=\{a\in R|a\le y\}$ \\
$[x,+\infty):=\{a\in R|a\ge y\}$ \\
\item extended interval division $X/Y$ with $0\in Y$ \\
$Y=Y_1\cup \{0\}\cup Y_2$ with $Y_1=[y,0)$ and $Y_2=(0,y]$ \\
$X/Y:=X/Y_1 \cup X/Y_2 =\{x/y|x\in X, y\i Y_1\} \cup \{x/y| x\in X, y\in Y_2\}$ \\
\todo{Here the things under $X/Y_1 \cup X/Y_2$ are missing!}
\begin{align*}
1/Z=1/Z_1 \cup 1/Z_2\cup \{0\}=[-\frac{1}{4},0)\cup (0,\frac{1}{2}]\cup \{0\}=Y/X
\end{align*}
For $0\in Y: $ \\
\[
	X/Y=
	\begin{cases}
		(-\infty,+\infty) & \text{if } X=0 \vee 0\in X \vee Y=0\\
		(-\infty, x/y]\cup [x/y,+\infty) & \text{if } X\le 0 \wedge y\le y=0 \\
		(-\infty, x/y] & \text{if } X\le 0 \nd 0=y< y 
	\end{cases}
\]
Here just the circle above the $X$ is missing in the first case \\
\[
	X/Y=
	\begin{cases}
		(-\infty,X/Y]\cup[X/Y,+\infty), & \text{if } 0\le X \nd 0\in Y \\
		(-\infty, x/y], & \text{if } 0\le x \text{ and } y<y=0 \\
		[x/y,+\infty), & \text{if } 0\le x \text{ and } 0=y<y 
	\end{cases}
\]
Here the circle above $Y$ in the first case is missing \\
For $0\not \in Y: 1/y=[\frac{1}{y},\frac{1}{y}]$ \\
For $0\in Y: 1/y=(-\infty,\frac{1}{y}]\cup [\frac{1}{y},+\infty)$\\
$\frac{1}{1/y}=[\frac{1}{1/y})\cup \{0\}\cup (0,\frac{1}{1/y}]=[y,y]=Y$\\
\item $W_i(X)\subseteq f(X)$ \\
Assumption: $f:J\to R, J=[a,b]\subset R, f\in C^1(J)$ \\
For arbitrary intervals $x\subseteq J$ it holds: \\
For a $z\in X$ (here a circle above $X$) (eg. $z=x=\mid(X)$)per mean value theorem there is a point $\xi\in X$ $\forall x\in X$ with: \\
$f(x)=f(z)+f'(\xi)(x-z) \in f(z)+f'(X)(X-z)=:f_m(X)$ \\
$f_m$ is called a mean value form or centered form. \\
Note: If $\cup_{k=1}^n X_k=X\subseteq J$ a decomposition of $X$ in part-intervalls , then there are constants $c_1,c_2 \in R_+$ which are only dependant of $f$ and $J$, so that the distance \\
$d_1:=d(W_f(X), \cup_{k=1}^n f(X_k))\le c_1\cdot max_k (\diam(X_k))$ \\
while the distance \\
$d_2:=d(W_f(X),\cup_{k=1}^nf_m(X_k))\le c_2\cdot max_k(\diam^2(X_k))$ \\
$d_1\implies $ overestimation decreases linearly \\
$d_2\implies$ overestimation decreases quadratically.
\item Classic Newton-method:
$f\in C^1(\xi)$, start point/- approximation $X_0$ \\
$X_{n+1}=X_n-\frac{f(X_n)}{f^\prime(X_n)}, n=0,1,2,..$ \\
local linearisatin of the function $f$ at point $(x_0,f(x_0))$ \\
In general the Newton's method is not convergent. \\
If it converges (towards a zero of the function, which is a eventually non near the starting point $x_0$ lying point), then the convergence rate is quadratic, i.e. the error is squared in each iteration, i.e. the number of correct points is doubled.
\item Interval- Newton- method:
$f\in C^1(J), J=[a,b]\subseteq R$ start-/search interval $X_0=[x_0,x_0]$ \\
Wanted: Zero(s) of $f$ in $X_0$, i.e. $x^*\in X_0$ with $f(x^*)=0$ \\
For every $x\in X_0, x\not = x^* \exists \xi \in X_0$ (between $x$ and $x^*$), with: \\
$f(x^*)-f(x)=f^\prime(\xi)\cdot (x^*-x)$ \\
$0=f(x^*)=f(x)+f^\prime(\xi)\cdot (x^*-x)$ \\
$x^*=x-\frac{f(x)}{f^\prime(\xi)} \in x-\frac{f(X)}{f^\prime(X)}$ \\
\\
Interval-Newton Operator: \\
$N(f,x,X):=x-\frac{f(x)}{f'(X)} \implies $ Interval -Newton Iteration $\implies$ Interval nesting \\
with $x\in X$ (Here signs above both $x$) \\
\\
$X_{n+1}:=N(f,x_n,X_n)\cup X_n\le X_n$ \\
\\
Note: 
\begin{enumerate}
\item Every zero of $f$ in $X$ has to be in $N(f,x,X)$ for $x\in X$ 
\item One obtains interval nesting through $\cup X_n$ \\
$\implies$ convergence towards the zero (assumption: $0\notin f'(X)$) 
\item If $x$ is chosen to be a midpoint of $X$, then the convergence is better than the bisection method \\
Here comes the graph \\
One can show that the convergence ( as in the classic Newton-method) is quadratic. ($0 \notin f'(X_0)$). \\
\\
3 cases: (occur in the iteration) with $0\not \in f'(X)$
\begin{enumerate}
\item $N(f,x,X)\subseteq X \implies \exists!$ zero of $f\in N(f,x,X)\subseteq X\subseteq X_0$ 
\item The intersection is empty: $N(f,x,X)\cup X= \empty$ \\
$\implies$ there does not exist a zero in the start interval $X_0$
\item Otherwise: the iteration has to be continued until case 1 or case 2 occurs. 
\end{enumerate}
%\todo{In practice, this results in the typical interval methods result, ie, a fining intervals is numerically impossible and 1) or 2) have not been met $\implies$ outlay: zero possible, but not verified.}
\end{enumerate}
\item Transition to floating-point: $IR$ to $IR$
	\begin{align*}
		IR:=set{X=[x,x]\mid x\le x, x,x\in R =\text{  floating point (?)}}
	\end{align*}
	\todo{missing!}
\item Interval rounding: $\Diamond : IR \to IR, \Diamond X:=[\nabla x, \triangle x] \supseteq X$ \\
Properties: \\
$\Diamond X=X, \forall X\in R$ (R1) \\
$X<Y \implies \Diamond X\le \Diamond Y$ for $X,Y\in IR$ (R2) \\
$\Diamond(-X)=-\Diamond X \forall X\in IR$ (R3) \\
Proof for (R3): \\
With (R4) with the directed roundings $\triangle(-x)=-\nabla x$ \\
$\Diamond(-X)=\Diamond (-[x,x])=\Diamond[-x,-x]=[\nabla(-x),\triangle(-x)]=[-\triangle x,-\nabla x]=-[\nabla x,\triangle x]=-\Diamond X$
\item Properties of the floating point interval arithmetics: \\
$+,\ast$ are commutative, but not associative! \\
Example: $\epsilon:=$ \person{Wilkinson}-$\epsilon=b^{1-l}$
\begin{align*}
	([-\frac{\epsilon}{2},\frac{\epsilon}{2}]\Diamond[-\frac{\epsilon}{2},\frac{\epsilon}{2}])\Diamond \frac{3}{2}=[-\epsilon,+\epsilon] \Diamond \frac{3}{2} =[\frac{3}{2}-\epsilon, \frac{3}{2}+\epsilon]
\end{align*}
\begin{align*}
	[-\frac{\epsilon}{2},\frac{\epsilon}{2}]\Diamond([-\frac{\epsilon}{2},-\frac{\epsilon}{2}]\Diamond \frac{3}{2})=[-\frac{\epsilon}{2},\frac{\epsilon}{2}]\Diamond [\frac{3}{2}-\epsilon , \frac{3}{2}+\epsilon]= [\frac{3}{2}-2\epsilon, \frac{3}{2}+2\epsilon]
\end{align*}
(rounding!)
\\
$\exists$ neutral elements $0=[0,0], 1=[1,1]$ with $x+0=0+x=x, 1*x=x*1=x$ \\
In general there does not exist inverse elements (as already in $IR$) \\
\\
Subdistributivity applies almost always, factoring out common subexpressions should be, however, always put in practice by reducing the (?) \\ 
\todo{I dont understand the next sentence}
\\
\item Floating point numbers:  \\
$\ulp:=b^{e_x-l}$ with $x=(-1)^{S_x}\cdot m_x \cdot b^e_x$ \\
with $l=$length of the mantissa $=$ number of digits to a base $b$ \\
succ$(1)=1+ulp(1)=1+\epsilon$, since $1=(-1)^0\cdot 0.100...0\cdot b^1$ and $\epsilon =ulp(1)=b^{1-l}$ \\
\\
Sometimes it is more practical to write floating point numbers in another form, namely as a integer multiple of ulp(x): \\
$x=(-1)^{S_x}\cdot m_x\cdot b^{e_x}=(-1)^{S_x}\cdot M_x\cdot b^{e_x-l}=(-1)^{S_x}\cdot M_x\cdot \ulp(x)$ \\

If $x$ is normalized, then it holds $\frac{1}{b}\le m_x<1$, i.e. $b^{l-1}\le M_x\le b^l-1, M_x\in N$ \\
\\
If we involve the sign in the mantissa $M_x$ and additionally allow the zero and the denormalized FPN then it holds: \\
$x=M_x\cdot \ulp(x)$ with $M_x\in Z, |M_x|\le b^l-1$ \\
In the following the exponent range is initially to be looked as unlimited) i.e. eventually occurring under or overflow is recognized at the very end, after rounding.
\item $4$ basic calculation ways in FPN (semimorph (?) computer arithmetic) \todo{something missing for ?}\\
\\
\todo{Here comes the graph}
\\
Semimorph C-arithmetics: $\circ\in \{+,-,*,/\}, \bigcirc\in \{\square,\triangle, \nabla, ..\}$ Here some signs are missing \\
Then \\
$x(?)y:=\bigcirc(x\circ y), \forall x,y\in R$ (floating point (?))\todo{find symbol}
Additional rounding for general basis $b\in set{2,3,4,..}$ \\
Here Im just going to write everything, you can just copy-paste it in the brackets as needed
Defined help-function: $x\in R, x>0 $ \\
$S_\mu(x):=\nabla x+\frac{\triangle x-\nabla x}{b}\cdot \mu $, \\
for $x\in R: S_\mu(x):= \nabla x+ulp(x)\cdot \frac{\mu}{b}$ 
\\
\todo{Is here something missing?}
with $\mu\in \{1,2,..,b-1\}$ \\
for $\mu=0$ ... Here write \\
for $\mu=b$ ... Here write \\
,i.e. ... Here write \\
\[
\square \mu(x):=
	\begin{cases}
		0, & \text{if } x\in [0,x_{min})=[0,b^{e-1})\\
		\nabla x,  & \text{if } x\in [\nabla x, S_\mu(x)), x>0\\
		\triangle x,  & \text{if } x\in [s_\mu(x), \triangle x), x>0\\
		-\square \mu(-x),  & \text{if } x<0 \text{(antisymmetric)}  
	\end{cases}
\]
An example: \\
$b=10, l=4: \square_(?) (0.5739; 5)=0.5739$ \\
$\square_(?) (0.5739; 6)=0.5740$ \\

\item 5 steps for a arithmetic floating- point operation:
\begin{enumerate}
\item decomposition of the operands  \\
$x\to (S_x,m_x,e_x)=(M_xl,e_x-l)$ \\
\item $y\to (S_y,m_y,e_y)=(M_y,e_y-l)$ \\
where, $M_x, M_y$ is favoured in 2er-complement 
\item Computation of a (approximated) intermediate result  \\
$z:=z\circ y \approx x\circ y = z$ \\
(with accuracy requirement that $Oz=Oz=$ result 
\item Normalising: \\
Adjust To eliminate the leading zero digits in the mantissa through $L(left)-Shift_k$ $\to$ (by k positions) and adjust the exponent (substract k) \\
or: $AR-Shift_1$, if a leading 1 has come as a carry before the mantissa. 
\item Rounding: \\
$z=x\circ y=Oz=Z(x\circ y)=O(x\circ y)$ \\
In rare cases (if the mantissa results in $m_z=1.00...0$) the result of the rounding must be normalized again, i.e. $ARShift_1$ yields $0.100...0$ and increase the exponent $+1$.
\item composition $(M_z, ex-l)\to (s_z,m_z,l_z)\to z$ \\
4 basic operations (arithmetically) \\
\begin{enumerate}
\item multiplication: $z=x\circ y$ \\
Case 1: $x=0 \lor y=0 \implies z=0$ \\
Case 2: $x\not = 0 \land y\not =0: $\\
$z=(-1)^{S_x+S_y}\cdot m_x\cdot m_y\cdot b^{e_x+e_y} $ \\
$=(-1)^{S_x+S_y}\cdot M_x\cdot M_y\cdot b^{e_x+e_y-2l} $ \\
$ =(-1)^{S_x+S_y}\cdot M_x\cdot M_y\cdot ulp(x)\cdot ulp(y) $ \\
\\
If $x,y$ are normalized: $B^{2l-2}\le M_x\cdot M_y< b^{2l}$ \\
$\implies$ $M_x\cdot M_y$ need $2l-1$ or $2l$ digits \\
Computation of the exact double-long product is necessary, so that one can round always correctly! \\
\todo{add itemize}
Case 1: $M_x\cdot M_y=b^{2l-1} (\implies 2l$ digits $)\implies ez:=ex+ey$ \\
Case 2: $M_x\cdot M_y<b^{2l-1} (\implies 2l-1$ digits $)$, then $m_x\cdot m_y<b^{-1}$ is not normalized $\implies$ $LShift_1$, i.e. $m_z=m_xm_y\cdot b$ (rounding) and $ez:=ex+ey-1$\\
Note: $AR_Shift =$ arithmetics (?) Shift \todo{? missing}
\begin{itemize}
	\item Rounding is always based on the first digit, which doesn not fit in the mantissa anymore (ie $l + 1$ significant digit plus information whether after that something else comes, which is not zero (ie rest from $l + 2$, digits $\not = 0$)
	\item Round digit = $l+1$ significant digit
	\item sticky bit $s=V_{i=l+2}m_i =$
	\begin{align*}
		s=V_{i=l+2}m_i =
		\begin{cases}
			0, & \text{if rest}=0\\
			1,  & \text{otherwise}
		\end{cases}
	\end{align*}
	$0.m_1m_2 \cdot \cdot \cdot m_l $\\ 
	\todo{Here idk how to write this.}
	This holds especially for our roundings :
	\todo{write the roundings?}
\end{itemize}
\item Division: $z=x/y$ with $y\not=0$ [otherwise$\to$ error] \\
Case 1: $x=0 \implies z=0$ \\
Case 2: $x\not = 0 (y\not =0):$ \\
$z=x/y-(-1)^{S_x+S_y}\cdot \frac{m_x}{m_y}\cdot b^{e_x-e_y}=(-1)^{S_z}\cdot \frac{M_x}{M_y}\cdot b^{e_x-e_y}$, with $S_z:=MOD(S_x+S_y,Z)$ and here the approximation is missing 
\\
$1/b<m_z^*<b$ with normal $x$ and $y$.\\
\begin{itemize}
	\item If $m_x\ge m_y \iff m_z^*\ge 1$, then the result has to be shifted with $AR-Shift_1$ (or to write the (HW?) of the first digits on the "correct" position of the mantissa and to increase it by 1 $\implies$ normalisation is not necessary
	\item If $m_x<m_y \implies m_z^*<1$, then adjusts the position of the digits 
\end{itemize}
\end{enumerate}
Implementation of the mantissa division manually: \\
with $x>0, y>0:$ \\
0. accumulator: $A:=x$ \\
1. for $i=1,..,l+1$
\begin{itemize}
	\item Compute the quotient digit $q_i$ from left to right. In addition to that one has to ... the maximal multiple of $y$ by disconnecting the multiples of, that can be pulled off yet from the accumulator A.
	\item $A:=A-q_i\cdot y\cdot b^{-i}$ 
\end{itemize}
2. Quotient $z=0.q_1q_2...q_l \mid q_{l+1} \to m_z^*$ with rest $R=A_{l+1}$ (last akku-content) \\
3. Rounding: with rounding digits $r=q_{l+1}$ and sticky bit \\
\begin{align*}
	s =
	\begin{cases}
		0, & \text{if R}=0\\
		1,  & \text{if R}\not= 0
	\end{cases}
\end{align*}

\item Addition and substraction (4) $z=x+-y$ \\
substraction: build 2er complement of $y$ and add\\
assumption: $e_x\ge e_y$ and $x\neq 0$ and $y\neq 0$ (or stricter: $|x|\ge |y|$) \\
special cases are not going to be considered separately \\
\\
$e_x\ge e_y$ if this doesnt hold, then switch $x$ and $y$ \\

\\
Mantissa obtain here also the sign (2er complement/ $b-$complement konchierung?) \\
- exact result: $z=x+y$\\
- intermediate result: $z=z+y$ (enough for correct rounding, so that the floating point result: $z:=O(z)=O(z)$ ...? \\
\\
General assumption $x\not=0, y\not=0, |x|>|y|$ \\
Algorithm with long accumulator: with $(2l+1)$ digits (to a base $b$) and $1$ carry-out-bit. (Here comes the drawing) \\
\\
Case 1: Exponent difference $d:=e_x-e_y\ge l+2$ \\
$b=10, l=6$ \\
\todo{DRAWING}
\\
Rounding: \\
a) $\square\mu (\mu\in \{1,..,b-1\})$ + drawing \\
b) the signs, $m_z=b^{-(l+2)}\cdot (b^{l+1}\cdot m_x+sign(y)$, where 

\begin{align*}
	sign(y) =
	\begin{cases}
		1, & \text{if } y>0\\
		0,  & \text{if} y=0\\
		-1,  & \text{if} y<0
	\end{cases}
\end{align*}
$z=x+b^{-2}\cdot \ulp(x)\cdot sign(y)$ \\
rounding yields $\pred(x)$ or $x$ or $\succ(x)$\\

\\
Case 2: $d=e_x-e_y=l+1$ \\
\todo{DRAWING} \\
\\
Rounding: $\nabla$, $\square \mu (\mu=6,..,10)$, result + drawing \\
$\square, \square_{IEEE}, \triangle, \square\mu(\mu=0,..,5)$, result + drawing \\ 
b) \todo{DRAWING} \\
Rounding: $\square, \square_{IEEE}, \nabla, \square\mu(\mu=5,..,10), \triangle, \square\mu (\mu=0,..,4)$, result + drawing \\
\\
$\implies$ Idea: short accumulator with $1$ bit $+l z_i+1\cdot z_i+1\cdot z_i+1$bit \\
\\ here the things under the line are missing \\
$\implies$ $l+2$ digits $+2$ bit \\
(for $b=2$ i.e. $l+4$ bit) \\
\end{enumerate}
\end{itemize}

