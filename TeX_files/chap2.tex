% !TeX spellcheck = en_US
\textbf{28.10.2019}\\
A floating point number is 
\begin{align*}
	x = \begin{cases}
	= (-1)^{S_x}\cdot m_xb^{e_x}=(S_x,m_x,e_x) &\quad \\
	0 \quad
	\end{cases}
\end{align*}
with sign bit (Vorzeichen-bit) $S_x\in \{0,1\}$, mantissa $m_x=0.m_1m_2..m_l$ and exponent $e_x\in \{\emin,e+1,\dots,\emax\} \with \emin \approx -\emax$. The mantissa digits are $m_i\in S-\{0,1,..,b\}$ with $b:=b-1$ and mantissa length $l\in N^+$ ($b =$ base, $\emin =$ emin, $\emax =$ emax and $\N^+ = \N\setminus \set{0,1}$).
\begin{itemize}
    \item  Floating point format $R=R(b,l,\emin,\emax,\denorm)$
    with
    \begin{align*}
    	\denorm = \begin{cases}
    		\true &\quad\text{if denormalized FPN is allowed}\\
    		\false &\quad\\
    	\end{cases}
    \end{align*}
  \item normalized FPN $x\neq 1$: has as first mantissa digit $m_1\neq 0$ $(b=2 \implies m_1=1!)$
  \item unnormalized FPN $x\neq0$ has $m_1=0$
  \item Normalization of a FPN $x \neq 0$: mantissa digits to push for $k$ digits to the left ($k =$  number of leading 0-digits) and subtracting $k$ from $e_x$. $e_{x_neu}=e_x-k$.
  \newline Normalization is only possible if $e_{xneu} = e_x-k \geqslant e$
  \item denormalized FPN: $e_x = e$ and mantissa not normalized or can not be normalized
  \item number line is symmetrical to zero! Here its supposed to be the line.
  \item \begriff{\person{Wilkinson}-epsilon}
  \begin{align*}
  	\epsilon:=b^{1-l}=\frac{1}{b^{l-1}}
  \end{align*}
  is the biggest relative gap between two neighbouring normalized FPN $\sim$ the biggest relative error in solving 
  %TODO add example!
  \item \begriff{unit in last place} ulp:
  \begin{align*}
    \ulp(x) := \begin{cases}
    	b^{e_x-l} &\quad\text{if x is normal}\\
    	b^{e-l} &\quad\text{if x is denormalized or }0
    \end{cases}
  \end{align*} 
 ``unit in the last place'' (at the mantissa of $x$. 
  \begin{align*}
  	x=0.m_1 m_2...m_l\cdot b^{e_x} \with m_l = b^{e_x - l}
  \end{align*}
  \item \begriff{successor} ($k \in \Z$)
  \begin{align*}
  	\success(x) := \begin{cases}
  		x+\ulp(x) &\quad \text{if } x\neq -b^k\\
  		x+(\frac{\ulp(x)}{b}) &\quad \text{if } x=-b^k
  	\end{cases}
  \end{align*}
  \item \begriff{predecessor} ($k\in \Z$)
  \begin{align*}
  	\pred(x) := \begin{cases}
  		x-\ulp(x)& \quad\text{if } x\neq b^k\\
  		x-(\frac{\ulp(x)}{b})& \quad\text{if } x=b^k	
  	\end{cases}
  \end{align*}
  \begin{tabular}{l|r} %{\textwidth}
  	%\hline
  	\textbf{$x$}   & $\digits_b(x)$\\ \hline
  	0 & 1\\
  	$b^0 = 1$ & $\vdots$ \\
  	$\vdots$ & $\vdots$ \\
  	$b-1$ & 1\\
  	$b^1 = [10]_b$ & 2\\
  	$\vdots$ & $\vdots$\\
  	$b^2 -1$ & 2\\
  	$b^2 = [100]_b$ & 3\\
  \end{tabular}
	where we have base $b \in \N\setminus \set{0,1}$ and $\digits_b(x) = \floor{ \log_b(x)} +1$
\end{itemize}
\subsection*{Interval arithmetics (Wrap-around) in two-part complement}
	\begin{itemize}
		\item exact result: $z^* =x\frac{\pm}{\ast} y$, $x,y\in I_n$ (Integer with $n$ bits inclusive sign)
		\item generated result: 
		\begin{align*}
			z:=((z^*+z^{n-1})\mod2^n)-2^{n-1}\in I_n
		\end{align*}
		\item for normal mantissa $m_x$ it holds: $\frac{1}{b}\leqslant m_x<1$
		\item  for normal mantissa $m_x$ it holds: $0\leqslant m_x<\frac{1}{b}$
	\end{itemize}

\begin{itemize}
    \item 2 equivalent display possibilities for FPN $x\neq 0:$
    \begin{align*}
    	x=(-1)^{S_x}\cdot m_x\cdot b^{e_x}=(-1)^{S_x}\cdot M_x\cdot b^{e_x-l}\\
        M_x=m_x\cdot b^l \in \N \with b^{l-1}\leqslant M_x<b^l
    \end{align*}
    Optionally one could also add $M_x \implies \abs{M_x}<b^l$, $M_x\in \Z$\\
    $\implies$ Every FPN $x$ is an integer multiple of its $\ulp$`s.
    \item $\ulp(1)=\epsilon$, since $1=0,10\dots0\cdot b^1$ and $\epsilon=b^{1-l}=\ulp(1)$ $x\not =0, x\in R \leftarrow$ grid: 
    \begin{itemize}
        \item $\success(x)\leqslant x(1+\epsilon)=x(1+\ulp(1= = 1\cdot \succ(1)$
        \item $\success(x)\geqslant x(1-\epsilon)=x(1-\ulp(1)=x\cdot \pred(1)$
    \end{itemize}
    \item relative distance between 2 neighboring FPN:
    %TODO add graph 
   \end{itemize}
    \begin{definition}[Rounding]
    	Rounding is a map $o:\R\to R $ FP-grid with the following properties
    	\begin{defenum}
    		\item ($R1$) $ox=x, \forall x\in R$ (projection) \label{def_rounding_1}
    		\item ($R2$) $x,y \in R: x\leqslant y \implies ox\leqslant oy$ (monotonicity) \label{def_rounding_2}
    		\item ($R3$) $o(-x)=-ox, \forall x\in R$ \label{def_rounding_3}
    	\end{defenum}
    \end{definition}
	%TODO add graph!
	\begin{remark}
		An antisymmetric rounding i guess additionally satisfies also \ref{def_rounding_3}.
	\end{remark}
	\begin{example}
		Established rounding possibilities are:
		\begin{itemize}
			\item $\square x$: \i to nearest floating point number $\to$
			\begin{itemize}
				\item error $\le \frac{1}{2}\ulp$
				\item (stochastically calculated: ``round to even'' $=$ to next FPN with even mantissa, i.e. end digit = 0 (binary)
				\item \emph{If} the value which is to be rounded lays exactly in the middle between 2 FPN
				\item satisfies \ref{def_rounding_3}): $\square (-x)=-\square x$
			\end{itemize}
		
			\item $\sqcup x$: truncation, rounding $\to 0$
			\begin{itemize}
				\item $\abs{\sqcup x} \le \le\abs{\sqcup x}$
				\item maximal error $< 1 \ulp$
				\item satisfies \ref{def_rounding_3}
			\end{itemize}
			\item $\sqcap x$: augmentation($=$ the, in absolute value, biggest FPN), 
			\begin{itemize}
				\item $\abs{x} \le \abs{\sqcap x}$
				\item maximal error $<1 \ulp$
				\item satisfies \ref{def_rounding_3}
			\end{itemize}
			\item $\rndup x$: upward rounding $\to$ $+\infty$
			\begin{itemize}
				\item $x \le \rndup x$
				\item maximal error $< 1 \ulp$
				\item satisfies \emph{not} \ref{def_rounding_3} 
			\end{itemize}
			\item $\rnddown x$: downwards rounding $\to -\infty$
			\begin{itemize}
				\item $\nabla x\le x$
				\item maximum error $< 1 \ulp$
				\item satisfies \emph{not} \ref{def_rounding_3}
				\item instead of \ref{def_rounding_3} the antisymmetry holds:
				\begin{align*}
					\rnddown(-x)=-\rndup x \Leftrightarrow \rndup(-x) = - \rnddown x
				\end{align*}
			\end{itemize}
		\end{itemize}
	\end{example}
\subsection*{Intervals}
	\begin{itemize}
		\item $I\R \set{X=[\minval{x},\maxval{x}] \mid \minval{x}\le \maxval{x}\nd \minval{x},\maxval{x} \in \R}$ , i.e. set of the bounded, closed, real intervals
		\item Floating-point intervals: $IR =  \set{X=[\minval{x},\maxval{x}] \mid \minval{x}\le \maxval{x}; \minval{x},\maxval{x} \in R}$ intervals with floating point boundaries
	\end{itemize}
\begin{definition}
	Interval rounding as map 
	\begin{align*}
		\rndinterval \colon \R \to R \with X=[\rndup x, \rnddown x]\subseteq X=[\minval{x},\maxval{x}]
	\end{align*}
	with properties
	\begin{defenum}
		\item $R1$ $\rndinterval X=X \quad\forall X\in \R$ \label{def_interval_rounding_1}
		\item $R2$ $X,Y\in R: X\subseteq Y \implies \rndinterval X\le \rndinterval Y$ (Inclusion isotonic) \label{def_interval_rounding_2}
		\item $R3$ $\rndinterval(-X)=-\rndinterval X$, hence $\rndup(-x)=-\rnddown(x)$, $\rnddown(-x)=-\rndup x \in R$ \label{def_interval_rounding_3}
	\end{defenum}
	Then holds:
	\begin{align*}
	\begin{matrix}
	X + Y &= [\minval{x} + \minval{y}, \maxval{x} + \maxval{y}]\\
	X - Y &= [\minval{x} - \maxval{y}, \maxval{x} - \minval{y}]
	\end{matrix} \quad \diam(X\pm Y) = \diam X + \diam Y
	\end{align*}
\end{definition}
% % % % % % % % % % % % % % % compling until here without problems

%\newline \textbf{11.11.19}
%\newline \begin{itemize}
%	\item Interval arithmetics $(A,B,C,X\in IR)$
%	\newline $A+B:=[a+b,a+b]$
%	\newline $-B:=[-b,-b]$
%	\newline $A-B:=[a-b,a-b]=A+(-B)$
%	\newline $A*B:=[$min $S_x$,max $S_x]$
%	\newline $1/B:=[1/b,1/b]$, $0\not \in B$
%	\newline $A/B:=[$min $S_x,$ max $S_x]$ for $0\not \in B, $A\B=A*1\B$
%	\newline For $0\in B=$int$B$: $1\B:=\frac{1}{[b,0)}\cup \frac{1}{(0,b]}$, inverse value mapping yields again $b$!
%	\newline General division $A\B$ with $0\inB$ in exercise 3
%	\item Transition to floating -point intervals $A,B \in IR$
%	\newline $A(?)B=[a\nabla b,a\triangle b]$ (Here the arrow?)
%	\newline $A$ is bad, since (?)
%	\newline $A(?)B=[a\nalba b,a\triangle b]$
%	\newline $A(?)B=[$min $S_{\nabla}$, max$S_{\triangle}]$ with $S_{\nabla}:=$, $S_\triangle:=$
%	\newline $A(?)B=[$min $S_{\nabla}$, max$S_{triangle}]\subseteq A(?)(1(?)B)$
%	\item Inclusion property:
%	\newline $A\subseteq A^\prime, B\subseteq B^\prime \implies A\cdot B\leqslant A^\prime \cdot B^\prime $
%	\newline $f(A)\leqslant f(A^\prime)$
%	\newline $g(A,B)\subseteq g(A^\prime, B^\prime)$
%	\item Overestimation 
%	\newline $X-X=[x-x,x-x]=[-$diam$(X),+$diam$(X)]$ with diam$X=x-x$ (diameter of $X$)
%	\newline $sin^2X+cos^2X=[(?),(?)]\ni 1$
%	\item Implementation of interval functions
%	\nwline Monotonicity is eventually piecewise necessary
%	\newline $s^X=[e^x,e^x]$, $\log_b X=[\log_b x, \log_b x]$, $\square(X)=[\square(x),\square(x)]
%	\newline to consider other (eg. trigonometric functions) functions: local minimum/maximum 
%	\newline Here the graph is missing
%\end{itemize} \bigskip
%
%\newline \textbf{12.11.19}
%\newline \begin{itemize}
%	\item Goal: reliable arithmetic ( and also standard-function) for usual space of numerical calculations
%	\newline $Z\supset Z_8\supset Z_4 \supset Z_2 \gets$ formally with 2 
%	\newline $R \supset R_10 \supset R_8 \supset R_4 \supset R_2$
%	\newline VR (vectors) $\supset $ VR
%	\newline MR (matrices) $\supset $ MR $\gets $ (?) R \to floating point format
%	\newline P(R) $\supset $ IR $\supset $ IR ...
%	\newline P(VR) $\supset $ IVR $\supset $ IVR ...
%	\newline P(MR) $\supset $ IMR $\supset $ IMR
%	\newline Here i dont know where to put the signs?
%	\newline analogously C, VC, MC, P(C)$\supset $IC, P(VC)$\supset $IVC, P(MC) $\supset $ IMC
%	\newline (Ulrich) Kuhlisch: not traditional(?) definition of arithmetic (i.e. all with help of real basic operations in R)
%	But rather: horizontal definition of arithmetic, separately for each level/ layer (with optimal rounding) 
%	\newline \implies so called SEMIMORPHE ARITHMETIC
%	\newline for each unit operation $\circ\in \{+,-, *,/,\cdot\}$, where $\cdot$ is a scalar product, the result should be calculated in the following way:
%	\newline AS IF the mathematical exact result was calculated, and then (with the (?) (?)) was rounded. ($\circ \in \{\square, \nabla, \triangle,..\}$)
%	\newline $\forall x,y\in $"nummerical space" it has to hold $x\circ y:=\circ(x(?)y)$, where $\circ$ is an operation rounding (computer operation) and (?) is an exact operation
%	\item Interval spaces
%	\newline \to $IR:=\{A=[a,a]| a\leqslant a, a,a \in R\}$
%	\newline \to $IC:=\{C=[c,c]| c\leqslant c, c,c\in C\}$, where the order relation $\leqslant$ is on $C$ in such a way defined, so that $a\leqslant b \iff $Re$(a)\leqslant $Re$(b) \land$ Im$(a)\leqslant $Im$(b)$ \smallskip
%	\newline Subset property:
%	\newline in $IR: A\subseteq B \iff b\leqslant b \land a\leqslant b$
%	\newline in $IC: $ for Re and Im simultaneously \smallskip
%	\newline Portrayal of complex intervals:
%	\newline $+$ either: rectangles in $C$:
%	$C=[c,c]:=\{c\in C| $Re$(c)\leqslant $Re$(c)\leqslant $$Re$(c) \land $ Im$(c)\leqslant$ Im$(c)\leqslant$ Im$(c), c,c\in C\}, c\leqslant c$
%	\neqline $[c,c]=[(a,b),(a,b)]=$pair complex numbers
%	\neqline $[c,c]=([a,a],[b,b])=$pair real intervals
%	\newline Here should come the graph
%	\newline $+$ as complex "circles" or even ellipse
%	\item In general the result of every operation (or function) $\circ: IR\times IR\xrightarrow{} IR, \circ\in\{+,-,*,/\}$ (or with $C$ instead of $R$) must be defined in the following way:
%	\newline $A,B\in IR ($ or $IC): A\circB:=$interval hull (So), with So$:=\{a\circb|a\in A, b\in B\}$
%	\newline in $IR: A\circB=[$minSo,maxSo$]$ and it suffices to consider the set So$:=\{a\circ b,a\circ b,a\circ b,a\circ b\}$
%	\item Algebraic properties (in $IR$)
%	\newline $+,*$ are commutative and associative and there exist neutral elements: $0=[0,0], 1=[1,1]$. but in general not inverse elements.
%	\newline $A+X=0, A+X=1$ not solvable, if diam$(A)>0$.
%	\newline Note:
%	\newline \to $A-A\ni 0$, but $A-A\not =0$ if diam$(A)>0$
%	\newline \to $A/A\ni 1$, but $A/A\not =1$ if diam$(A)>0$
%	\newline Distributive law does not hold, but the subdistributivity: $(A*(B+C))\subseteq A*B+A*C$
%	\newline An example: 
%	\newline $[0,1]*([1,1]+[1,1])=[0,0]=0$
%	\newline $[0,1]*1+[0,1]*(-1)=[0,1]+[-1,0]=[-1,+1]$
%	\newline \to $0\nsubseteq [-1,+1]$ \smallskip
%	\newline BUT $a*(B+C)=a*B+a*C, \forall a\in R$
%	\newline $A*(B+C)=A*C+B*C$ if $b*c\geqslant 0, \forall b\in B, c\in C$
%	\item Algebraic properties in $IC$
%	\newline Associative law does not hold
%	\newline An example:
%	\newline $A=[2,4]+i[0,0], B=[1,1]+i[1,1]=C$
%	\newline $(A\cdot B)\cdot C=([2,4]+i[2,4])\cdot C=[-2,2]+i[4,8]$
%	\newline $A\cdot (B\cdot C)=A\cdot ([0,0]+i[2,2])=[0,0]+i[4,8]$
%	\newline \implies $[-2,2]+i[4,8]\not =[0,0]+i[4,8]$
%	\newline Here if you can do an arrow from above to below with in-equal would be good.
%	\newline Inclusion property: $A\subseteq A^\prime , B\subseteq B^\prime \implies A\circ B\subseteq A^\prime \circ B^\prime$
%	\newline for functions: $A\subseteq A^\prime \implies f(A)\subseteq f(A^\prime)$
%	\newline or $A_i\subseteq A_i^\prime \forall i \implies f(A_1,..,A_n)\subseteqf(A_1^\prime,..,A_n^\prime)$\smallskip
%	\newline In composition of operations and functions to mathematical formulas and calculation rules, this property will remain valid. (?) I dont understand this next sentence: Es werden Einschließungen der Lösung berecht?
%	
%	\newline An example: $(X-2)^2-4$ for $X=[1,4]$
%	\newline $(X-2)^2-4=[-1,2]^2-4=[0,4]-4=[-4,0]$
%	\newline $(X-2)\cdot(X-2)-4=[-1,2]*[-1,2]-4=[-2,3]-4=[-6,0]$
%	\newline $X*(X-4)=[1,4]*[-3,0]=[-12,0]$
%	\newline $X*X-X*4=[1,(?)]-[4,16]=[-15,12]$
%	\newline \to $[-4,0]\subseteq [-6,0] \subseteq [-12,0] \subseteq [-15,12]$
%\end{itemize} \bigskip 
%
%\newline \textbf{18.11.19}
%\begin{itemize}
%	\item $A\circ B:=$ intervals $\{a\circ b| a\in A, b\in B\}=[$minSo, maxSo$]$, with So$=\{a\circ b,a\circ b,a\circ b,a\circ b\}$
%	\item In general $\nexists $inverse:
%	\newline If diam$(A)=a-a>0, $i.e. $a<a$:
%	\begin{enumerate}
%		\item $A+X=[0,0]=0 \iff a+x=0 \land a+x=0 \implies x=-a>-a=x$, i.e. $x>x$ is prohibited
%		\item $A*X=[1,1]=1 \iff a\cdot x=1 \land a\cdot x=1 \implies x=\frac{1}{a}>\frac{1}{a}=x$ (Here the sign at the end is missing)
%	\end{enumerate}
%	\item Kürzungsregel (in english?)
%	\begin{enumerate}
%		\item $A+X=B+X \iff A=B$
%		\item $A*X=B*X \iff A=B$ it holds $\forall X\in IR$ with $0\not \in X$
%	\end{enumerate}
%	\newline An example:
%	\newline $X=[-1,+1]: A=-1,B=+1: A*X=X=B*X=[-1,a]*X=[b,+1]*X, \forall a\in X, b\in X$
%	\newline $X=[0,1]: [0,2]*X=[1,2]*X=[2,2]*X=[0,2]$
%	\item Calculation example:
%	\newline $A=[-3,1], B=[2,4]$
%	\newline $A+B=[-1,5]$ (and diam$(A+B)=$diam$(A)+$diam$(B)$)
%	\newline $A-B=[-7,-1]$ (and diam$(A-B)=$diam$(A)-$diam$(B)$)
%	\newline $B-A=[1,7]=-(A-B)$
%	\newline $A-A=[-4,+4]$
%	\newline $B-B=[-2,+2]$
%	\newline $A*B=[-12,+4]$, S$_*=\{-6,-12,2,4\}$
%	\newline $A*A=[-3,+9]$, S$_*=\{+9,1,-3,-3\}$
%	\newline $B*B=B^2=[4,16]$
%	\newline $A^2=[0,+9]$
%	\newline $A/B=[-3/2,+1/2]$, S$_/=\{-3/2,-3/4,1/2,1/4\}$
%	\newline $A*1/B=[-3,1]*[1/b,1/b]=[-3/2,1/2]$
%	\newline $B/A=$ prohibited, since $0\in A$
%	\newline $B/B=[2/4,4/2]*[1/2,2]$, S$_/=\{1,2/4,4/2,1\}$
%	\item The value range of a function over an interval $X=[x,x]$
%	\newlie W$_f(X)=\{f(x)|x\in X\}$ \to interval hulls take hull(W$_f(X)$)$\in IR$
%	\newline An example: $X=[0,\pi/6]$
%	\begin{enumerate}
%		\item $f(x)=sin(x)$, $sin([0,\pi/6])=[0,\frac{1}{2}]$
%		\item $f(x)=cos(x)$, $cos([0,\pi/6])=[\sqrt{3}/2,1]$
%		\item $f(x)=tan(x)$, $tan([0,\pi/6])=[0,1/sqrt{3}] \not = \frac{sin(x)}{cos(x)}$, here "in general" is missing 
%		\item $f(x)=cot(x)$, $cot([0,\pi/6])=[\sqrt{3}, \infty]$
%		\item $f(y)=e^y$, $e^y=[e^y, e^y]$, here the line above/below $y$ is missing
%	\end{enumerate}
%	\item Interval evaluation (Interval-extension) of a function $f$ over an interval $X\in IR$
%	\newline Is obtained by substituting $X$ (as a set of real numbers) in each occurrence of the variable $x$ in the calculation rule
%	\newline $f(X)\supseteq $W$_f(X)$, usually with overestimation of the WB(?)
%	\newline Especially if $x$ occurs multiple times in the calculation rule
%	\newline An example 
%	\newline $+$ $f(x)=x-x \iff g(x)=0$
%	\newline $f(X)=[-$diam$(X),+$diam$(X)], q(X)=[0,0]$
%	\newline $+$ $f(x)=x*x \iff g(x)=x^2$
%	\newline $F(X)$ contains also negative values, if $x<0$ and $x>0$, 
%	\newline $g(X)=[-,($max$\{|x|,|x|\})^2]$
%	\newline $+$ $f(x)=sin^2x+cos^2x \iff g(x)=1$
%	\newline $f([0,\frac{\pi}{6}])=[0,\frac{1}{4}]+[\frac{3}{4}.1]=[\frac{3}{4},\frac{5}{4}]\ni 1$, $g([0,\frac{\pi}{6}])=[1,1]$
%	\item Solving the equality $A*X=B$ with $A,B,X\in IR, A\not =[0,0]$
%	\newline Help-function: 
%	$$
%	h(A):=
%	\begin{cases}
%	a/a, & \text{if } |a|\leqslant|a|\\
%	a/a, & \text{otherwise } \\
%	\end{cases}
%	$$
%	\newline SATZ(Ratschek):
%	\newline $AX=B$ has a (algebraic) solution $X\in IR$ if and only if $h(A)\geqslant h(B)$
%	\newline If $h(A)=h(B)\leqslant 0$, then the solution is not unique 
%	\newline An example
%	\begin{enumerate}
%		\item $A=[1,2], B=[-1,3]$, so: $[1,2]*[x,x]=[-1,3] \implies X_{general}=[-1/2,3/2]$ is unique, since $h(A)=1/2, h(B)=-1/3 \implies h(A)(?)h(B)$ \to $\exists$ uniqueness
%		\newline Here the sign between $h(A)$ and $h(B)$ is missing
%	\end{enumerate}
%	\newline BUT: the set of all point solutions for the equations $ax=b$ with $a\in A, b\in B $ is:
%	\newline $X_{set}=\{x=\frac{a}{b}| a\in A, b\in B\}=\frac{B}{A}=\frac{[-1,+3]}{[1,2]}=[-1,3]\supseteq X_{general}=[-\frac{1}{2},\frac{3}{2}]$
%	\newline \square In general, it holds: the set of all point-solutions $X_{set}=\frac{B}{A}\supseteq X_{general}$ (algebraic solutions)
%	\newline If $AX=B $ with $0\not \in A$ has the algebraic solution $X\in IR$, then $X_{general}\subseteq B/A$, since $\forall x\in X_{general}$ it holds $\forall a\in A: a\cdot x=b \in B$, while $X_{set}=B/A:=\{x=b/a| b\in B, a\in A\}$.
%	\newline An example 2)
%	\newline $A=[-1/3,1], B=[-1,2]$ so $[-1/3,1]\cdot X=[-1,2]$ \implies $x=[-1,2]$ is a unique solution
%	\newline $h(A)=-\frac{1}{3}, h(B)=-\frac{1}{2} \implies h(A)(?)h(B) \implies \exists$ a unique solution!
%	\newline (Here also its missing the sign between the $h(A)$ and $h(B)$.
%	\newline BUT:
%	\newline The set of the point-solutions yields $R$ since $\frac{B}{A}$ is a division through a null-interval. 
%\end{itemize} 
%
%%Definition: For $A\in IR$ the diameter is diam$(A):=a-a= \max\limits_{x,y\in A}|x-y|$
%%\newline .. is the radius $rad(A):=\frac{a-a{2}=\frac{diam(A)}{2}$
%%\newline .. is the midpoint: $mid(A)=a:=\frac{a+a}{2}=a+rad(A)=a-rad(A)$
%
%\textbf{25.11.2019} \\
%Definition: For $A\in IR$ the diameter is $diam(A):=a-a=\max\limits_{x,y\in A}|x-y|$ \\
%... the radius is: $rad(A):=\frac{a-a}{2}=\frac{diam(A)}{2}$ \\
%... the midpoint is: $mid(A)=a:=\frac{a+a}{2}=a+rad(A)=a-rad(A)$ \\
%... is the maximum of the absolute value \\
%$|A|:=\max\limits_{a\in A}|a|=max\{|a|,|a|\}$ \\
%... is the minimum of the absolute value:
%$$
%<A>=\min\limits_{a\in A}|a|=
%\begin{cases}
%min\{|a|,|a|\}, & \text{if } 0\not \in A\\
%0, & \text{if } 0\in A \\
%\end{cases}
%$$ \\
%(An alternative definition of the absolute value of an interval: $|A|^*\[<A>,|A|]$) \\
%
%Definition: The distance between two intervalss $A,B\in IR$ is $d(A,B):=max\{|a-b},|a-b|\}$ \\
%
%Note: For point-intervalss $A=[a,a], B=[b,b]$ this is a real distance $d(A,B)=d(a,b)=|a-b|$ \\
%
%Definition: (Alternative definition): $d(A,B):=max\{\sup\limits_{a\in A}\{d(a,B)\}, \sup\limits_{b\in B}\{d(b,A)\}\}$, with $\sup\limits_{a\in A}\{d(a,B)\}:=\sup\limits_{a\in A}\inf\limits_{b\in B}\}$ \\
%
%Definition (further alternative definition) 
%\begin{itemize}
%\item $d(A,B):=|mid(A)-mid(B)|+|rad(A)-rad(B)|$
%\item $d(A,B):=\inf\{q\in R_0^+| A\subseteq B+[-q,+q] \land B\subseteq A+[-q,+q]\}$
%\implies in every definition $d(A,B)$ is a metric. \\
%We use the first definition \\
%\cdot $d(A,B)\geqslant 0$ \\
%$d(A,B)= 0 \iff a=b \land a=b \iff A=B$ \\
%$d(A,B)\leqslant d(A,C)+d(B,C) \to (*)$ \\
%$d(A,B)=d(B,A)$ \\
%(*): $d(A,C)+d(B,C)= \max\{|a-c|,|a-c|\}+\max\{|b-c|,|b-c|\} \geqslant \max\{|a-c|+|b-c|,|a-c|,|b-c|\} \geqslant \max\{|a-b|,|a-b|\}= d(A,B)$ \\
%\end{itemize}
%
%$IR$ is with this metrix a complete metric space, i.e. every cauchy-sequence $(x_n)$ in $IR$ converges towards an interval $X\in IR$. \\
%
%$\forall \epsilon>0 \exists N_{\epsilon}\in N: d(X_n,X_m)<\epsilon \forall n,m \geqslant N_{\epsilon}$, then it holds \\ $\lim_{n\to\infty} x_n=x$ for a $X\in IR$ with $X=[x,x]$ \\
%$\lim_{n\to\infty} x_n=x \iff \lim_{n\to\infty} d(x_n,x)=0 \iff $(\lim_{n\to\infty} x_n=x \land \lim_{n\to\infty} x_n=x$). \\
%
%Important special cases for cauchy sequences (nested intervals): \\
%Every sequence of interleaved intervals $(A_k)$ with $A_0\supseteq A_1\supseteq A_2 \supseteq ... $ converges towards $A=\bigcap_k^\infty A_{k} $ \\
%
%Consequence: all (?) basic operations (and usual standard functions) are (piece-wise) continuous in $IR$. \\
%
%To conduct the proof on the lower-/upper boundaries of the intervals. \\
%
%Consequences\\
%\begin{itemize}
%\item $|A|=d(A,[0,0])=\max\{|a|,|a|\}=max\{\sup\limits_{a\in A}\{d(a,0)\}=max\{\sup\limits_{a\in A}\{|a|}\}$
%\item $A\subseteq B \implies |A|\leqslant |B|$
%\end{itemize} \\
%
%Properties of the distance: \\
%\begin{enumerate}
%\item $d(A+B,A+C)=d(B,C)$ 
%\item $d(A+B,C+D)\leqslant d(A,C)+ d(B,D)$
%\item $d(a\cdot B, a\cdot C)=|A|\cdot d(B,C)$
%\item $d(AB,AC)\leqslant |A|\cdot d(B,C)$
%\end{enumerate} \\
%
%Properties of the diameter: \\
%\begin{enumerate}
%\item diam$(A)=a-a\geqslant 0$
%\item defined point interval $X$ through diam$(X)=0 \iff x=x$
%\item $A\subseteq B \implies $diam$(A)\leqslant$ diam$(B)$
%\item diam$(A+- B)=$diam$(A)+$diam$(B)$ 
%%Here the signs +, - should be above each other
%\item diam$(AB)\leqslant $diam$(A)\cdot|B|+$diam$(B)\cdot|A|$
%\item diam$(AB)\geqslant $diam$(A)\cdot|B|+$diam$(B)\cdot|A|$
%\item diam$(a\cdot B)=|a|\cdot$ diam$(B)$
%\item diam$(A^n)\leqslant n\cdot$diam$(A)\cdot |A|^{n-1}, (n\in N), (A^n:=A\cdot A\cdot ...)$ (*)
%\item diam$((A-x)^n)\leqslant($diam$(X))^n, \forall x\in X$ (#)
%\item $|C|\leqslant$diam$(C)\leqslant2\cdot |C|$, with $O\in C$ (?) Is this O?
%\end{enumerate}
%
%Proof for (*) \\
%$n=1 \implies$ diam$(A^1)=$diam$(A)\leqslant 1\cdot |A|^0\cdot$diam$(A)$ \\
%Induction hypothesis: The claim holds for $n, n>1$ \\
%Induction step: diam$(A^{n+1})=$diam$(A^n\cdot A)\leqslant$ diam$(A^n)\cdot|A|\cdot$diam$(A)\cdot|A^n|$ \\
%(?)$\leqslant n\cdot |A|^{n-1}\cdot$diam$(A)\cdot|A|+|A^n|\cdot$diam$(A)$ \\
%$=(n+1)\cdot|A^n|\cdot$diam$(A)$ \\
%\\
%Further properties: $A,B\in IR$ \\
%\begin{enumerate}
%\item diam$(A)=|A-A|$ 
%\item $A\subseteq B \implies \frac{1}{2}\cdot$diam$(B)\cdot$diam$(A))\leqslant d(A,B)\leqslant diam(B)\cdot diam(A)$
%\item $A \subseteq B\iff b\leqslant a\leqslant a\leqslant b$ \\
%$\implies d(A,B)=\max\{|a-b|,|a-b|\}=\max\{a-b,b-a\}$ \\
%$\leqslantb-a+a-b=b-b-(a-a)=diam(B)-diam(A)$ \\
%$\implies d(A,B)=\max\{a-b,b-a\}\geqslant \frac{1}{2}(a-b+b-a)$ \\
%$=\frac{1}{2}\cdot(diam(B)-diam(A))$ \\
%\\
%\end{enumerate}
%Further operations in $IR$ \\
%Definition: Intersection(set): precondition is: $A,B\in IR$ are not disjoint. Then: \\
%$A\cupB:=\{c|c\in A \land c\in B\}=[\max\{a,b},\min\{a,b\}]$ \\
%Property: (inclusion monotonicity) $A\subseteq C, B\subseteq D \implies (A\cup B)\subseteq(C\cup D)$ \\
%Definition: Interval hull: (in $IR$ convex hull) \\
%$A\cup B:=[\min\{a,b\},\max\{a,b\}]$ \\
%(?) Here i dont know how to write the sign between A and B \\
%\\
%
%\textbf{26.11.19} \\
%Proof for (#) \\
%$c\leqslant 0\leqslant c \implies diam(C)=c-c=|c|+|c|\geqslant \max\{c,c\}=|c|$ (?) Im not sure if here its big or small C \\
%and $diam(C)=|c|+|c|\leqslant 2\cdot \mac\{|c|,|c|\}=2|c|$ \\
%\\
%Properties with a symmetric interval $A=-A=[-a,a]$ \\
%\begin{itemize}
%\item $A\cdot B=|B|\cdot A$
%\item $diam(A,B)=|B|\cdot diam(A)$ 
%\end{itemize} \\
%
%\begin{itemize}
%\item Subdistributivity (distributivity) \\
%In $IR$ the subdistributivity holds $A*(B+C)\leqslant A*B+A*C$ \\
%The distributivity holds, if either a) or b) hold: \\
%a) $A=[a,a]$ is a point interval\\
%b) $B*C\geqslant 0$, i.e. either: $(B\geqslant 0 \land C\geqslant 0)$ or $(B\leqslant 0 \land C\leqslant 0)$ \\
%\\
%Proof for a): \\
%$a\cdot (B+C):=\{a*(b+c)| b\in B, c\in C\}$ \\
%$a\cdot B+ac\dot C:=\{ab| b\in B\} \cup \{ac|c\in C\}=\{ab+ac| b\in B,c\in C\}$ \\
%\implies $a\cdot (B+C)=a\cdot B+ac\dot C$ \\
%\\
%Proof for b): \\
%If $B\geqslant 0 \land C\geqslant 0: B+C\geqslant 0, B*C\geqslant 0$ (products are $\geqslant 0$) \\
%1st case: $A\geqslant 0 \iff a\geqslant 0$: \\
%$A*(B+C)=[a\cdot(b+c),a\cdot(b+c)]$ \\
%$A*B+A*C=[a\cdot b+a\cdot c,a\cdot b+a\cdot c]$ \\
%\implies $A*(B+C)=A*B+A*C$ \\
%
%2nd case: $A\leqslant 0 \iff a\leqslant 0 $ (all products $\leqslant 0$) \\
%$A*(B+C)=[a\cdot(b+c),a\cdot(b+c)]$ \\
%$A*B+A*C=[a\cdot b+a\cdot c,a\cdot b+a\cdot c]$ \\
%\implies $A*(B+C)=A*B+A*C$ \\
%
%3rd case: $a\cdot a<0 \iff a<0<a \iff 0\in A \from int(A)$ \\
%$A*(B+C)=[a\cdot(b+c),a\cdot(b+c)]$ \\
%$A*B+A*C=[a\cdot b+a\cdot c,a\cdot b+a\cdot c]$ \\
%\implies $A*(B+C)=A*B+A*C$ \\
%
%\\
%\\
%If $B\leqslant 0 \land C\leqslant 0: B+C\leqslant 0, B*C\geqslant 0$ \\
%1st case: $A\geqslant 0 \iff a\geqslant 0$: \\
%analogously to the 1st case from above, with switching all boundaries \\
%$A*(B+C)=[a\cdot(b+c),a\cdot(b+c)]$, (products $\leqslant 0$) \\
%
%2nd case: $A\leqslant 0 \iff a\leqslant 0 $ \\
%analgously to the 1st case from above with switching the boundaries
%$A*(B+C)=[a\cdot(b+c),a\cdot(b+c)]$, (products $\geqslant 0$) \\
%
%3rd case: $A\ni 0:$ analogously to case 3 from above with switching the boundaries (ALL) \\
%$A*(B+C)=[a\cdot(b+c),a\cdot(b+c)]$ \\
%
%\item Interval vectors (interval boxes) \\
%$IR^n =$ set of vectors with $n$ components in $IR$ ($IC^n$ analogously) \\
%$V\in IR^n: V=\begin{bmatrix}v_1 \\ . \\ . \\ .\\ v_n\end{bmatrix}= \begin{bmatrix} [v_1,v_1] \\ . \\ . \\ .\\ [v_n,v_n]\end{bmatrix}$, i.e. $V_i=[v_i,v_i] \in IR$ \\
%alternative portrayal as pair of point vectors with components being all lower boundaries, i.e. all upper boundaries $(V_1,V_2 \in R^n)$ \\
%$V=[V,V]=[V_1,V_2]=[\begin{bmatrix}v_1 \\ . \\ . \\ .\\ v_n\end{bmatrix}-\begin{bmatrix}v_1 \\ . \\ . \\ .\\ v_n\end{bmatrix}]$ \\
%Geometrically we also work with achsis (?) quaders with $Z^n$ edge points $\in \{v=(v_1,..,v_n)^t\in R^n$ with $V_i\in\{v_i,v_i\}, \forall i=1,..,n\}$ \\
%
%\item Important properties of the interval arithmetics 
%\begin{enumerate}
%\item Inclusion property
%\begin{enumerate}
%\item arithmetic operations \\
%$\circ \in\{+,-,*,/\}: A\subseteq A^\prime, B\subseteq B^\prime \implies A\circ B\subseteq A^\prime \circ B^\prime $ \\
%\item elementary standard functions \\
%$i \in \{sin, cos, tan, exp, log, sqrt{},..\}:A\subseteq A^\prime \implies f(A)\subseteq f(A^\prime) $
%\item also for arbitrary functions which are given with calculation rules (formulas) (\to composition of the basic operations and functions) \to computational graph; \\
%Input: parameter values \to output: function values \\
%Graph: (Here comes the graph) \\
%\implies the inclusion property is through everz computational graph propagandized \\
%\implies inclusion property \implies guaranteed inclusion of the solution(set). \\
%\end{enumerate}
%\item Overestimation of the solution set: \\
%occurs mainly as a consequence  of the fact that the same variable   occurs repeatedly in a Ber(?)-graph because everytime through the set based definition of the interval arithmetics, the elements are seen decoupled from one another in a variable interval. \\
%\begin{enumerate}
%\item An example: $X-X=[-diam(X),+diam(X)]\to$ strech as much as possible, and replace through O?
%\item \implies Recommendation: To keep the number of (?) of each variable in the calculation rule as much as possible. For example, by factoring out common subexpression (example: distributive law). \\
%\item further countermeasures \\
%interval subdivision of parameter intervalls and individual evaluation of the individual sub-intervals, to take interval hull of this enclosure \\
%\item use of smoothness characteristic of the calculation rule, ie use of derivatives (mind $f^\prime$) and estimation of the value range (?)
%\end{enumerate}
%\item Wrapping-effect (in multidimensional, $IR^n$) \\
%An example: $f:R^2 \xrightarrow{} R^2$, rotation around the angle $\phi$ \\
%$f(X)=A\phi\cdot X$ with $A=\begin{bmatrix}cos(\phi) sin(\phi)\\ -sin(\phi) cos(\phi) \end{bmatrix}$ \\
%repeated use: $X_{i+1}:= f(X_i)=A\phi\cdot X_i=A\phi\cdot x_i$ \\
%$=A\phi\cdot A\phi\cdot X_{i-1} = .. = A\phi^{(i+1)}\cdot x_0$ \\
%$=\begin{bmatrix}cos(i \phi) sin(i \phi)\\ -sin(i \phi) cos(i \phi) \end{bmatrix} \cdot x_0$ \\
%In $IR: $ square $X_0:=\begin{bmatrix}[-\epsilon, +\epsilon]\\ [-\epsilon, +\epsilon] \end{bmatrix}, \epsilon>0 $ \\
%$X_1=begin{bmatrix}[-\epsilon, +\epsilon]\cdot cos(\phi) [-\epsilon, +\epsilon]\cdot sin(\phi)\\ [-\epsilon, +\epsilon]\cdot (-sin(\phi)) [-\epsilon, +\epsilon]\cdot cos(\phi) \end{bmatrix}$ \\
%$=\begin{bmatrix}|sin(\phi)|+|cos(\phi)|\end{bmatrix} \cdot \begin{bmatrix} [-\epsilon,+\epsilon] \\[-\epsilon,+\epsilon] \end{bmatrix}$ \\
%\\
%$X_n= (|sin(\phi)|+|cos(\phi)|)^n \cdot X_0$ the diameter grows exponentially! \\
%\\
%Here comes the graph \\
%\end{enumerate}
%\end{itemize}\\
%
%\textbf{2.12.19} 
%\begin{itemize}
%\item Extension of the number field $R$ \\
%$\Omega:=\{-\infty, +\infty\}$, $R^*:=R \cup \Omega$ \\
%$IR:=\{[x,y]|x\leqslant y, x,y\in R\}$ \\
%$IR^*:=IR\cup \{(-\infty,y]|y\in R\}\cup \{[x,+\infty)|x\in R\} \cup \{(-\infty, +\infty)\}$, \\
%where \\
%$(-\infty,y]:=\{a\in R|a\leqslant y\}$ \\
%$[x,+\infty):=\{a\in R|a\geqslant y\}$ \\
%\item extended interval division $X/Y$ with $0\in Y$ \\
%$Y=Y_1\cup \{0\}\cup Y_2$ with $Y_1=[y,0)$ and $Y_2=(0,y]$ \\
%$X/Y:=X/Y_1 \cup X/Y_2 =\{x/y|x\in X, y\i Y_1\} \cup \{x/y| x\in X, y\in Y_2\}$ \\
%Here the things under $X/Y_1 \cup X/Y_2$ are missing! \\
%
%\\
%$1/Z=1/Z_1 \cup 1/Z_2\cup \{0\}=[-\frac{1}{4},0)\cup (0,\frac{1}{2}]\cup \{0\}=Y/X$ \\
%
%For $0\in Y: $ \\
%$$
%X/Y=
%\begin{cases}
%(-\infty,+\infty), & \text{if } X=0 \text{or } 0\in X \text{or } Y=0\\
%(-\infty, x/y]\cup [x/y,+\infty), & \text{if } X\leqslant0 \text{and } y\leqslant y=0 \\
%(-\infty, x/y], & \text{if } X\leqslant0 \text{and } 0=y< y 
%\end{cases}
%$$ \\ 
%Here just the circle above the $X$ is missing in the first case \\
%$$
%X/Y=
%\begin{cases}
%(-\infty,X/Y]\cup[X/Y,+\infty), & \text{if } 0\leqslant X \text{and } 0\in Y \\
%(-\infty, x/y], & \text{if } 0\leqslantx \text{and } y<y=0 \\
%[x/y,+\infty), & \text{if } 0\leqslantx \text{and } 0=y<y 
%\end{cases}
%$$ \\
%Here the circle above $Y$ in the first case is missing \\
%For $0\not \in Y: 1/y=[\frac{1}{y},\frac{1}{y}]$ \\
%For $0\in Y: 1/y=(-\infty,\frac{1}{y}]\cup [\frac{1}{y},+\infty)$\\
%$\frac{1}{1/y}=[\frac{1}{1/y})\cup \{0\}\cup (0,\frac{1}{1/y}]=[y,y]=Y$\\
%\item $W_i(X)\subseteqf(X)$ \\
%Assumption: $f:J\xrightarrow{}R, J=[a,b]\subset R, f\in C^1(J)$ \\
%For arbitrary intervalls $x\subseteq J$ it holds: \\
%For a $z\in X$ (here a circle above $X$) (eg. $z=x=mid(X)$)per mean value theorem there is a point $\xi\in X$ $\forall x\in X$ with: \\
%$f(x)=f(z)+f^\prime(\xi)(x-z) \in f(z)+f^\prime(X)(X-z)=:f_m(X)$ \\
%$f_m$ is called a mean value form or centered form. \\
%Note: If $\cup_{k=1}^n X_k=X\subseteq J$ a decomposition of $X$ in part-intervalls , then there are constants $c_1,c_2 \in R_+$ which are only dependant of $f$ and $J$, so that the distance \\
%$d_1:=d(W_f(X), \cup_{k=1}^n f(X_k))\leqslant c_1\cdot max_k (diam(X_k))$ \\
%while the distance \\
%$d_2:=d(W_f(X),\cup_{k=1}^nf_m(X_k))\leqslant c_2\cdot max_k(diam^2(X_k))$ \\
%$d_1\implies $ overestimation decreases linearly \\
%$d_2\implies$ overestimation decreases quadratically$. \\
%
%\item Classic Newton-method: \\
%$f\in C^1(\xi)$, start point/- approximation $X_0$ \\
%$X_{n+1}=X_n-\frac{f(X_n)}{f^\prime(X_n)}, n=0,1,2,..$ \\
%local linearisatin of the function $f$ at point $(x_0,f(x_0))$ \\
%In general the Newton's method is not convergent. \\
%If it converges (towards a zero of the function, which is a eventually non near the starting point x_0 lying point), then the convergence rate is quadratic, i.e. the error is squared in each iteration, i.e. the number of correct points is doubled. \\
%\item Interval- Newton- method \\
%$f\in C^1(J), J=[a,b]\subseteq R$ start-/searchinterval $X_0=[x_0,x_0]$ \\
%Wanted: Zero(s) of $f$ in $X_0$, i.e. $x^*\in X_0$ with $f(x^*)=0$ \\
%For every $x\in X_0, x\not = x^* \exists \xi \in X_0$ (between $x$ and $x^*$), with: \\
%$f(x^*)-f(x)=f^\prime(\xi)\cdot (x^*-x)$ \\
%$0=f(x^*)=f(x)+f^\prime(\xi)\cdot (x^*-x)$ \\
%$x^*=x-\frac{f(x)}{f^\prime(\xi)} \in x-\frac{f(X)}{f^\prime(X)}$ \\
%\\
%Interval-Newton Operator: \\
%$N(f,x,X):=x-\frac{f(x)}{f^\prime(X)} \implies $ Interval -Newton Iteration \implies Interval nesting \\
%with $x\in X$ (Here signs above both $x$) \\
%\\
%$X_{n+1}:=N(f,x_n,X_n)\cup X_n\leqslant X_n$ \\
%\\
%Note: 
%\begin{enumerate}
%\item Every zero of $f$ in $X$ has to be in $N(f,x,X)$ for $x\in X$ 
%\item One obtains interval nesting through $\cup X_n$ \\
%\implies convergence towards the zero (assumption: $0\not \in f^\prime(X)$) 
%\item If $x$ is chosen to be a midpoint of $X$, then the convergence is better than the bisection method \\
%Here comes the graph \\
%One can show that the convergence ( as in the classic Newton-method) is quadratic. ($0\not \in f^\prime(X_0)$). \\
%\\
%3 cases: (occur in the iteration) with $0\not \in f^\prime(X)$ \\
%\begin{enumerate}
%\item $N(f,x,X)\subseteq X \implies \exists! $ zero of $f\in N(f,x,X)\subseteq X\subseteq X_0$ 
%\item The intersection is empty: $N(f,x,X)\cup X= \empty$ \\
%\implies there does not exist a zero in the start interval $X_0$
%\item Otherwise: the iteration has to be continued until case 1 or case 2 occurs. 
%\end{enumerate}
%% In practice, this results in the typical interval methods result, ie, a fining intervals is numerically impossible and 1) or 2) have not been met \implies outlay: zero possible, but not verified.
%%Check
%\end{enumerate}
%\item Transition to floating-point: $IR$ to $IR$ \\
%$IR:=\{X=[x,x]|x\leqslant x, x,x\in R =$ floating point (?) $\}$
%
%\item Interval rounding: $\bDiamond : IR \xrightarrow{} IR, \bDiamond X:=[\nablax, \triangle x] \supseteq X$ \\
%Properties: \\
%$\bDiamond X=X, \forall X\in R$ (R1) \\
%$X<Y \implies \bDiamond X\leqslant \bDiamond Y$ for $X,Y\in IR$ (R2) \\
%$\bDiamond(-X)=-\bDiamond X \forall X\in IR$ (R3) \\
%Proof for (R3): \\
%With (R4) with the directed roundings $\triangle(-x)=-\nabla x$ \\
%$\bDiamond(-X)=\bDiamond (-[x,x])=\bDiamond[-x,-x]=[\nalba(-x),\triangle(-x)]=[-\triangle x,-\nabla x]=-[\nabla x,\triangle x]=-\bDiamond X$ \\
%\item Properties of the floating point interval arithmetics: \\
%$+,*$ are commutative, but not associative! \\
%Example: $\epsilon:=$ Wilkinson$-\epsilon=b^{1-l}$ \\
%([-\frac{\epsilon}{2},\frac{\epsilon}{2}]\bDiamond[-\frac{\epsilon}{2},\frac{\epsilon}{2}])\bDiamond \frac{3}{2}=[-\epsilon,+\epsilon] \bDiamond \frac{3}{2} =[\frac{3}{2}-\epsilon, \frac{3}{2}+\epsilon] \\
%\\
%$[-\frac{\epsilon}{2},\frac{\epsilon}{2}]\bDiamond([-\frac{\epsilon}{2},-\frac{\epsilon}{2}]\bDiamond \frac{3}{2})=[-\frac{\epsilon}{2},\frac{\epsilon}{2}]\bDiamond [\frac{3}{2}-\epsilon , \frac{3}{2}+\epsilon]= [\frac{3}{2}-2\epsilon, \frac{3}{2}+2\epsilon]$ (rounding!)  \\
%\\
%$\exists$ neutral elements $0=[0,0], 1=[1,1]$ with $x+0=0+x=x, 1*x=x*1=x$ \\
%In general there does not exist inverse elements (as already in $IR$) \\
%\\
%Subdistributivity applies almost always, factoring out common subexpressions should be, however, always put in practice by reducing the (?) \\ 
%
%I dont understand the next sentence \\
%\\
%\item Floating point numbers:  \\
%ulp$:=b^{e_x-l}$ with $x=(-1)^{S_x}\cdot m_x \cdot b^e_x$ \\
%with $l=$length of the mantissa $=$ number of digits to a base $b$ \\
%succ$(1)=1+ulp(1)=1+\epsilon$, since $1=(-1)^0\cdot 0.100...0\cdot b^1$ and $\epsilon =ulp(1)=b^{1-l}$ \\
%\\
%Sometimes it is more practical to write floating point numbers in another form, namely as a integer multiple of ulp(x): \\
%$x=(-1)^{S_x}\cdot m_x\cdot b^{e_x}=(-1)^{S_x}\cdot M_x\cdot b^{e_x-l}=(-1)^{S_x}\cdot M_x\cdot ulp(x)$ \\
%
%If $x$ is normalized, then it holds $\frac{1}{b}\leqslant m_x<1$, i.e. $b^{l-1}\leqslant M_x\leqslant b^l-1, M_x\in N$ \\
%\\
%If we involve the sign in the mantissa $M_x$ and additionally allow the zero and the denormalized FPN then it holds: \\
%$x=M_x\cdot ulp(x)$ with $M_x\in Z, |M_x|\leqslant b^l-1$ \\
%
%In the following the exponent range is initially to be looked as unlimited) i.e. eventually occurring under or overflow is recognized at the very end, after rounding. \\
%
%\item $4$ basic calculation ways in FPN (semimorph (?) computer arithmetic) \\
%\\
%Here comes the graph \\
%\\
%Semimorph C-arithmetics: $\circ\in \{+,-,*,/\}, \bigcirc\in \{\square,\triangle, \nabla, ..\}$ Here some signs are missing \\
%
%Then \\
%$x(?)y:=\bigcirc(x\circ y), \forall x,y\in R$ (floating point (?)) \\
%
%Additional rounding for general basis $b\in \{2,3,4,..\}$ \\
%\\
%\\ 
%Here Im just going to write everything, you can just copy-paste it in the brackets as needed  \\
%\\
%Defined help-function: $x\in R, x>0 $ \\
%$S_\mu(x):=\nabla x+\frac{\triangle x-\nabla x}{b}\cdot \mu $, \\
%for $x\in R: S_\mu(x):= \nabla x+ulp(x)\cdot \frac{\mu}{b}$ 
%\\
%\\
%
%with $\mu\in \{1,2,..,b-1\}$ \\
%for $\mu=0$ ... Here write \\
%for $\mu=b$ ... Here write \\
%,i.e. ... Here write \\
%
%
%$$
%\square \mu(x):=
%\begin{cases}
%0, & \text{if } x\in [0,x_{min})=[0,b^{e-1})\\
%\nabla x,  & \text{if } x\in [\nabla x, S_\mu(x)), x>0\\
%\triangle x,  & \text{if } x\in [s_\mu(x), \triangle x), x>0\\
%-\square \mu(-x),  & \text{if } x<0 \text{(antisymmetric)}  
%\end{cases}
%$$ \\
%
%An example: \\
%$b=10, l=4: \square_(?) (0.5739; 5)=0.5739$ \\
%$\square_(?) (0.5739; 6)=0.5740$ \\
%
%\item 5 steps for a arithmetic floating- point operation: \\
%\begin{enumerate}
%\item decomposition of the operands  \\
%$x\to (S_x,m_x,e_x)=(M_xl,e_x-l)$ \\
%\item $y\to (S_y,m_y,e_y)=(M_y,e_y-l)$ \\
%where, $M_x, M_y$ is favoured in 2er-complement 
%\item Computation of a (approximated) intermediate result  \\
%$z:=z\circ y \approx x\circ y = z$ \\
%(with accuracy requirement that $Oz=Oz=$ result 
%\item Normalising: \\
%Adjust To eliminate the leading zero digits in the mantissa through L(left)-Shift_k\from (by k positions) and adjust the exponent (substract k) \\
%
%or: AR-Shift_1, if a leading 1 has come as a carry before the mantissa. 
%
%\item Rounding: \\
%$z=x\circ y=Oz=Z(x\circ y)=O(x\circ y)$ \\
%In rare cases (if the mantissa results in $m_z=1.00...0$) the result of the rounding must be normalized again, i.e. ARShift_1 yields $0.100...0$ and increase the exponent $+1$.
%
%\item composition $(M_z, ex-l)\to (s_z,m_z,l_z)\to z$ \\
%4 basic operations (arithmetically) \\
%\begin{enumerate}
%\item multiplication: $z=x\circ y$ \\
%Case 1: $x=0 \lor y=0 \implies z=0$ \\
%Case 2: $x\not = 0 \land y\not =0: $\\
%$z=(-1)^{S_x+S_y}\cdot m_x\cdot m_y\cdot b^{e_x+e_y} $ \\
%$=(-1)^{S_x+S_y}\cdot M_x\cdot M_y\cdot b^{e_x+e_y-2l} $ \\
%$ =(-1)^{S_x+S_y}\cdot M_x\cdot M_y\cdot ulp(x)\cdot ulp(y) $ \\
%\\
%If $x,y$ are normalized: $B^{2l-2}\leqslant M_x\cdot M_y< b^{2l}$ \\
%\implies $M_x\cdot M_y$ need $2l-1$ or $2l$ digits \\
%\\
%Computation of the exact double-long product is necessary, so that one can round always correctly! \\
%
%Case 1: $M_x\cdot M_y=b^{2l-1} (\implies 2l$ digits $)\implies ez:=ex+ey$ \\
%Case 2: $M_x\cdot M_y<b^{2l-1} (\implies 2l-1$ digits $)$, then $m_x\cdot m_y<b^{-1}$ is not normalized \implies LShift_1, i.e. $m_z=m_xm_y\cdot b$ (rounding) and $ez:=ex+ey-1$//
%//
%Note: AR_Shift = arithmetics (?) Shift
%
%\begin{itemize}
%	\item Rounding is always based on the first digit, which doesn not fit in the mantissa anymore (ie $l + 1$ significant digit plus information whether after that something else comes, which is not zero (ie rest from $l + 2$, digits $\not = 0$)
%	
%	\item Round digit = $l+1$ significant digit
%	
%	\item sticky bit $s=V_{i=l+2}m_i =$
%	$$
%	s=V_{i=l+2}m_i =
%	\begin{cases}
%	0, & \text{if rest}=0\\
%	1,  & \text{otherwise}
%	\end{cases}
%	$$ \\
%	$0.m_1m_2 \cdot \cdot \cdot m_l $
%	\\ Here idk how to write this.
%	
%	\\
%	This holds especially for our roundings :
%	(write the roundings?)\\
%\end{itemize}
%\item Division: $z=x/y$ with $y\not=0$ [otherwise\to error] \\
%Case 1: $x=0 \implies z=0$ \\
%Case 2: $x\not = 0 (y\not =0):$ \\
%$z=x/y-(-1)^{S_x+S_y}\cdot \frac{m_x}{m_y}\cdot b^{e_x-e_y}=(-1)^{S_z}\cdot \frac{M_x}{M_y}\cdot b^{e_x-e_y}$, with $S_z:=MOD(S_x+S_y,Z)$ and here the approximation is missing 
%\\
%$1/b<m_z^*<b$ with normal $x$ and $y$.\\
%\\
%\begin{itemize}
%	\item If $m_x\geqslant m_y \iff m_z^*\geqslant 1$, then the result has to be shifted with AR-Shift_1 (or to write the (HW?) of the first digits on the "correct" position of the mantissa and to increase it by 1 \implies normalisation is not necessary
%	\item If $m_x<m_y \implies m_z^*<1$, then adjusts the position of the digits 
%\end{itemize}
%\\
%
%\end{enumerate}
%Implementation of the mantissa division manually: \\
%with $x>0, y>0:$ \\
%0. accumulator: $A:=x$ \\
%1. for $i=1,..,l+1$
%\begin{itemize}
%%\item Compute the quotient digit $q_i$ from left to right. In addition to that one has to ... the maximal multiple of $y$ by disconnecting the multiples of, that can be pulled off yet from the accumulator A.
%\item $A:=A-q_i\cdot y\cdot b^{-i}$ 
%\end{itemize}
%2. Quotient $z=0.q_1q_2...q_l|q_{l+1} \from m_z^*$ with rest $R=A_{l+1}$ (last akku-content) \\
%3. Rounding: with rounding digits $r=q_{l+1}$ and sticky bit \\
%$$
%s =
%\begin{cases}
%0, & \text{if R}=0\\
%1,  & \text{if R}\not= 0
%\end{cases}
%$$ \\
%
%\item Addition and substraction (4) $z=x+-y$ \\
%substraction: build 2er complement of $y$ and add\\
%assumption: $e_x\geqslant e_y$ and $x\not=0$ and $y\not=0$ (or stricter: $|x|\geqslant |y|$) \\
%special cases are not going to be considered separately \\
%\\
%$e_x\geqslant e_y$ if this doesnt hold, then switch $x$ and $y$ \\
%
%\\
%Mantissa obtaine here also the sign \\
%(2er complement/ $b-$complement kochierung?) \\
%- exact result: $z=x+y$\\
%- intermediate result: $z=z+y$ (enough for correct rounding, so that the floating point result: $z:=O(z)=O(z)$ ...? \\
%\\
%General assumption $x\not=0, y\not=0, |x|>|y|$ \\
%Algorithm with long accumulator: with $(2l+1)$ digits (to a base $b$) and $1$ carry-out-bit. (Here comes the drawing) \\
%\\
%Case 1: Exponent difference $d:=e_x-e_y\geqslant l+2$ \\
%$b=10, l=6$ \\
%DRAWING\\
%\\
%Rounding: \\
%a) $\square\mu (\mu\in \{1,..,b-1\})$ + drawing \\
%b) the signs, $m_z=b^{-(l+2)}\cdot (b^{l+1}\cdot m_x+sign(y)$, where 
%
%$$
%sign(y) =
%\begin{cases}
%1, & \text{if } y>0\\
%0,  & \text{if} y=0\\
%-1,  & \text{if} y<0
%\end{cases}
%$$ \\
%$z=x+b^{-2}\cdot ulp(x)\cdot sign(y)$ \\
%rounding yields $prod(x)$ or $x$ or $succ(x)$\\
%
%\\
%Case 2: $d=e_x-e_y=l+1$ \\
%DRAWING \\
%\\
%Rounding: \nabla, $\square \mu (\mu=6,..,10)$, result + drawing \\
%$\square, \square_{IEEE}, \triangle, \square\mu(\mu=0,..,5)$, result + drawing \\ 
%b) DRAWING \\
%Rounding: $\square, \square_{IEEE}, \nabla, \square\mu(\mu=5,..,10), \triangle, \square\mu (\mu=0,..,4)$, result + drawing $ \\
%\\
%$\implies$ Idea: short accumulator with $1$ bit $+l z_i+1\cdot z_i+1\cdot z_i+1$bit \\
%\\ here the things under the line are missing \\
%\implies $l+2$ digits $+2$ bit \\
%(for $b=2$ i.e. $l+4$ bit) \\
%\\
%\end{enumerate}
%\end{itemize} \bigskip