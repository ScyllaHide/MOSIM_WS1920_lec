%TODO still to reformat and fill ?
\begin{itemize}
    \item $n$ FA work parallel \\
    reduce $3-n$ bit numbers to $2$ numbers \\
    C contains all (?)-bits, S all sum-bits
    \item for final sums $\sigma$ add $S+2C$ \\
    Here I dont know how to do the sums + the two graphs
\end{itemize} \smallskip

$T(n)=T_{walace}+T_{CLA}=\O(\log n)$ \\
addition of $n$ numbers requires $(n-2)$ CSA

\begin{itemize}
    \item MUltiplication ($n\times n$ bit) \\
    (?): with little HW
    \begin{itemize}
        \item A Carry-Ripple-Adder: $T=O(n^2)$
        \item A Carry-Skip/Select-Adder: $T=O(n\cdot \sqrt{n})$
        \item A CLA (Binary tree): $T=O(n\cdot \log n)$
        \item A CLA (Carry-Safe-Adder): $T\approx (n-2)+2n\approx 3n=O(n)$
    \end{itemize}
    Otherwise with a lot of HW \\
    \begin{itemize}
        \item $(n-2)$ CSA and $1$ CLA
        \item for individual multiplications ($1$-dimensional layout): $T_1=O(n+\log n)=O(n) \rightarrow$ with (?) \\
        $T_p=T_1/$ number of pipeline-steps $=O(n)/O(n)=O(1)$
        \item Wallace-Tree-Layout: $T_1=\log_{\frac{3}{2}} n+2\log_2 \approx O(\log n)$ \\
        $\rightarrow $ with Pipeline: $T_p=T_1/O(\log n)=O(1)$ \\
        ((?) = number of pipeline-steps $\approx O(\log n)$)
    \end{itemize}
   
    \item Division  \\
    $q:=(?) (a/b)$ with rest $r$$\to$do{missing}, with assumption $a>0, b>0$. \\
    For the rest it holds: $q\cdot b+r=a$ \\
    Iterative division generates $n$ "partial remainders" started with $r_0:=a$ and ended in $q:[q_{n-1} ... q_1 q_0]$ and final rest $r:=r_n$ after $n$ steps/iterations. \\
    Here comes the graph \\
    \\
    \item (?) Division:$\todo{missing}
    Init: $a \to A, b \to B, o \to P$ (possible to calculate and to save b). \\
    For $i=1,..,n$:
    \begin{enumerate}
        \item $LShift_1(R)$ (all $2n+1$ bit)
        \item $P:=P-B$
        \item If $P<0$, set low-rder position of A to O \\
        otherwise to $1$ ($q_{n-1}:=p_n$)
        \item If $P<0$ restore old $P:=P+B$
    \end{enumerate}
    After $n$ steps A contains $q$ and P contains $r=r_n$.
    \item Nonrestoring Division: \\
    init: (?) above \\
    For $i=1,..,n$ \\
    If $P<0$ \\
    \begin{enumerate}
        \item $L-Shift(R): (2n+1$ bit$)$
        \item $P:=P+B$
    \end{enumerate}
    Else $(P\ge 0)$
    \begin{enumerate}
        \item $L-Shift(R):$
        \item $P:=P-B$
    \end{enumerate}
    End if \\
    3)Set Quotient digits:
\[
	q_{n-1}:
	\begin{cases}
	0, & \text{if } P<1\\
	1 & \text{if } P\ge 0
	\end{cases}
\]
After $n$ steps: \\
4) if $(P<0)$, restore remainder by $P:=P+B$, making $P\ge 0 $ \\

\item Example: $n=8$, \\
$a=211=0.11010011\cdot 2^8$, $q=01000110=70$ \\
$b=3=0.00000011\cdot 2^8$, $r=00000001=1$ \\

Here comes the drawing. \\
Resoring & nonrestoring division are equivalent: \\
step k: $q_{n-k}$ is based on sign of $2\cdot r_{k-1}-2^n\cdot b$ in restoring devision
\begin{enumerate}
    \item if $2r_{k-1}-2^n\cdot b \ge 0:$ both algorithms do the same new partial remainder $r_k:=2r_{k-1}-2^n\cdot b$, $q_{n-k}:=1$
    \item if $2r_{k-1}-2^n\cdot b < 0:$ restoring division $\to$  $r_k:=2r_{k-1}$ \\
    Here the arrow is missing \\
    step $k+1: q_{n-(k+1)}$ is based on sign of \\
    $2r_k+2^nb=4r_{k-1}-2^nb$ \\
    \\
    non restoring division $\to$ we keep the negative partial remainder $r_k:=2r_{k-1}-2^n\cdot b (<0)$ \\
    Here the arrow is missing  \\
    step $k+1: q_{n-(k+1)}$ is based on sign of \\
    $2r_k+2^nb=2\cdot (2r_{k-2}-2^n\cdot b) + 2^nb=4r_{k-1}-2^nb$ \\
   
    $\implies$ both algorithms do the same \\
    $\to$ produce same quotient digits \\
    $\to$ but different sequence of partial remainders
\end{enumerate}
\item An example: restoring division \\
$n=4, A=14=1110, B=3=0011$ \\
\todo{Here comes the table}
\item An example: nonrestoring division \\
$n=4, A=14=1110, B=3=00011, -B=11101$ \\
\todo{Here comes the table}
\item Redundant number display: \\
Base $\beta\ge 2, \beta \in N$ \\
A number representation-system is redundant if the digit set S contains more than $\beta$ digits. We assume that any digit set S contains $0$ and its digits are continuous (no holes): \\
$S:=\{d,d+1,..,d\}$, where $d\le 0\le d$ \\
In practice we only consider digit sets S with $d\ge 1-\beta$ and $d\le \beta-1$ \\
A symmetric digit sets satisfies $d=-d$ \\
A minimally redundant digit set S contains exactly $\beta -1$ digits \\
A minimally redundant digit set S \\
$\to$ for $\beta$ even, $S:=\set{-\beta/2, \beta/2+1,..,\beta/2}$ \\
$\to$ for $\beta$ odd, $S:=\set{-(\beta+1)/2, ,..,(\beta+1)/2}$ \\

\item SRT-division ((?) \todo{missing} \person{Robertson}, \person{Tocher}, 1958) $[a/b]$ \\
class of division algorithms \\
- normalise divisor $b$ at the beginning (with $L-Shift_k$) \\
- requirement for redundant symmetric digit set for quotient digits $[q_{n-1}q_{n-2} ... q_0]$ \\
- quotient digit selection $q_{n-1}$, based on only a feq leading digits if partial remainder $r_{i-1}$ and of the divisor $b$ \\
$\implies$ use of a ($2-$dim) lookup table is possible \\
- partial remainders $r_i$ are represented in a redundant number system, e.g. as $2n-$bit, $#s$ (useful for CSA)\\

Note: $r_i=\beta\cdot r_{i-1}-q_{n-i}\cdot b$
\item single radix $2$, SRT-division: \\
-$a/b\to$ (?), assuming $b\not =0$ \\
-$a,b$ floating point, $#s$, binary point just left to registers $P/A$ and $B$.\\
1) determine the number of leading $k$ zeros in B ($n$ digits), $L-Shift_k(B), L-Shift_k(R), $ (R=P|A) \\
Note: $0\le k\le n-1 $ (since $b\not =0$) \\
$\implies$ leading bit of P is $0$ $\implies$ $|r_0|\le 1/2$ \\

2) For $i=1,..,n$ \\
a) If 3 leading bits are the same $(000 $ or $111)$, set $q_{n-1}:=0, L-Shift_1(R)$ \\
b) If 3 leading bits are NOT all the same and $P<0$, $(000 $ or $111)$, set $q_{n-1}:=(-1)=1, L-Shift_1(R), P:=P+B$ \\
c) If 3 leading bits are NOT all the same and $P\ge 0$, $(000 $ or $111)$, set $q_{n-1}:=1, L-Shift_1(R) $ and $P:=P-B$.\\

3) If the final remainder $P<0$, restore it $:P:=P+B$ and correct the quotient: $q=q-1$ (substract of $1$ in the last position $a_0$)\\

4) One obtains $r_n$ with $AR-Shift_k(P)$ \\
\implies In the SRT -Division: $|\text{remainder}|\le 1/2$ in every step \\
$\beta=2^m, S=\{-2^{m-1},..,+2^{m-1}\}$ \\

\item An example: $\beta=4, S=\{-2,-1,0,1,2\}\to$ SRT $4$ division, compute $2$ bits of quotient per iteration minimally redundant symmetric digit set \\

$\frac{r_{i+1}}{b}=4\cdot \frac{r_i}{b}-q_i$, basis $=4$. \\

I dont understand the things with pencil? \\

(1) If $|r_i/b|\le 1\implies |\frac{r_{i+1}}{b}|\le 2 \implies |\frac{r_{i+2}}{b}|\le 6$ \\
requires an extra bit for rem. per iteration, remainder becomes larger and larger \\
\implies connection of quotient via successive bit is generally impossible  \\

(2) If $|r_i/b|\le 3/4 \implies |\frac{r_{i+1}}{b}|\le 1 \implies$ case (1) \\

(3) If $|r_i/b|\le 1/2 \implies $  \\

\[
	4\cdot \frac{r_i}{b} \in
	\begin{cases}
	(\frac{3}{2},2] \implies q_{k-(i+1)}=2\\
	(\frac{1}{2},\frac{3}{2}) \implies q=1\\
	(-\frac{1}{2},\frac{1}{2}) \implies q=0\\
	(-\frac{3}{2},-\frac{1}{2}) \implies q=-1\\
	[-2,-\frac{3}{2}) \implies q=-2\\
	\end{cases}
\]

To ensure $|\frac{r_{i+1}}{b}|\le \frac{1}{2}$ \\
$\implies$ case (2) to severe. \\

Here the drawing is missing. \\
\\

\item SRT-8-Division, i.e. $\beta=8$, (?) $\to$ todo{missing} $\alpha=4, $ i.e. $S=\{-4,-3,..,3,4\}$ \\
$\implies$ best choice: $|\frac{r_i}{b}|\le \frac{4}{7} \implies |\frac{r_{i+1}}{b}|\le |\frac{32}{7}-4|=\frac{4}{7}$


\item P-D-Diagrams (partial remainder-divisor) \\
\todo{Here the graphs are missing}
\item Speeding up division vie SRT
	\begin{itemize}
		\item increase radix (use groups of digits!)
		\item compute only approximation $r_i$ or $r_i$, to be used in table look up process
		\item represent $r_i$ as $r_{i,c}$ and $r_{i,s}$ and (Here the thing under $r_{i,c}$ is missing)
		\item use CSSA to compute new partial remainder 
		$r_i=\beta\cdot r_{i-1} -q_{n-i}\cdot b$ \todo{Here the thing under $\beta\cdot r_{i-1}$ is missing}
		\item use short CPA (eg CLA) ro compute approx. remainder $r_i=r_{i,c}+r_{i,s}$
	\end{itemize}
\end{itemize}

\section*{3.2.2020}
\begin{itemize}
    \item SRT-4-Division: \\
    Assume $r=5$ bits, $b=4$ bits are used for quotient digit selection. \\
    In general we could compute the largest diameter of any uncertainly interval and compare it with the length of the overlap regions ($1/12$ in our case). \\
   
    $r: 6$ bits (including sign), $|r|\in [\frac{k}{32},\frac{k+1}{32})$, for $k=0,1,..,32$, $(0,_ _ _ _ _), 1/32 $ is min\\
   
    $b: 4$ bits (normalized), $b\in [\frac{m-1}{16},\frac{m}{16})$, for $m=9,..,16$ $(0,1, _ _ _ _), 1/16 $\\
   
    $\implies$ $|\frac{r}{b}|\in (\frac{k}{2m},\frac{k+1}{2(m-1)})=:I_{k,m}$, $\forall k\in\{0,..,31\}$, $\forall m\in\{9,..,16\}$ \\
   
    diam$(I_{k,m})=\frac{k+1}{2(m-1)}-\frac{k}{2m}=\frac{1}{2}(\frac{m(k+1)-(m-1)k}{m(m-1)})=\frac{m+k}{2m(m-1)}$ \\
   
    Maximal diameter attained for $m=9$, $k=31$ \\
    $\max diam(I_{k,m})=diam(I_{31,9})=\frac{9+31}{2\cdot 9\cdot 8}=\frac{40}{144}=\frac{5}{18}$ \\
   
    each overlap region has length $\frac{1}{12}$ \\
    the one with largest values $[1/3,5/12]$ \\
   
    Since $b$ is normalised: $\frac{1}{2}\le \frac{1}{1}=1$ \\
    Also, $|\frac{r}{b}|\le \frac{2}{3}$
    \begin{itemize}
        \item if $b=\frac{1}{2} \implies |r|=\frac{1}{3}$
        \item in any case $(forall b\in [\frac{1}{2},1))\implies |r|\le \frac{2}{3}$
    \end{itemize}
   
    If $b=\frac{1}{2} \implies m=9 \implies |r|\le \frac{1}{7}<\frac{11}{32} \implies k=10 \from$ largest possible $k$ \\
    $\implies$ $\max \diam (I_{k,m})=\diam(I_{10,9})=\frac{19}{144}>\frac{1}{12}$, but far from any overlap region \\
    Overlap region $[\frac{1}{3},\frac{5}{12}]:$ assume: $|\frac{r}{b}|\le \frac{5}{12}$ \\
    $\implies$ $|r|\le \frac{5}{12}\cdot b\in [\frac{5}{24},\frac{5}{12})$ \\
    Smallest $b(=\frac{1}{2}), m=9 \implies |r|\le \frac{5}{24}<\frac{7}{32} \implies$ largest $k=6$ \\
    $\diam(I_{6,9})=\frac{15}{144}=\frac{5}{48}(>\frac{1}{12})$ \\
   
    Uncertainty interval $(I_{6,9})$ \\
    $\frac{6}{32}\cdot \frac{16}{9}=\frac{1}{3}<\frac{r}{b}>\frac{7}{16}=\frac{7}{32}\cdot \frac{1}{1}(>\frac{5}{12}!)$ \\
    For $k=6, m=10 (b=0.1001) (I_{6,10})$ \\
    $\frac{6}{32}\cdot \frac{16}{10}=\frac{3}{10}<\frac{r}{b}<\frac{7}{18}=\frac{7}{32}\cdot \frac{7}{32}\cdot \frac{16}{9}(<\frac{5}{12}!)$ \\
   
    $\implies$ try $k=7: \frac{7}{31}\le |r| <\frac{1}{4}, b=\frac{12}{5}\cdot |r| \implies \frac{21}{40}\le b<\frac{3}{5}$ \\
    $\implies$ $m=9$ or $10$ \\
   
    for $k=7,m=9, (I_{7,9}): \frac{7}{32}\cdot \frac{16}{9}=\frac{7}{18}<\frac{r}{b}<\frac{1}{2}=\frac{1}{4}\cdot \frac{2}{1}$ \\
    $\diam(I_{7,9})=\frac{16}{144}=\frac{1}{9}>\frac{1}{12}$, but OK!\\
    \\
    for $k=7,m=10, (I_{7,10}): \frac{7}{32}\cdot \frac{8}{5}=\frac{7}{20}<\frac{r}{b}<\frac{4}{9}=\frac{1}{4}\cdot \frac{16}{9}>\frac{5}{12}$ \\
    $\diam(I_{7,10})=\frac{17}{180}>\frac{1}{12}$, but OK!\\
    \\
    for $k=7,m=11, (I_{7,11}): \frac{7}{22}<\frac{r}{b}<\frac{2}{5}<\frac{5}{12}$ \\
    $\diam(I_{7,11})=\frac{18}{220}<\frac{1}{12}$, so, always OK!\\
    \\
    for $m\ge 11 \implies \diam(I_{7,m})<\frac{1}{12}$ always \\
   
    Other cases are less critical! \\
    For $6$ bits of $r$ and $3$ bits of $b$ (leading bit always $=1$), we need a table with $2^6\times 2^3=64\times 8=512$ entries (of (?))\\
   
    \item example \\
    $r=+0.0011$ \\
    $b=+0.1001$ \\
    $\implies$ $0.0011\le r< 0.001 \iff \frac{3}{16}\le r< \frac{1}{4}$ \\
    $\implies$ $0.1001\le b< 0.101 \iff \frac{9}{16}\le b< \frac{5}{8}$ \\
   
    $\implies \frac{3}{16}\cdot \frac{8}{5}=\frac{3}{10}< \fracr{}{b}<\frac{4}{9}=\frac{1}{4}\cdot \frac{16}{9}$ \\
   
    \todo{Here the arrows are missing}
    \\
    $\implies $ overlap region $[\frac{1}{3}, \frac{5}{12}]\subset [\frac{3}{10}, \frac{4}{9}]$ \\
   
    We cannot decide wheather to use $q_{n-1}=+1$ or $+2$ \\
   
    Assume $r=6$ bit , $b=$ bit \\
    \todo{Here the arrows are missing}
    \begin{enumerate}
        \item $r=0.00110: \frac{3}{16}<r<\frac{7}{12}$ \\
        $\frac{3}{16}\cdot \frac{8}{5}=\frac{3}{10}<\frac{r}{b}<\frac{7}{18}=\frac{7}{32}]cdot \frac{16}{9}$
        \item $r=0.00111: \frac{7}{32}\le r<\frac{1}{4}$ \\
        $\frac{7}{32}\cdot \frac{5}{8}=\frac{7}{10}<\frac{r}{b}<\frac{4}{9}=\frac{1}{4}\cdot \frac{16}{9}$ \todo{Here the arrows are missing} 
    \end{enumerate}
    \item radix $\beta$ SRT-division \\
    Given $\beta\in N$ (even radix, usually a power of $2$) and given $s\in N$ (largest absolute value of any digit) and digit set $S:=\{-s,-s+1,..,s-1,s\}$ (symmetric (?)) \\
   
    Assume $\beta/2\le s\le \beta-1$ \\
   
    If we use $|\frac{r_i}{b}|\le x$, then $|\frac{r_{i+1}}{b}|\le \beta \cdot |\frac{r_i}{b}|-s\le \beta \cdot x-s=x$ \\
    \implies $s=(\beta-1)\cdot x \implies x=\frac{s}{\beta-1}$ (constant bounded) \\
   
    \todo{Here the arrows are missing}.
    This is also our measure of redundency. \\
    It is contained in $(\frac{1}{2},1]$ \\
    for minimal $s=\frac{1}{2}: |\frac{r_i}{b}|\le \frac{\beta}{2(\beta-1)} \implies $ small overlap! \\
    $\beta=4: |\frac{r_i}{b}|\le \frac{4}{2\cdot 3}=\frac{2}{3}$ \\
    $\beta=8: |\frac{r_i}{b}|\le \frac{4}{7}$
    for maximal $s=\beta-1: |\frac{r_i}{b}|\le 1\implies$ large overlap!
\end{itemize}

\section*{04.02.2020} \\
Interval $[\frac{-s}{\beta-1}, \frac{+s}{\beta-1}], $length $l=\frac{2s}{\beta-1}$ \\
can be exactly covered with $2\beta$ digits $\implies$ every digit covers a length $\frac{l}{\beta}=\frac{2s}{\beta(\beta-1)}=L$ \\
With $(2s+1)=d$ digits a length of $(2s+1)$ can be covered, i.e. a length of $(2s+1-\beta)\cdot L=(2s+1)\cdot L-l$ remains for overlap area, which is divided in $2s$ areas. \\

$(2s+1)\cdot L=(2s+1)\cdot L-l=d\cdot L-l=d\cdot \frac{l}{\beta}-l=\frac{d-\beta}{\beta}\cdot l=\frac{2s+1-\beta}{\beta}\cdot \frac{2s}{\beta-1}$ \\

With that every overlapping has an area of $\frac{2s+1\beta}{\beta(\beta-1)}$ \\
minimal redundance: $s=\frac{\beta}{2}, d=\beta+1$ \\
$|\frac{r_i}{b}|\le \frac{\beta}{2(\beta-1)}\implies l=\frac{\beta}{\beta-1}$, i.e. can be covered with exactly $\beta$ digits. \\

Every digit covers $L=\frac{1}{\beta-1}$ \\
An additional digit can be divided on (?) \\

Here the tableau is missing. \\
\\
Maximal redundance: $s=\beta-1, d=(2\cdot \beta)-1$\\
$|\frac{r_i}{\beta}|\le 1 \implies I=[-1,+1], l=2$ \\
exactly $\beta$ digits cover $I$ $\implies$ $L=\frac{l}{\beta}=\frac{2}{\beta}$ \\

With $\beta-1$ extra digits one can distribute $\frac{2(\beta-1)}{\beta}$on $2(\beta-1)=d-1$ $\implies$ every overlapping area has length of $\frac{1}{\beta}$.
\todo{Here the tableau is missing}
\begin{itemize}
    \item division through iterative methods:
    \begin{itemize}
        \item Building the inverse $\frac{1}{b}$ with $\frac{1}{2}\le b<1 $(normalised mantissa)
        \item We are searching for: $\frac{1}{b}\in (1,2]$
        \item $f(x)=\frac{1}{x}-b=0$ has a solution $x=\frac{1}{b}$ \\
        Over $x$ there needs to be a dash .
       
        \item Display of $b=[b_{n-1},b_{n-2},...,b_1,b_0]_2$ \\
        We change in $W(b)=\sum_{i=0}^{n-1} b_i\cdot 2^{i-(n-1)} \in [0,2-2^{n-1}]$ \\
        $0=f(x)=f(x)+(x-x)\cdot f^\prime (x)$ (here the dashes are missing. ) \\
       
        Fixed-point equality (Newton):
        $x=x-\frac{f(x))}{f^\prime(x)}\implies$ Newton-iteration :\\
        $x_{i+1}=x_i-\frac{f(x_i))}{f^\prime(x_i)}=x_i-\frac{\frac{1}{x_i}-b}{-\frac{1}{b^2}}=x_i+x_i^2(\frac{1}{x_i}-b)$\\
        $=2x_i-b\cdot x_i^2=(2-bx_i)\cdot x_i\implies $ e multiplications + 1 addition/subtraction. \\
       
        In every iteration the result is calculated with $3-$times the length (given $x_i$), always truncated, so that one stays at the same side of the solution $\to$ $x_{i+1}>x_i$. \\
       
        - error: $\varepsilon_i=\frac{1}{b}-x_i$ \\
        $\varepsilon_{i+1}=\frac{1}{b}-x_{i+1}=\frac{1}{b}-2x_i+bx_i^2=b(\frac{1}{b}-x_i)^2=b\cdot \varepsilon_i^2<\varepsilon_i^2$ \\
        $\implies$ quadratic convergence. \\
        \\
        - For example, for $n=64: 8$ bit inverse from the tableau, then $3$ Newton -iterations $\implies$ $\varepsilon<(((2^{-8})^2)^2)^2=2^{-64}$.
    \end{itemize}
\end{itemize}

\section*{WALTER STUFF}
Time delays of the logic gates and electronic circuits \\
tu:= logic gate time unit\\
$0$ tu for NOT ($\not$, -)\\
$1$ tu for AND ($\cdot$), OR (+), NAND(?), NOR(?) \todo{lots of signs}\\
$2$ tu for XOR(?), HA (half adder) \\
$4$ tu for FA(full adder), but only 3tu for carry! \\
$3$ tu per bit position in carry-ripple +1 tu (to finish final sum bit) \\
$1$ fatu:= 4tu (or 3 tu) = 1 full-adder time unit computing $S_i$ takes $4$ tu\\

$c_{i+1}$ takes $3$ tu: $c_{i+1}=a\cdot (b+c)+b\cdot c$ \\
Here the drawing is missing. \\

In 2`s complement addition \\

Overflow can be detected by remembering? the carry into the leading (sign) bit part? and company? it to the final carry-out overflow occurs $\iff c_n\not = c_{n-1}$ \\

Example: $n=4:$ \\
$0110=+\varepsilon $
$+0110=+\varepsilon$

Here I dont know how to write the line, so its better that you write the things. \\

Computed result is: \\
too small by $2^n$ (positive overflow), too large by $2^n$ (negative overflow) (?Here I dont know if the power is $n$ or $h$?) \\
When wrap-around occurs! \\

$c_{i+1}=G_{0,i}+P_{0,i}\cdot c_0$ \\

$G_{0,i}=q_i+p_iq_{i-1}+p_2p_{i-1}q_{i-2}+..+p_io_{i-1}...p_1p_0$ \\
$P_{0,i}=p_ip_{i-1}...p_1p_0$, where $p_i=a_i+b_i \forall i$ \\

$n=4: (i=3): $ \\
Here the drawing is missing. \\

Roughly speaking, the propagate bit $P_{i,k}$ of a group of $n=k-i+1$ bits can be computer in the same time as it takes one FA to produce s and c ((?) $4$ tu) for all $n\le 8$. \\

Carry-Skip Adder: \\
Build only carry-ripple cicuit? (for groups of bits) with extra hardware to compute propagat bits $P_{i,k}$ for these groups. \\

In order to compute the generate bits $G_{i,k}$ for these groups, we use the C-Ripple circuit(?) where $c_{ik}=?$ is always assumed ($c_{ik}=? \implies c_{out}=G_{i,k}$). \\
Carry-Skip-Adders: irregular (?)\todo{missing} \\
use $k$ groups of lengths $l_k,l_{k-1},..,l_2,l_1$ \\
Example: $n=16:$ \\

Here the drawing is missing. \\

minimal $_{n-16, irr.(?)}=7$ fatu \\

Optimal "regular" design for $n=16:$ \\
$l_{opt}=\sqrt{\frac{n}{2}}=sqrt{8} \implies $ either $l=2$ or $3$ \\
$k_{opt}=\sqrt{2n}=sqrt{32} \implies $ either $k=5$ or $6$  \\
\\
$k=5$ groups: + drawing \\
$T=10 fatu$ \\
\\
$k=6$ groups: + drawing \\
$T=8$ fatu$=2\cdot l +k-2=2\cdot 2+6-2$ \\
\\

Optimal design of irregular comp-skip adder: \\
Here the tableau is missing. \\

\\
Carry-Select adders: \\
regular design: $k$ groups of $l$ bits each $[n=l\cdot k]$ \\
$T=l+(k-1)$, the words under the equality are missing. \\

irregular design: $k$ groups of lengths $l_kl_{k-1}...l_1;$ \\
should use: $[l_1+k-2][l_1+k-3]...[l_1+1][l_1][l_1]$ \\

$T=l_1+k-1$ where $k$ is less than in the regular design! \\
$T=O(\sqrt{n})$\\

Here the tableau is missing. \\

Shifting: $n-$ bit $#s$: \\
$LShift_1(a)$ + drawing
$LShift_k(a)$, $1\le k<n$ + drawing \\
$RShift_k(a)$ + drawing \\

arithmetic right shift: duplicates leading (sign) bit \\
$ARShift_1(a)$ +drawing \\
\\
$\sum_{i=1}^{n-1} a_i \cdot 2^{i-1}-a_{n-i}2^{n-1}=\sum_{i=1}^{n-2} a_i \cdot 2^{i-1}-a_{n-i}2^{n-(?)}=\sum_{i=1}^{n-2} a_i \cdot 2^{i-1}-a_{n-i}2^{n-2}=2$`s complement ..(?)\todo{missing} \\

\\
$1111= -1$ \\
$\to$ $1111=-1=\frac{-1}{2}$ \todo{Here the sign is missing} \\
$1110= -2$ \\
$\to$ $1111=-1=\frac{-2}{2}$ \todo{Here the sign is missing} \\
$1101= -3$ \\
$\to$ $1110=-2=\frac{-3}{2}$ \todo{Here the sign is missing} \\
$1100= -4$ \\
$\to$ $1110=-2=\frac{-4}{2}$ \todo{Here the sign is missing}
Multiplication of $2 n-$ bit (?)\todo{missing}: \\
$a*b$\to$ r$ via repeated summation of partial products $a_i\cdot b:$ \\
$r;+\sum_{i=0}^{n-1} a_i\cdot 2^i\cdot b$ , here the drawing is missing. \\

Double-length register $R=P|A$ (here the box arounf $R$ is missing + drawing. \\

Algorithm: initially, set $B:=b, A:=a $ and $P:=0$ (including $c:=0$) \\
$P_0 n $steps: \\
\begin{enumerate}
    \item $P:=P+a_0\cdot B; ARShift_1(R)$
    \item $P:=P+a_1\cdot B; ARShift_1(R)$
    \item
    \item $P:=P+a_n\cdot B; ARShift_1(R)$
\end{enumerate}
(*) Here i dont know how to write the 3 dots in the enumeration. \\

After $n$ steps, $R=P|A$ (here the box is missing) contains the full double-length product $r=a*b [2n$ bits$]$. P contains approx. product and A (?) exact corresponding error [$\to$ floating -point] \\
\implies $T(n)=O_{CLA}(n\cdot \log n)$ or $O(n\cdot \sqrt{n})$ (The words under are missing.
\\
\\

Possible arguments:
\begin{enumerate}
    \item if we have groups of $k$ (?) in a, then immediatelly do $ARShift_k(R)$
    \item if we have any group of $k$ ibes ub am then $a= ... 01111 ...10....$ (The bracket under is missing) \\
    $\sum_{i=j}^l 2^i=2^{l+1}-a^j=2^{j+k}+-2^j$ \\
    \implies replace $k$ additions by $1+$ and $1-$.
\end{enumerate}

Booth recoding: recode the multiplier a\\
In the $i-$th step:
\begin{enumerate}
    \item $a_i=0$ and $a_{i-1}=0$: do nothing (only (?))
    \item $a_i=0$ and $a_{i-1}=1$: $P:=P+B$
    \item $a_i=1$ and $a_{i-1}=0$: $P:=P-B$
    \item $a_i=1$ and $a_{i-1}=1$: do nothing (only (?))
\end{enumerate}

For the first step, set $a_{-1}=0$. \\
Booth recoding is like adding $B\cdot (a_{i-1}-a_i)\cdot (?)$ in the $i-$th step. \\
\implies Less work only if $k\ge 3$, but that is not the main goal!\\
\bigskip

\section*{NEWPAGE}

$\sum_{i=0}^{n-1} B(a_{i-1}-a_i)\cdot a^i=B[(a_{n-2}-a_{n-1})2^{n-1}+(a_{n-3}-a_{n-2})2^{n-2}+ (a_{n-4}-a_{n-3})2^{n-3}+ ... + (a_{0}-a_{1})2^{1}+(a_{-1}-a_{0})2^{0}$ \\
(Here you just need to do it in scales.. \\

$=B\cdot [a_{-1}+a_0\cdot 2^0+a_1\cdot 2^1+a_2\cdot 2^2+...+a_{n-2}\cdot 2^{n-2}-a_{n-i}\cdot 2^{n-1}=B*A$, where $i$ is a sign bit.\\
(Here the bracket under is missing) \\

Booth recoding might lead to less or the same or more effort to compute the product:  \\
Example: $a=[01010101]_2$ with (?) algorithm $4+$ \\
Booth: $a=[11111111]_2$ with Booth: $4+ & 4-$. \\
$1:=-1$
(Here the dashes above the ones are missing. \\
\implies General 2`s complement multiplicated with a and b of arbitrary sign!! \\

Need:
\begin{itemize}
    \item $ARShift_1$
    \item store the most recently shifted -out bit of a [$\to$ $a_{-1}$]
    \item use 2`s complement adder [for $P+B & P-B$]
    \item ignore (throw away) $c_{out}$[eliminate extra]
\end{itemize}

Example: $n=4:$ multiply $(-6)\cdot (-5)$, where $a:=-6, b:=-5$ \\
$b=-5=1011_2$\\
$a=-6=1010_2$\\
$-b=+5=0101_2$ \\
Here the tableau is missing. \\
\bigskip

\section*{NEWPAGE}
Booth recoding uses signes bits, i.e. the digit set $D:=\{-1,0,+1\}$. \\
To reduce the # of iterations, i.e. higher radix (basis), e.g. $2^2=4$ or $2^3=8$. \\

Minimal Symmetric digit sets are: \\
$S=\{-2^{k-1},-2^{k-1}+1,..,0,..,+2^{k-1}\}$ when using radix $2^k$, i.e. $k-$bit groups. \\
This reduces the # of iterations to $\frac{n}{k}$. Here the sign is missing. \\

Example: If we use $k=1$ bits at a time, we will need $\frac{n}{2}$ (the sign is missing) iterations using $S=\{-2,-1,0,1,2\}$. This radix $4$ Booth recoding also requires last shifted -out bit: \\
(Here the tableau is missing. \\

Radix $4=2^2$ only requires $LShift_1$ (for $+-2B$) (The signs are supposed to be one over another), and `s complement of $B$ ($B$ (with dash) for negative (?)). \\

Table of multiples of $B$:\\
Here the tableau is missing. \\

Precomputing multiples of $B$ typically requires only $2$ (or maybe $3$) operations ($LShifts & $ addiion). Larger radixes are not very useful ([normally $2^2$ or (?)]). \bigskip

\section*{NEWPAGE}

How to add many $(k), n-$digit #s: +drawing: \\

Using $n$ FAs, we can add $2($ or $3) n-$digits #s $D_1=[d_{n-1}^1d_{n-2}^1...d_{1}^1d_{0}^1]_2$ and $D_2=[d_{n-1}^2d_{n-2}^2...d_{1}^2d_{0}^2]_2$ in the first FA-cycle $(T=1$ fatu$)$ and then $1$ additional $D_i=[d_{n-1}^id_{n-2}^i...d_{1}^id_{0}^i]_2$ in every FA-cycle until the last # $D_k$ has been added (after $(k-1)$ or $(k-2)$ fatu).  \\

To finish all carry propagation requires an extra $(n-1)$ fatu, so we need a total of $T(n,k)=n+k-2$ fatu, instead of (or $n+k-3$) \\
$O(n\cdot k)$ fatu if carries are (?) every time! \\

Here the 2 drawings is missing. \\
\bigskip

\section*{NEWPAGE}
Wallace Tree: + drawing \\

Wallace Tree: (radix $2$) \\
$T(n):=T_{Wallace}(n)+T_{CLA}(2)=O(\log_{\frac{3}{2}} n+\log_2 n)=O(\log n)$ + drawing
\bigskip

\section*{NEWPAGE}
Multiplication ($n$ bit \times $n$ bit)\\
Iterative:\\
with little HW: \\
\begin{itemize}
    \item $1 CRippleA: T=O(n^2)$
    \item  $1 CSkipA|CSelectA: T=O(n\cdot \sqrt{n})$
    \item $1$ CLA (binary tree): $T=O(n\cdot \log n)$
    \item $1$ CSA (carry (?) adder): $T\approx (n-2)+(2n)\approx 2n=O(n)$\\
    Here the words under the approximation are mising.
\end{itemize}

With substantially more HW: use $(n-2)$ CSA plus $1$ CLA. \\
$\to$ for single multiplication: $(n-2)$ CSA + $1$ CLA \\
(one-dimensional layout) $T_1=O(n+\log n)=O(n)$\\
\\
$\to$ pipelined:  $T_p=\frac{T_1}{\text{# pipeline stages}}=\frac{O(n)}{O(n)}=O(1)$, (Here the bracket under is missing) \\

Wallace tree layout: \\
$\to$ (?) : $-(n-2)$ CSA $+1$ CLA:  $T_1\spprox \log_{\frac{3}{2}} n+2\cdot \log_{2} n=O(\log n)$ \\
$\to$ pipelined: $T_p\frac{T_1}{O(\log n)}=O(1)$ \\
Speedup \approx # of pipeline stages: $O(\log n)$. \bigskip

\section*{NEWPAGE}
Division: $q:=(?)(a/b)$ with remainder $r$ where we assume $a\ge 0 $ and $b>0$. Result must satusfy $a=q\cdot b+r$. Iterative division produces $n$ partial remainders starting with $r_0:=a$ and resulting in $q:=[q_{n-1}q_{n-2}...q_{1}q_{0}]$ and "find" remainder $r:=r_n$ in $n$ steps (iterations). + drawing. \\

Restoring division: \\
Initialisation: $a$\to$ A, b$\to$ B$ possible precompute (?) $0$\to$ P$ \\

For $i=1,2,..,n:$ \\
\begin{enumerate}
    \item $LShift_1(R) $[all $2n+1$ bits]
    \item $P:=P-B$
    \item If $P<0$, set (?) position of $A$ to (?), otherwise to $1$ [i.e. $q_{n-i}:=p_n$] (?) (Over $p$ a dash is missing).
    \item If $P<0$, restore old $P:=P+B$.
\end{enumerate}
After $n$ steps, A contains q and P contains (?). \\

Nonresting Division: samo init-phase. \\
For $i=1,2,..,n:$ \\
If $P<0$:
\begin{enumerate}
    \item $LShift(R)$ [$2n+1$ bits]
    \item $P:=P+B$
\end{enumerate}
Else ($P\ge 0$):
\begin{enumerate}
    \item $LShift(R)$
    \item $P:=P-B$
\end{enumerate}
EndIf.\\
(Here the 1 and 2 have to be with signs + and -).\\

3. Set quotient digit
$$
q_{n-i}:=
\begin{cases}
0, & \text{for } 0>P\\
1,  & \text{for } P\ge 0
\end{cases}
$$
After $n$ steps, \\
4. If $(P<0)$, restore temainder by $P:P+B$, (?) making $P\ge 0$. \bigskip

\section*{NEWPAGE} \\

Why are restoring and (?) division equivalent: \\
In step $k$, $q_{n-k}$ is based on sign of $2r_{k-1}-2^n\cdot b$ in restoring division. \\
I. If $2r_{k-1}-2^n\cdot b \ge 0$, both algorithms do the same: new partiel remainder $r_k:=2r_{k-1}-2^n\cdot b$ and $q_{n-k}:=1$. \\
II. If $2r_{k-1}-2^n\cdot b < 0$, restoring div. sets $r_k:=2r_{k-1}$ and in step $k+1$, $q_{n-(k+1)}$ is based on sign of $2\cdot r_k-2^n\cdot b=4\cdot r_{k-1}-2^n\cdot b$, whereas in nonrestoring division, we keep the negative partial remainder $r_k=2r_{k-1}-2^n\cdot b[<0]$ and in step $(k+1, q_{n-(k+1)}$ is based on sign of $2\cdot r_k+2^n\cdot b=2\cdot (2r_{k-1}-2^n\cdot b)+1^n\cdot b=4\cdot r_{k-1}-2^n\cdot b$, so both algorithms do the same thing, i.e. produce the same quotient digits, but possibly a different (?) of partial rem.(?). If after $n$ steps (in nonrestoring division) $r_n<0$, then restore it to $r_n+2^n\cdot [\ge 0]$. \\

Example: $n=4:$ restoring division: + tableau. \\
\bigskip

\section*{NEWPAGE} \\
drawing \\

$q= 01000110=70$ \\
$r= 00000001=1$ \\
$a=211=0.11010011\cdot 2^8$ \\
$b=3=0.00000011\cdot 2^8$ \\
Maybe you could do these so the digits fall in the same column?\\

\\

Redundant number representation\\
radix(=base): $\beta\ge 2, \beta\in N$. \\

A number representing system is redundant if the digit set $S$ contains more than $\beta$ digits. We assume that (?) digit set $S$ contains (?) and its digits are (?) (no holes): \\

$S=\{d,d+1,..,d\}$ where $d\le 0\le (?)$ \\

In practise we only consider digit sets $S$ with $d\ge 1-\beta$ and $d\le \beta-1$ (here a dash over d is missing). \\

A symmetric digit set satisfies $d=-d$ (dashes are missing). \\

A minimally readundant digit set S contains exactly $\beta+1$ digits. \\

A minimally redundant symmetric digit set (?): \\
- for $\beta$ even is $S=\{-\frac{\beta}{2},-\frac{\beta}{2}+1,...,+\frac{\beta}{2}\}$ \\
- for $\beta$ odd is $S=\{-\frac{\beta+1}{2},...,+\frac{\beta+1}{2}\}$ \\

Example: "signed bits": $S=\{-1,0,1\}$\\
minimum (?) summetric radix $4$: $S=\{-2,-1,0,1,2\}$. \bigskip

\section*{NEWPAGE}
SRT Division: (?) independently\\
by D. (?) \\

SRT is a class of division algorithms characterized by: \\
\begin{itemize}
    \item divisor $b$ is initially "normalized" (by $LShift_k$)
    \item requirement for redundant symmetric digit set for quotient digits $q_{n-i} [q_{n-1}q_{n-2}...q_{0}]$
    \item quotient digit selecton based on only a few leading digits of partial remainder $r_{i-1}$ and of the divisor $b$. \\
    \implies use of a ($2-$ dimension) lookup table is possible!\\
    \item partial remainders $(r_i)$ are represented in a redundant number system, e.g. as $2 n-$ bit #s (useful for CSA). \\
\end{itemize}

Simple radix $2$ SRT division: $\frac{a}{b}-(?)$ \\
Assume $b\not = 0$ \\
Think of $a$ and $b$ as floating-point numbers with the binary point just to the left of (?) P|A (box is missing) and B. \\

1. Determine number of leading zeros $k$ in B ($n$ bit) and $LShift_k(B)$, $LShift_k(R)$, [R=P|A].\\

Note: $0\le k\le n-1$ (since $b\not =0$) \implies leading bit of P in (?) \implies $|r_0|\le \frac{1}{2}$. \\

2. For $i=1,,.,n$:\\
\begin{itemize}
    \item If leading $3$ bits of P are all equal (or (?)), set $q_{n-i}:=0$ and $LShift_1(R)$.
    \item If leading $3$ bits of P are not all equal and $P<0$, set $q_{n-i}:=-1=T$ and $LShift_1(R)$ and $P:=P+B$.
    \item If leading $3$ bits of P are not all equal and $P\ge 0$, set $q_{n-i}:=+1$ and $LShift_1(R)$ and $P:=P-B$.
\end{itemize}

3. If final remainder $P<0$, restore it: $P:=P+B$ and correct the quotient $q:=q-1$. (here the bracket underneath is missing). \\
[substraction of $1$ in last position $a_0$]. \\

4. $r_n$ is obtained $ARShift_k(P)$ (?). \bigskip

\section*{NEWPAGE} \\
Analysis of algorithms: \\
Via initial $LShift_k(B)$, the value stored in B (as floating point number) has $|B|\ge \frac{1}{2}$ and $|B|<1$. \\

Initial remainder $r_0:=LShift_k(R)$ satisfies $-\frac{1}{2}\le r_o\le \frac{1}{2}$. \\
(Here the thing above R is missing). \\

In steps: \\
Wuotient-digit selection:
$$
X(m,n)=
\begin{cases}
1, & \text{if } \frac{1}{4}\le r+{i-1}<\frac{1}{2}\\
0, & \text{if } -\frac{1}{4}\le r_{i-1}<\frac{1}{4}\\
-1, & \text{if } [-\frac{1}{2}\le] r_{i-1}<-\frac{1}{4}
\end{cases}
$$

New remainder $r_i:=2\cdot r_{i-1}-q_{n-i}\cdot b$ remains bounded: \\
(The arrow under 2 is missing). \\

$-\frac{1}{2}\le r_i< \frac{1}{2}$ \\
because: If $q_{n-i}=1(r_{i-1}\ge+\frac{1}{4})\implies$ \\
$r_i\le 2\cdot [\frac{1}{4},\frac{1}{2})-[\frac{1}{2},1)=[-\frac{1}{2},\frac{1}{2})$ \\
$q_{n-i}=0:$ obvious! \\
If $q_{n-i}=-1(r_{i-1}<-\frac{1}{4})\implies r_i\in[-\frac{1}{2},\frac{1}{2})$ \\

So SRT div. keeps $|\text{remainder}|\le \frac{1}{2}$, but only does work with $r\ge \frac{1}{4}$ or $r<-\frac{1}{4}$. \\
(The bracket under is missing). \\

\implies P only required $n$ bits so eliminate extra leading bit in P (and B). \\
(?) steps 2.a) and 2.b) only (?) bits are tested. \\

P-D-diagrams (partial -remainder division) + drawings. \bigskip

\section*{NEWPAGE} \\
SRT $4$ division: computer $2$ bits of quotient per iteration minimally redundant symmetric digit set $S=\{-2,-1,0,1,2\}$. \\

$\frac{r_{i+1}}{b}=4\cdot \frac{r_i}{b}-q_i$. (Here radix 4 is missing). \\

1. If $|r_i/b|\le 1 \implies |\frac{r_{i+1}}{b}|\le 2\implies|\frac{r_{i+2}}{b}| \le 4$\\
requires an extra bit for rem. per iteration, remainder becomes larger & larger \implies correction of quotient via successive bits is generally impossible!\\
2. If $|r_i/b|\le \frac{3}{4}\implies |\frac{r_{i+1}}{b}|\le 1\implies$ case 1!\\
3. If $|r_i/b|\le \frac{1}{2} \implies $?\\
\implies $$
4\cdot \frac{r_i}{b} \in
\begin{cases}
(\frac{3}{2},2], & \text{\implies } q_{n-(i+1)}=2\\
(\frac{1}{3},\frac{3}{2}], & \text{\implies } q=1\\
[-\frac{1}{2},\frac{1}{2}], & \text{\implies } q=0\\
[-\frac{3}{2},-\frac{1}{2}), & \text{\implies } q=-1\\
[-2,-\frac{3}{2}), & \text{\implies } q=-2
\end{cases}
$$
to ensure $|\frac{r_{i+1}}{b}|\le \frac{1}{2}$ also, etc. \\

(?) No overlap of these ranges \implies requires exact division $\frac{r_i}{b}$, \implies no fast quotient selection, no table (?).\\

4. If $|\frac{r_{i}}{b}|\le \frac{2}{3} \implies |\frac{r_{i+1}}{b}|\le |\frac{2}{3}-2|\le (?)$ \\

\implies best choice with maximal ovelap. + drawing \\

$4$ overlap regions of length $\frac{1}{12}$ each! \\
Assume that $5$ bits of $r$ and $4$ bis of $b$ are used for quotient digit selection: \\
\bigskip

\section*{NEWPAGE}\\
In general, we could compute the largest diameter of any uncertainty interval and compare it with the length of the overlap regions (\frac{1}{12} in our case). \\

$r: 6$ bits (including sign) \\
$|r|\in [\frac{k}{32},\frac{k+1}{32})$, for $k=0,1,..,32$ \smallskip

$b: 4$ bits (normalised) \\
$b\in [\frac{m-1}{16},\frac{m}{16})$, for $m=9,..,16$ \\
(This is supposed to be in a table).

$|\frac{r}{b}|\in (\frac{k}{2m},\frac{k+1}{2(m-1)}):=I_{k,m}$ $\forall k=0,..,31, \forall m=9,..,16$. \\

$\diam(I_{k,m})=\frac{k+1}{2(m-1)}-\frac{k}{2m}=\frac{1}{2}(\frac{m\cdot (k+1)-(m-1)\cdot k}{m(m-1)})$ \\
\implies $\diam (I_{k,m})=\frac{m+k}{2m(m-1)}$. \\

Maximum diameter attained for $m=9$ and $k=31$. \\

$\max \diam(I_{k,m})=\diam(I_{1,9})=\frac{9+31}{2\cdot 9\cdot 8}=\frac{40}{144}=\frac{5}{18}$.  (Here m,k under max is missing). \\

Each overlap region has length $\frac{1}{12}$; the one with largest absolute values $[\frac{1}{3},\frac{5}{12}]$. \\

Since $b$ is normalized, $\frac{1}{2}\le b<1$. \\
Also $|\frac{r}{b}|\le \frac{2}{3}$, so if $b=\frac{1}{2}$, then $|r|\le \frac{1}{3}$ and in any case $[\forall b\in [\frac{1}{2},1)]$, $|r|\le \frac{2}{3}$. \\

If $b=\frac{1}{2}$, then $m=9 \implies |r|\le \frac{1}{3}<\frac{11}{32}$ \implies largest possible $k=10$. \\

So if $b=\frac{1}{2}$, $\max \diam (I_{k,m})=\diam(I_{10,9})=\frac{19}{144}$, which is $>\frac{1}{12}$, but far from any overlap region.  \\

Overlap region $[\frac{1}{3}, \frac{5}{12}]: $ assume $|\frac{r}{b}|\le \frac{5}{12} \implies |r|\le \frac{5}{12}\cdot b\in [\frac{5}{24}, \frac{5}{12})$. \\

smallest $b(=\frac{1}{2}), m=9\implies |r|\le \frac{5}{24}<\frac{7}{32}\implies$ largest $k=6$: \\

$\diam(I_{b,9})=\frac{15}{144}=\frac{5}{48}[>\frac{1}{12}]$ \\

Uncertainty interval\\
$I_{6,9}: \frac{6}{32}\cdot \frac{16}{9}=\frac{1}{3}<\frac{r}{b}<\frac{7}{16}=\frac{7}{32}\cdot \frac{2}{1}>\frac{5}{12}!$ \\
Here the brackets under are missing . \\

for $k=6, m=10 b=0.1001..._2$: \\
$I_{6,10}: \frac{6}{32}\cdot \frac{16}{10}=\frac{3}{10}<\frac{r}{b}<\frac{7}{18}=\frac{7}{32}\cdot \frac{16}{9}<\frac{5}{12}$.\bigskip

\section*{NEWPAGE}
\begin{itemize}
    \item $k=7: \frac{7}{32}\le |r|<\frac{1}{4}, b=\frac{12}{5}\cdot |r| \implies \frac{21}{40}\le b<\frac{3}{5} \implies m=9 $ or $10$. \item $m=9: I_{7,9}: \frac{7}{32}\cdot \frac{16}{9}=\frac{7}{18}<\frac{r}{b}<\frac{1}{2}=\frac{1}{4}\cdot \frac{2}{1}$ \\
    $\diam(I_{7,9})=\frac{16}{144}=\frac{1}{9}>\frac{1}{12}$, but OK! (arrow missing).
   
    \item $m=10: I_{7,10}: \frac{7}{32}\cdot \frac{8}{5}=\frac{7}{20}<\frac{r}{b}<\frac{4}{9}=\frac{1}{4}\cdot \frac{16}{9}>\frac{5}{12}$ \\
    $\diam(I_{7,10})=\frac{17}{180}>\frac{1}{12}$, but OK! (arrow missing).
   
    \item $m=11: I_{7,11}: \frac{7}{32}\cdot \frac{16}{11}=\frac{7}{22}<\frac{r}{b}<\frac{2}{5}=\frac{1}{4}\cdot \frac{8}{5}<\frac{5}{12}$ \\
    $\diam(I_{7,11})=\frac{18}{220}<\frac{1}{12}$, so always OK! (arrow missing).
    \end{itemize}
    for longer $m\ge 11$, $\diam(I_{7,m})<\frac{1}{12}$ always! $=\frac{7+m}{2m(m-1)}$. \\
    etc. other (?) less critical! \\
    For $6$ bits of $r$ and $3$ bits of $b$ (leading bit always $=1$) we need a lookup table with $2^6\times 2^3=64\times 8=512$ (?) (at most). +table $\to$do{missing}
   
    \section*{NEWPAGE} \\
    $\to$do{Just 2 tables.}
    \bigskip
   
    \section*{NEWPAGE} \\
    $\to$do{graph +} \\
   
    Radix $\beta$ SRT division:\\
    Given $\beta$ (even radix usually a power of $2$) and given s (largest absolute value of (?) digit) and digit set $S:=\{-s,-s+1,..,s-1,(?)\}$ [symmetric & redundant], we assume that $\frac{\beta}{2}\le \beta-1$. \\
   
    If we use $|\frac{r_i}{b}|\le x$, then $|\frac{r_{i+1}}{b}|\leqls
     \beta \cdot |\frac{r_i}{b}|-s\le \beta\cdot x-s = (?) $ (Here the arrows are missing). \\
     
     \implies $s=(\beta-1)\cdot x\implies x=\frac{s}{\beta-1}$ const. (?)\\
     This is also our measure of redundance. \\
     It is contained in $(\frac{1}{2},1]$. \\
     For minimal $s=\frac{1}{2}: |\frac{r_i}{b}|\le \frac{\beta}{2(\beta-1)}\implies (?)$ \\
     
     $\beta=4: |\frac{r_i}{b}|\le \frac{4}{2\cdot 3}=\frac{2}{3}$;\\
     $\beta=8: |\frac{r_i}{b}|\le $(?);\\
     
     For maximal $s=\beta-1: |\frac{r_i}{b}|\le 1\implies$ large (?).\bigskip
     
     \section*{NEWPAGE}\\
     
     Counterexample: $r=+0.00011 ..._2$ \\
     $b=+0.1001..._2$ \\
     $\frac{3}{16}\le r<\frac{1}{4}$ \\
     \implies $0.00011\le r<0.001 \iff \frac{3}{32}\le r<\frac{1}{8}$ (here the line over the 0 is missing). \\
     
     $0.1001\le b<0.101 \iff \frac{9}{16}\le b<\frac{5}{8}$ \\
     
     $\frac{3}{18}\cdot \frac{8}{5}=\frac{3}{10}<\frac{r}{b}<\frac{4}{9}=\frac{1}{4}\cdot \frac{16}{9}=0.3<\frac{1}{3}$\\
     
     $0.444...>\frac{5}{12}=0.41666...$ \\
     
     \implies overlap region $[\frac{1}{3},\frac{5}{12}]\subset [\frac{3}{10},\frac{4}{9}]$ \\
     \implies we cannot decide wheather to use $q_{n-1}=+1$ or $+2$!??? \\
     would have required lookup table of $32\times 8$ entries.  \\
     
     Assume we use $b$ bits of $r$ (and still $4$ of $b$): \\
     \implies for case above we now have $2$ cases:
     \begin{enumerate}
         \item $r=0.00110..._2$: $\frac{3}{16}\le r<\frac{7}{32} $ \\
         $\frac{3}{16}\cdot \frac{8}{5}=\frac{3}{10}<\frac{r}{b}<\frac{7}{18}=\frac{7}{32}\cdot \frac{16}{9}$\\
         $\frac{9}{10}<\frac{1}{3}$, $0.3888..<\frac{5}{12}=0.416$ (dash over 6). \\
         
         \item $r=0.00111..._2$: $\frac{7}{32}\le r<\frac{1}{4} $ \\
         $\frac{7}{32}\cdot \frac{8}{5}=\frac{7}{20}<\frac{r}{b}<\frac{4}{9}=\frac{1}{(?)}\cdot \frac{16}{9}$\\
         $\frac{1}{2}<0.35$, $\frac{5}{(?)}=\frac{4}{(?)}$ \\
         
         \end{enumerate}