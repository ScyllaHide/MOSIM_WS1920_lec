% !TeX spellcheck = en_US
\section{numerical quadrature and integration}
\subsection{Interpolation quadrature formulas}
	\begin{*notation}
		We will will make a few convenient notations
		\begin{itemize}
			\item $\powerset_n=$ set of polynomials of degree $\le n$
			\item $[n]$ for the set $\set{0,1,\dots,n}$
			\item the natural numbers $\N$ always include the 0
			\item We will use a few abbreviations: interpolation \text polynomial with ``Ipolpoly'', Newton-Cotes formula as ``NCF'' \todo[inline]{more please add here!}
		\end{itemize}
	\end{*notation}
We have $(n+1)$ \emph{base points} (Stützstellen) $(x_i,y_i)$, $i \in \N_0$ with abscissa canonical order on $(x_i,y_i)$ base points. Values are represented by function $f\colon y_i=f(x_i)$.\\
We need a polynomial $p(x) \in \powerset_n \with p(x_i) = y_i$ for all $i \in \N_0$.
\begin{*definition}[\person{Lagrange}-\person{Newton} interpolation polynomial]
	There $\exists!$ solution $p \in \powerset_n$, for $j \in [n], l_j \in \powerset_n$ and we define $l_j\colon \R \to \R$
	\begin{align*}
	l_j(x_j) &= \prod_{\stackrel{i=0}{i\neq j}}^n \frac{(x-x_i)}{x_j-x_i} = \frac{(x-x_0)(x-x_1)\cdots(x-x_{j-1}(x-x_{j+1}))\cdots (x-x_n)}{(x_j-x_0)(x_j-x_1)\cdots(x_j-x_{j-1}(x_j-x_{j+1}))\cdots (x_j-x_n)}\\
	l_j(x_i) &= \delta_{ij} \quad \forall j \in [n]\\
	p(x) &:= \sum_{j=0}^n y_j \cdot l_j(x) \in \powerset_n \nd p(x_i) = y_i l_i(x_i) = y_i
	\end{align*}
	is the \begriff{interpolation polynomial}
\end{*definition}
Assumption: $p,q \in \powerset_n$ and $p(x_i) = q(x_i)= y_i \forall i \in [n]$, then we have
\begin{align*}
	r = p-q \in \powerset_n \nd r(x_i) = p(x_i) = p(x_i)-q(x_i) \quad \forall i \in [n]
\end{align*}
and this repsentation is unique. (Why?)
\begin{*definition}[weight function]
	Let $f\in \CC[a,b]$ be integrand, $\omega \in \CC[a,b]$ weight function with $\omega(x) \ge 0$, $0 < \int_a^b \omega(x) \d x < \infty$ (often $\omega(x) \equiv 1$). If $f$ has $n+1$ base points with $a \le x_0 <\dots < x_n \le b$. We know, that we can set
	\begin{align*}
	I(p_n) = \int_a^b p_n(x)\cdot \omega(x)\d x \approx I(f)
	\end{align*}
	as approximation calculated with $p_n \in \powerset_n$.
\end{*definition}
		The Ipolpoly for the base points $x_j, f(x_j)$, $j  \in [n]$
		\begin{align*}
			p_n(x) = \sum_{j=0}^n f(x_j)\cdot l_j(x)
		\end{align*}
\begin{*definition}[quadrature formula]
	The quadrature formula is defined by
	\begin{align*}
	Q_n(f) &:= I(p_n) = \int_a^b\brackets{\sum_{j=0}^n f(x_j)\cdot l_j(x)}\omega(x)\d x\\
	&= \sum_{j=0}^n\brackets{f(x_j)\int_a^b l_j(x)\omega(x)\d x} = \sum_{j=0}^n \omega_j \cdot f(x_j)
	\end{align*}
	with \emph{weights}
	\begin{align*}
	\ww_j := \int_a^b l_j(x)\cdot \ww(x)\d x
	\end{align*}
\end{*definition}
		$Q_n(f)$ is the weighted sum of the function values for the base point value with fixed weights $\ww_j$. The weights only dependon the base points, the intervall bounds $a,b$ and the weight function $\ww$.
\begin{*definition}[Error]
	The errors for a quadrature formula is defined by $f(x) = p_n (x) + e_n(x)$ with $p_n, e_n \in \powerset_n$, $e \equiv 0$, if $f \in \powerset_n$. If $f \in \CC^{n+1}[a,b]$, then
	\begin{align*}
		e_n(x) &= \frac{f^{n+1}(\xi)}{(n+1)!}W_n(x) \with W_n(x) = \prod_{i\in [n]} (x-x_i)\quad \xi \in [a,b]
		\intertext{and}
		I(f) &= \int_a^b f(x)\ww(x)\d x = Q_n(f) + E_n(f) \with\\
		Q_n(f) &= \int_a^b p_n(x)\ww(x)\d x \nd E_n(f) = \int_a^b e_n(x)\ww(x) \d x = \int_a^b \frac{f^{n+1}(\xi)}{(n+1)!}W_n(x)
	\end{align*}
\end{*definition}
\begin{*definition}[Degree of accuracy]
	A quadrature formula $Q(f)$ has a \begriff{Degree of accuracy} $k$ if
	\begin{align*}
		I(f) = Q(f) \quad \forall f \in \powerset_n \nd \exists g \in \powerset_{k+1}\colon E(g) \neq 0
	\end{align*}
	Equivalent is: With $E(f) := I(f) - Q(f)$ holds \todo[inline]{here is a typo in the $E$ chain!}
	\begin{align*}
		E(1) = E(x) = E(x^2) = \dots = ...
	\end{align*}
\end{*definition}
\subparagraph{Newton-Cotes formulas (overview)}
Let $h_n = (b-a)/n$ with $[a,b]$ and $n \in \N$.
Let $f\colon \R \to \R$ and $\xi \in (a,b)$.\\
% Newton Cortez Formulas with an example(runge)
% !TeX spellcheck = en_US
  \begin{landscape}
  	\newcolumntype{Y}{>{\centering\arraybackslash}X} % centering for X-cols
%    \thispagestyle{empty} % probably to not use the predefined page style.
    \begin{tabularx}{\linewidth}{|c|c|Y|c|Y|Y|} %{\textwidth}{@{}llYlYl@{}} this part was for centering a few cols, but wasnt working
      \hline
      method & $n$ & weights & $\sum$ & error $E$ & degree of accuracy \\
      \hline
      Trapez & 1 & 1,1 & 2 & $-1/12 h_1^3 f''(\xi)$ & 1\\
      \person{Simpson} (\person{Kepler}) & 2 & 1,4,1 & 6 & $-1/30 h_3^5 f^(4)(\xi)$ & 3\\
      3/8 rule & 3 & 1,3,3,1 & 8 & $-1/80 h_3^5 f^(4)(\xi)$ & 3\\
      \person{Milne} & 4 & 7,32,12,32,7 & 90 & $-8/945 h_4^7 f^(6)(\xi)$ & 5\\
      \person{Milne} & 5 & 19,75,51,51,75,19 & 288 & $-275/12096 h_5^7 f^(6)(\xi)$ & 5\\
      \person{Weedle} & 6 & 41,216,27,27,216,41 & 288 & $-9/1400 h_5^9 f^(8)(\xi)$ & 7\\ \hline
    \end{tabularx}
	\begin{*remark}
		For $n \ge 8$ we get negative weights, numerical instability (loss of significance). The Newton-Cotes formulas will be used in general with orders $n=1,2$ (\person{Simpson}!) or order 4. So we use also \emph{summed variants} (split interval and use different orders). Open NCF use midpoints with sub intervals (von $[a,b]$) $\to$ $n$ base points.
	\end{*remark}
	\begin{*example}[\person{Runge}]
			$f:\R\to\R$, $f(x)=\frac{1}{1+25x^2}$ \\
	equidistant base points $x_0,...,x_n$, $p\in\Pi_n$ as Ipolpoly
	\begin{center}
		\begin{tabular}{l|p{8cm}}
			\textbf{base points} & \textbf{interpolated polynomial} \\
			\hline
			2 & $1-\frac{25x^2}{26}$ \\
			\hline
			4 & $3,31565x^4 - 4,27719x^2 + 1$ \\
			\hline
			8 & $53,6893x^8 - 102,815x^6 + 61,3672x^4 - 13,203x^2 + 1$ \\
			\hline
			16 & $15403,1x^{16} - 49713,5x^{14} + 63743,8x^{12} - 41870x^{10} + 15206x^8 - 3100,35x^6 + 351,984x^4 - 22,7759x^2 + 1$
		\end{tabular}
	\end{center}
	\begin{center}\begin{tikzpicture}
		\begin{axis}[
		xmin=-1, xmax=1, xlabel=$x$,
		ymin=-1.5, ymax=3, ylabel=$y$,
		samples=1200,
		axis y line=middle,
		axis x line=middle,
		width=0.9\textwidth,
		height=0.5\textheight,
		restrict x to domain=-1:1,
		restrict y to domain=-1.5:2
		]
		\addplot+[mark=none, line width=1mm] {1/(1+25*x^2)};
		\addlegendentry{Runge-function}
		\addplot+[mark=none] {1-(25*x^2)/(26)};
		\addlegendentry{interpolation 2 base points}
		\addplot+[mark=none, color=darkgreen] {3.31565*x^4 - 4.27719*x^2 + 1};
		\addlegendentry{interpolation 4 base points}
		\addplot+[mark=none, color=black] {53.6893*x^8 - 102.815*x^6 + 61.3672*x^4 - 13.203*x^2 + 1};
		\addlegendentry{interpolation 8 base points}
		\addplot+[mark=none, color=lime] {15403.1*x^16 - 49713.5*x^14 + 63743.8*x^12 - 41870*x^10 + 15206*x^8 - 3100.35*x^6 + 351.984*x^4 - 22.7759*x^2 + 1};
		\addlegendentry{interpolation 16 base points}
		\end{axis}
		\end{tikzpicture}
	\end{center} % tikz picture does compile very long!
	\end{*example}
  \end{landscape}
\begin{*remark}
	\person{Tchebychev} has improve this, compare Tschebychev polynomials.
	There exists a MATLAB package named \texttt{chebfun} by \person{Nick Trefethen}. \footnote[1]{\url{https://www.chebfun.org/}}
\end{*remark}
How do we calculate this now?
\subparagraph{Calculation of the factors of the NCF-remainders}
\begin{*example}
	We will take a look at two ``rules'' to understand the \textit{modus operandi}.\\
	We have $I = Q+E$ with
	\begin{align*}
		Q &= (b-a)\sum_{i=0}^n \ww_i \cdot f(x_i)\\
		E &= c(b-a)^{k+2}\cdot f^{k+1}(\xi) \with k = 2 \floor{n/2} +1
	\end{align*}
	\begin{itemize}
		\item Trapoid rule $(n=k=1)$:
		\begin{align*}
			Q &= 1/2 (f(a) + f(b)(b-1)\\
			E &= c(b-a)^3\cdot f''(\xi)\\
			Q(f) &= I(f)\forall f \in \powerset_1, Q(f) + E(f) \forall f \in \powerset_2, \text{ since } f'' \text{const}\\
			\intertext{set $f = x^2/2 \nd f' = 1$}
			\int_a^b x^2/2 &= x^3/6\mid_a^b = 1/6(b^3-a^3) = 1/6(b-a)(b^2 +ab + a^2)\\
			Q &= 1/4(b-a)(b^2 + a^2)\\
			E &= c(b-a)(b-a^2) = c(b-a)(b^2 - 2ab + b^2) \overset{\text{!}}{=} I - Q
		\end{align*}
		and we find $c = -1/12$
		\item Simpson rule $(n=2, k = 3)$, then $f^{(4)} = 1$. After a bit algebra we find $c = -1/2880 ? -(1/2)^5\cdot 1/90$
	\end{itemize}	
\end{*example}
\subsection{Extrapolation methods (\person{Richardson} Extrapolation $\to$ \person{Romberg}-integration)}
\subsection{\person{Gauss}-method}
\end{enumerate}