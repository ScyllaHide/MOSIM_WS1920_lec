\documentclass[ngerman,a4paper,order=firstname]{mathscript}
\usepackage{mathoperators}
\usepackage{bbm} 							% hbb one


% % % local commands
\DeclareMathOperator{\Ad}{Ad}				% Adjoint
\DeclareMathOperator{\PSL}{PSL} 			% projective linear group 
\newcommand{\with}{\text{ with }}
\newcommand{\nd}{\text{ and }}
\renewcommand{\rhd}{\triangleright}
\renewcommand{\lhd}{\triangleleft} 			% normal subgroups
\DeclareMathOperator{\Set}{Set}				% Category of sets
\DeclareMathOperator{\Vect}{Vect}			% Category of vector spaces
\DeclareMathOperator{\Grp}{Grp}				% Category of groups
\DeclareMathOperator{\Mod}{Mod}				% Cat of moduls
\DeclareMathOperator{\Ann}{Ann}				% annihilator
%\DeclareMathOperator{\diam}{diam}
\newcommand{\Circlearrowleft}{\rotatebox{180}{$\circlearrowright$}}
\DeclareMathOperator{\Cl}{Cl}				% conjugation class of something.
\newcommand{\rad}{rad}						%radius operator


% MOSIM
% % % rounding symbols
\newcommand{\rndup}{\Delta}
\newcommand{\rnddown}{\nabla}
\newcommand{\rndinterval}{\diamondsuit}
% % % rounding symbols for aritmetic, which can be filled with a math symbol!!!
\newcommand{\rndupin}[1]{%
	\ooalign{\Large $\Delta$\cr\hss\raisebox{0.4ex}{\scriptsize $#1$}\hss}}%
\newcommand{\rnddownin}[1]{%
	\ooalign{\Large $\nabla$\cr\hss\raisebox{0.9ex}{\scriptsize $#1$}\hss}}%
\newcommand{\rndintervalin}[1]{%
	\ooalign{\Large $\rndinterval$\cr\hss\raisebox{0.45 ex}{\scriptsize $#1$}\hss}}%
\newcommand{\compop}[1]{%
	\ooalign{\Large $O$\cr\hss\raisebox{0.3 ex}{\scriptsize $#1$}\hss}}% % computer operation to fill with something

% leq, geq
\renewcommand{\le}{\leqslant}
\renewcommand{\ge}{\geqslant}

%FPN
\newcommand{\emin}{\underline{e}}
\newcommand{\emax}{\overline{e}}
\newcommand{\maxval}[1]{\overline{#1}}
\newcommand{\minval}[1]{\underline{#1}}
\DeclareMathOperator{\denorm}{denorm}
\DeclareMathOperator{\ulp}{ulp}
\DeclareMathOperator{\pred}{pred}
\DeclareMathOperator{\success}{succ}
\DeclareMathOperator{\digits}{digits}
\newcommand{\ww}{\omega}
\newcommand{\CC}{\mathcal C}
%\DeclareMathOperator{\mid}{mid}
\newcommand{\one}{\mathbbm 1}				% interval 1
\newcommand{\zero}{\mathbbm 1}				% interval 0

% ceil and floor brackets
\DeclarePairedDelimiter\ceil{\lceil}{\rceil}
\DeclarePairedDelimiter\floor{\lfloor}{\rfloor}

% logical values
\newcommand{\true}{\text{true}}
\newcommand{\false}{\text{false}}

% get this stupid arrows:
%\usepackage{mathabx,graphicx}  % ---> add to mathoperators
%\def\Circlearrowleft{\ensuremath{%
%		\rotatebox[origin=c]{180}{$\circlearrowleft$}}}
%\def\Circlearrowright{\ensuremath{%
%		\rotatebox[origin=c]{180}{$\circlearrowright$}}}
%\def\CircleArrowleft{\ensuremath{%
%		\reflectbox{\rotatebox[origin=c]{180}{$\circlearrowleft$}}}}
%\def\CircleArrowright{\ensuremath{%
%		\reflectbox{\rotatebox[origin=c]{180}{$\circlearrowright$}}}}
%\begin{document}
%	\Huge
%	$\circlearrowleft \circlearrowright $
%	
%	$\Circlearrowleft \Circlearrowright $
%	
%	$\CircleArrowleft \CircleArrowright $

% % % local packages
%\usepackage{braids}

\newlist{remarkenum}{enumerate}{1}
\setlist[remarkenum]{label=(\alph*),ref=\theremark~(\alph*)}
\crefalias{remarkenumi}{remark}

\newlist{propenum}{enumerate}{1}
\setlist[propenum]{label=(\alph*),ref=\theproposition~(\alph*)}
\crefalias{propenumi}{proposition}

\newlist{expenum}{enumerate}{1}
\setlist[expenum]{label=(\alph*),ref=\theexample~(\alph*)}
\crefalias{expenumi}{example}

\newlist{lemmaenum}{enumerate}{1}
\setlist[lemmaenum]{label=(\alph*),ref=\thelemma~(\alph*)}
\crefalias{lemmaenumi}{lemma}

\newlist{defenum}{enumerate}{1}
\setlist[defenum]{label=(\roman*),ref=\thedefinition~(\roman*)}
\crefalias{defenumi}{definition}

\title{\textbf{MOSIM WS 19/20}}
\author{Dozent: Prof. Dr. \person{Wolfgang V. Walter}}

\begin{document}
\pagenumbering{roman}
\pagestyle{plain}

\maketitle

\hypertarget{tocpage}{}
\tableofcontents
\bookmark[dest=tocpage,level=1]{Inhaltsverzeichnis}

\pagebreak
\pagenumbering{arabic}
\pagestyle{fancy}

%\chapter*{Vorwort}
%\input{./TeX_files/Vorwort}
%\chapter{Introduction}
%% !TeX spellcheck = en_US
\textbf{14.10.2019}
\begin{itemize}
	\item Example:
	\begin{align*}
		e^{-20}=1-20+\frac{400}{2}-\frac{8000}{6}+..+-.. \frac{x^n}{n!}
	\end{align*}
	is problematic when you perform addition!
	\item Here comes the tableau
	\item IEEE 
	\begin{itemize}
		\item Standards: example: 754 (1985,2008,2018)
		\item Konverenz ARITH
		\item Institute of Electrical and Electronics Engineers
		\item Standard 1788 - for intervals/ interval arithmetic
	\end{itemize}
	\item Scientific Computation (step by step) %TODO make a node diagram later!
	\begin{enumerate}
		\item definition of problem
		\item \emph{simplification}
		\item physical problem
		\item \emph{model error}
		\item mathematical modeling
		\item \emph{approximation error}
		\item mathematical approximation
		\item \emph{rounding error}
		\item computation
		\item \emph{error analysis}
	\end{enumerate}
	So we have 4 possible sources of errors!
	\item numerical differentiation (derivatives)
	\begin{align*}
		f'(x)\approx \frac{f(x+h)-f(x-h)}{2h} \quad \text{ for small $h$}, h > 0
	\end{align*}
	With smaller $h$ we get better results, \emph{but} do not make $h$ to small. Then it is going to be even worse, because we get rounding - off error.
	\bigskip
	\textbf{15.10.2019}\\
	\item Scientific software in trouble\\
	\begin{tabularx}{\textwidth}{|X|X|X|}
		\hline
		\textbf{mathematics/brain}   & \textbf{app/software}    & \textbf{Hardware/Comp.} \\
		real, complex numbers & floating point numbers & FPN \\
		dense continuum & discrete grid & less fixed NFs\\
		strong structure & weak algebraic structure & arithemtic structure\\
		order relation (OR) $\le$ & OR $\le$ & directed rounding \\
		OR $\subseteq$ & OR $\subseteq$ & directed rounding \\
		intervals, sets & guaranteed inclusions & directed rounding\\
		function & function value inclusions & directed rounding\\
		math. notation & math. notation & machine code\\
		math. objects & data abstraction, OOP & memory management \\
		math. operations & operator overload & memory management\\
		sequential methods & parallelization & multicore/ M-processor\\
		\hline
	\end{tabularx}
	$\to$ \person{Willian Kahan} (``directed roundings'')
	\item Number formats (NFs)
	\begin{enumerate}
		\item single 32 bit
		\item double 64 bit
		\item extended 80 bit 
		\item quadruple 128 bit
	\end{enumerate}
	\item Rounding 
	\begin{center}
		\tikzset{every picture/.style={line width=0.75pt}} %set default line width to 0.75pt        
		
		\begin{tikzpicture}[x=0.75pt,y=0.75pt,yscale=-1,xscale=1]
		%uncomment if require: \path (0,310); %set diagram left start at 0, and has height of 310
		
		%Straight Lines [id:da8610701323480399] 
		\draw    (46,41) -- (122.83,40.67) ;
		%Straight Lines [id:da4813505677920028] 
		\draw    (55.83,31.67) -- (56,50) ;
		%Straight Lines [id:da6888001065658323] 
		\draw    (109.83,31.67) -- (110,50) ;
		%Straight Lines [id:da482468059869271] 
		\draw    (86.83,32.67) -- (87,51) ;
		
		% Text Node
		\draw (87,57) node   [align=left] {x};
		% Text Node
		\draw (110,57) node   [align=left] {$\rndup$x};
		% Text Node
		\draw (54,57) node   [align=left] {$\rnddown$x};
		
		\end{tikzpicture}
	\end{center}
	Interval $X=(x,x) , \rndinterval X=[\rnddown x, \rndup x]$, but then we have errors in PC ($10^{20}+13)-10^{20} \to 10^{20}-10^{20} \to 0$ instead of 13 (Here rounding - off error is missing)
	\item Interval mathematics\\
	$\to$ \emph{Slogan}: Interval-algorithm always possible close and mathematical correct enclosures of solutions.\\
	If this is \emph{not} possible, errors must be reported!
	\begin{itemize}
		\item control rounding errors
		\item enclosure approximation errors
		\item enclosure of derivatives and remainders
		\item guaranteed enclosures of the solution (sets) 
		\item proof of existence of a solution in calculated interval
		\item If necessary, proof of  uniqueness of the solution
		\item control of step size, order, accuracy of results
		\item sharp termination condition and reliable error detection
		\item no fantasy solutions or numerical artifacts
	\end{itemize}
	Tools:
	\begin{itemize}
		\item automatic differentation
		\item LGS, NLGS
		\item Fixed-point theorems (e.g. \person{Brouwer})
		\item integration ,..
	\end{itemize}
	\item classic \person{Newton} method
	\newline $x_0=$ start value \smallskip
	\newline
	\begin{align*}
		x_k=x_{k-1}-\frac{f(x_{k-1})}{f'(x_{k-1})} \quad \with x_0 = \text{ initial value} 
	\end{align*}
	\item Interval \person{Newton} method
	\begin{align*}
		x_k=(mid(x_{k-1})-\frac{f(mid(x_{k-1})}{f'(x_{k-1})})\cap x_{k-1} \quad \with x_0 = \text{ initial interval} 
	\end{align*}
	(has quadratic convergence)
	\newline CISC $\to$ RISC $\to$ better pipelines (RISC is today in every processor!)
	\begin{align*}
		S_0=1+16^N-16^N+16^{N-1}\pm \dots +16^{-N}-16^{-N}+16^N-16^{N-1}+16^{N-1}-16^{N-1}
	\end{align*}
	calculation with 4 pipelines 
	% TODO (Here the drawing is missing)
	we have here obliteration problems!
	\item \person{Knuth} - computer science pioneer, \person{Turing}, \person{Wilkinson} 
	\item Requirements for programming systems
	\begin{itemize}
		\item mathematical notation (simple program structure, clear, reusable, modulary, overloading, simple formulas)
		\item Data abstraction (new data objects/ operations from existing operators combining
		\item automatic memory management (dynamic objects variable size)
		\item accuracy and inclusion
	\end{itemize}
\end{itemize}
%\chapter{Floating point numbers}
%% !TeX spellcheck = en_US
\textbf{28.10.2019}\\
A floating point number is 
\begin{align*}
	x = \begin{cases}
	= (-1)^{S_x}\cdot m_xb^{e_x}=(S_x,m_x,e_x) &\quad \\
	0 \quad
	\end{cases}
\end{align*}
with sign bit (Vorzeichen-bit) $S_x\in \{0,1\}$, mantissa $m_x=0.m_1m_2..m_l$ and exponent $e_x\in \{\emin,e+1,\dots,\emax\} \with \emin \approx -\emax$. The mantissa digits are $m_i\in S-\{0,1,..,b\}$ with $b:=b-1$ and mantissa length $l\in N^+$ ($b =$ base, $\emin =$ emin, $\emax =$ emax and $\N^+ = \N\setminus \set{0,1}$).
\begin{itemize}
    \item  Floating point format $R=R(b,l,\emin,\emax,\denorm)$
    with
    \begin{align*}
    	\denorm = \begin{cases}
    		\true &\quad\text{if denormalized FPN is allowed}\\
    		\false &\quad\\
    	\end{cases}
    \end{align*}
  \item normalized FPN $x\neq 1$: has as first mantissa digit $m_1\neq 0$ $(b=2 \implies m_1=1!)$
  \item unnormalized FPN $x\neq0$ has $m_1=0$
  \item Normalization of a FPN $x \neq 0$: mantissa digits to push for $k$ digits to the left ($k =$  number of leading 0-digits) and subtracting $k$ from $e_x$. $e_{x_neu}=e_x-k$.
  \newline Normalization is only possible if $e_{xneu} = e_x-k \geqslant e$
  \item denormalized FPN: $e_x = e$ and mantissa not normalized or can not be normalized
  \item number line is symmetrical to zero! Here its supposed to be the line.
  \item \begriff{\person{Wilkinson}-epsilon}
  \begin{align*}
  	\epsilon:=b^{1-l}=\frac{1}{b^{l-1}}
  \end{align*}
  is the biggest relative gap between two neighbouring normalized FPN $\sim$ the biggest relative error in solving 
  %TODO add example!
  \item \begriff{unit in last place} ulp:
  \begin{align*}
    \ulp(x) := \begin{cases}
    	b^{e_x-l} &\quad\text{if x is normal}\\
    	b^{e-l} &\quad\text{if x is denormalized or }0
    \end{cases}
  \end{align*} 
 ``unit in the last place'' (at the mantissa of $x$. 
  \begin{align*}
  	x=0.m_1 m_2...m_l\cdot b^{e_x} \with m_l = b^{e_x - l}
  \end{align*}
  \item \begriff{successor} ($k \in \Z$)
  \begin{align*}
  	\succ(x) := \begin{cases}
  		x+\ulp(x) &\quad \text{if } x\neq -b^k\\
  		x+(\frac{\ulp(x)}{b}) &\quad \text{if } x=-b^k
  	\end{cases}
  \end{align*}
  \item \begriff{predecessor} ($k\in \Z$)
  \begin{align*}
  	\pred(x) := \begin{cases}
  		x-\ulp(x)& \quad\text{if } x\neqb^k\\
  		x-(\frac{\ulp(x)}{b})& \quad\text{if } x=b^k	
  	\end{cases}
  \end{align*}
  \begin{tabularx}{\textwidth}{|X|X|}
  	\hline
  	\textbf{$x$}   & $\digits_b(x)$\\
  	0 & 1\\
  	$b^0 = 1$ & \vdots \\
  	$\vdots$ & $\vdots$ \\
  	$b-1$ & 1\\
  	$b^1 = [10]_b$ & 2\\
  	$\vdots$ & $\vdots$\\
  	$b^2 -1$ & 2\\
  	$b^2 = [100]_b$ & 3
  	\hline
  \end{tabularx}
	where we have base $b \in \N\setminus \set{0,1}$ and $\digits_b(x) = \floor{ \log_b(x)} +1$
\newline Interval arithmetics ( Wrap-around) \from in two-part complement
\newline exact result: $z^*=x+-* y$, $x,y\in I_n$ (Integer with n bits inclusive sign)
\newline Here please make the signs one above the other
\newline generated result: $z:=((z^*+z^{n-1})mod2^n)-2^{n-1}\in I_n$
\newline for normal mantissa $m_x$ it holds: $\frac{1}{b}\leqslant m_x<1$
\newline for normal mantissa $m_x$ it holds: $0\leqslant m_x<\frac{1}{b}$ \smallskip

\begin{itemize}
    \item 2 equivalent display possibilities for FPN $x\not = 0:$
    \newline $x=(-1)^{S_x}\cdot m_x\cdot b^{e_x}=(-1)^{S_x}\cdot M_x\cdot b^{e_x-l}$; $M_x=m_x\cdot b^l \in N$
    \newline Here please make $N$ as the $N$ from the natural numbers
    \newline $b^{l-1}\leqslant M_x<b^l$
    \newline Optionally one could also add $M_x$
    \newline \implies $|M_x|<b^l$, $M_x\in Z$
    \newline \implies Every FPN $x$ is an integer multiple of its ulp(x)`s.
    \item ulp(1)=$\epsilon$, since $1=0,10...0\cdot b^1$ and $\epsilon=b^{1-l}=$ulp(1) 
    \newline Here just the bracket is missing. Compare the actual photo.
    \newline $x\not =0, x\in R$ \from Roster: 
    \begin{itemize}
        \item succ($x$)$\leqslant x(1+\epsilon)=x(1+$ulp($1$)$=x\cdot $(succ($1$))
        \item succ($x$)$\geqslant x(1-\epsilon)=x(1-$ulp($1$)$=x\cdot $(pred($1$))
    \end{itemize}
    \item relative distance between 2 neighbouring FPN:
    \newline Here comes the graph
    \item Rounding:
    \newline Rounding is a mapping $o:R\xrightarrow{}R $(?) with the following properties 
    \newline (R1) $ox=x, \forall x\in R$ (projection)
    \newline (R2) $x,y \in R: x\leqslant y \implies ox\leqslant oy$ (monotonicity)
    \newline Here again comes the graph
    \newline An antisymmetric rounding i guess additionally satisfies:
    \newline (R3) $o(-x)=-ox, \forall x\in R$
    \newline Here we just need to put R as in: real numbers
    \item Established rounding possibilities are:
    \begin{itemize}
        \item \square x to nearest floating point number 
        \newline \implies error$\leqslant \frac{1}{2}$ ulp
        \newline stochastically calculated: "round to even" = to next FPN with (?) mantissa, i.e. end digit = 0 (binary)
        \newline IF the value which is to be rounded lays exactly in the middle between 2 FPN (also satisfies R3): \square (-x)=-\square x
        \item I dont know how to write this symbol
        \newline truncation, rounding \to $0$
        \newline $|x|\leqslant |x|$
        \newline maximal error <1 ulp
        \newline satisfies (R3)
        \item I dont know how to write this symbol
        \newline augmentation(= the, in absolute value, biggest FPN)
        \newline |x|\leqslant |x|
        \newline maximal error <1 ulp
        \newline satisfies (R3)
        \newline \triangle x: up(ward(s)), rounding \to $+\infty$
        \newline $x\leqslant\triangle x$
        \newline maximum error <1 ulp
        \newline not antisymmetric
        \newline \nablax: down(ward(s)), rounding \to $\-infty$
        \newline $\nabla x\leqslant x$
        \newline maximum error $< 1$ ulp
        \newline not antisymmetric
        \item instead of (R3) the antisymmetry holds: 
        \newline $\nabla (-x)=-\triangle x$
        \newline or $\triangle(-x)=-\nabla x$ (equivalent)
        \newline Intervals:
        \newline IR:$=\{x=[x,x]|x\leqslant x; x,x \in R\}$, i.e. set of the bounded, closed, real intervals
        \item Floating- point intervals:
        \newline IR$=\{x=[x,x], x\leqslant x; x,x\in R\}$ \to intervals with floating point boundaries
        \item Interval rounding (?)the diamont sign $:R\xrightarrow{}R$, with $X=[\nabla x, \triangle x]\subseteq X=[x,x]$
        \newline (?) has the property
        \begin{enumerate}
            \item (R1) (?)$X=X$, $\forall X\in R$
            \item (R2) $X,Y\in R: X\subseteq Y \implies (?)X\leqslant (?)Y$ (Inclusions theory)
            \item (R3) $(?)(-X)=-(?)X$, since $\triangle(-x)=-\nabla(x)$, $\nabla(-x)=-\triangle x \in R$
            \newline Here the things written with pencil is missing
            \newline $X+Y=[x+y,x+y]$
            \newline $X-Y=[x-y,x-y]$
            \newline I dont know how to write the brackets
        \end{enumerate}
    \end{itemize}
\end{itemize} \bigskip
%\chapter{Boolean Algebra}
%% !TeX spellcheck = en_US
\section{Boolean Algebra}
%TODO still to reformat and fill ?
\begin{itemize}
    \item $n$ FA work parallel \\
    reduce $3-n$ bit numbers to $2$ numbers \\
    C contains all (?)-bits, S all sum-bits
    \item for final sums $\sigma$ add $S+2C$ \\
    Here I dont know how to do the sums + the two graphs
\end{itemize} \smallskip

$T(n)=T_{walace}+T_{CLA}=\O(\log n)$ \\
addition of $n$ numbers requires $(n-2)$ CSA

\begin{itemize}
    \item MUltiplication ($n\times n$ bit) \\
    (?): with little HW
    \begin{itemize}
        \item A Carry-Ripple-Adder: $T=O(n^2)$
        \item A Carry-Skip/Select-Adder: $T=O(n\cdot \sqrt{n})$
        \item A CLA (Binary tree): $T=O(n\cdot \log n)$
        \item A CLA (Carry-Safe-Adder): $T\approx (n-2)+2n\approx 3n=O(n)$
    \end{itemize}
    Otherwise with a lot of HW \\
    \begin{itemize}
        \item $(n-2)$ CSA and $1$ CLA
        \item for individual multiplications ($1$-dimensional layout): $T_1=O(n+\log n)=O(n) \rightarrow$ with (?) \\
        $T_p=T_1/$ number of pipeline-steps $=O(n)/O(n)=O(1)$
        \item Wallace-Tree-Layout: $T_1=\log_{\frac{3}{2}} n+2\log_2 \approx O(\log n)$ \\
        $\rightarrow $ with Pipeline: $T_p=T_1/O(\log n)=O(1)$ \\
        ((?) = number of pipeline-steps $\approx O(\log n)$)
    \end{itemize}
   
    \item Division  \\
    $q:=(?) (a/b)$ with rest $r$$\to$do{missing}, with assumption $a>0, b>0$. \\
    For the rest it holds: $q\cdot b+r=a$ \\
    Iterative division generates $n$ "partial remainders" started with $r_0:=a$ and ended in $q:[q_{n-1} ... q_1 q_0]$ and final rest $r:=r_n$ after $n$ steps/iterations. \\
    Here comes the graph \\
    \\
    \item (?) Division:$\todo{missing}
    Init: $a \to A, b \to B, o \to P$ (possible to calculate and to save b). \\
    For $i=1,..,n$:
    \begin{enumerate}
        \item $LShift_1(R)$ (all $2n+1$ bit)
        \item $P:=P-B$
        \item If $P<0$, set low-rder position of A to O \\
        otherwise to $1$ ($q_{n-1}:=p_n$)
        \item If $P<0$ restore old $P:=P+B$
    \end{enumerate}
    After $n$ steps A contains $q$ and P contains $r=r_n$.
    \item Nonrestoring Division: \\
    init: (?) above \\
    For $i=1,..,n$ \\
    If $P<0$ \\
    \begin{enumerate}
        \item $L-Shift(R): (2n+1$ bit$)$
        \item $P:=P+B$
    \end{enumerate}
    Else $(P\ge 0)$
    \begin{enumerate}
        \item $L-Shift(R):$
        \item $P:=P-B$
    \end{enumerate}
    End if \\
    3)Set Quotient digits:
\[
	q_{n-1}:
	\begin{cases}
	0, & \text{if } P<1\\
	1 & \text{if } P\ge 0
	\end{cases}
\]
After $n$ steps: \\
4) if $(P<0)$, restore remainder by $P:=P+B$, making $P\ge 0 $ \\

\item Example: $n=8$, \\
$a=211=0.11010011\cdot 2^8$, $q=01000110=70$ \\
$b=3=0.00000011\cdot 2^8$, $r=00000001=1$ \\

Here comes the drawing. \\
Resoring & nonrestoring division are equivalent: \\
step k: $q_{n-k}$ is based on sign of $2\cdot r_{k-1}-2^n\cdot b$ in restoring devision
\begin{enumerate}
    \item if $2r_{k-1}-2^n\cdot b \ge 0:$ both algorithms do the same new partial remainder $r_k:=2r_{k-1}-2^n\cdot b$, $q_{n-k}:=1$
    \item if $2r_{k-1}-2^n\cdot b < 0:$ restoring division $\to$  $r_k:=2r_{k-1}$ \\
    Here the arrow is missing \\
    step $k+1: q_{n-(k+1)}$ is based on sign of \\
    $2r_k+2^nb=4r_{k-1}-2^nb$ \\
    \\
    non restoring division $\to$ we keep the negative partial remainder $r_k:=2r_{k-1}-2^n\cdot b (<0)$ \\
    Here the arrow is missing  \\
    step $k+1: q_{n-(k+1)}$ is based on sign of \\
    $2r_k+2^nb=2\cdot (2r_{k-2}-2^n\cdot b) + 2^nb=4r_{k-1}-2^nb$ \\
   
    $\implies$ both algorithms do the same \\
    $\to$ produce same quotient digits \\
    $\to$ but different sequence of partial remainders
\end{enumerate}
\item An example: restoring division \\
$n=4, A=14=1110, B=3=0011$ \\
\todo{Here comes the table}
\item An example: nonrestoring division \\
$n=4, A=14=1110, B=3=00011, -B=11101$ \\
\todo{Here comes the table}
\item Redundant number display: \\
Base $\beta\ge 2, \beta \in N$ \\
A number representation-system is redundant if the digit set S contains more than $\beta$ digits. We assume that any digit set S contains $0$ and its digits are continuous (no holes): \\
$S:=\{d,d+1,..,d\}$, where $d\le 0\le d$ \\
In practice we only consider digit sets S with $d\ge 1-\beta$ and $d\le \beta-1$ \\
A symmetric digit sets satisfies $d=-d$ \\
A minimally redundant digit set S contains exactly $\beta -1$ digits \\
A minimally redundant digit set S \\
$\to$ for $\beta$ even, $S:=\set{-\beta/2, \beta/2+1,..,\beta/2}$ \\
$\to$ for $\beta$ odd, $S:=\set{-(\beta+1)/2, ,..,(\beta+1)/2}$ \\

\item SRT-division ((?) \todo{missing} \person{Robertson}, \person{Tocher}, 1958) $[a/b]$ \\
class of division algorithms \\
- normalise divisor $b$ at the beginning (with $L-Shift_k$) \\
- requirement for redundant symmetric digit set for quotient digits $[q_{n-1}q_{n-2} ... q_0]$ \\
- quotient digit selection $q_{n-1}$, based on only a feq leading digits if partial remainder $r_{i-1}$ and of the divisor $b$ \\
$\implies$ use of a ($2-$dim) lookup table is possible \\
- partial remainders $r_i$ are represented in a redundant number system, e.g. as $2n-$bit, $#s$ (useful for CSA)\\

Note: $r_i=\beta\cdot r_{i-1}-q_{n-i}\cdot b$
\item single radix $2$, SRT-division: \\
-$a/b\to$ (?), assuming $b\not =0$ \\
-$a,b$ floating point, $#s$, binary point just left to registers $P/A$ and $B$.\\
1) determine the number of leading $k$ zeros in B ($n$ digits), $L-Shift_k(B), L-Shift_k(R), $ (R=P|A) \\
Note: $0\le k\le n-1 $ (since $b\not =0$) \\
$\implies$ leading bit of P is $0$ $\implies$ $|r_0|\le 1/2$ \\

2) For $i=1,..,n$ \\
a) If 3 leading bits are the same $(000 $ or $111)$, set $q_{n-1}:=0, L-Shift_1(R)$ \\
b) If 3 leading bits are NOT all the same and $P<0$, $(000 $ or $111)$, set $q_{n-1}:=(-1)=1, L-Shift_1(R), P:=P+B$ \\
c) If 3 leading bits are NOT all the same and $P\ge 0$, $(000 $ or $111)$, set $q_{n-1}:=1, L-Shift_1(R) $ and $P:=P-B$.\\

3) If the final remainder $P<0$, restore it $:P:=P+B$ and correct the quotient: $q=q-1$ (substract of $1$ in the last position $a_0$)\\

4) One obtains $r_n$ with $AR-Shift_k(P)$ \\
\implies In the SRT -Division: $|\text{remainder}|\le 1/2$ in every step \\
$\beta=2^m, S=\{-2^{m-1},..,+2^{m-1}\}$ \\

\item An example: $\beta=4, S=\{-2,-1,0,1,2\}\to$ SRT $4$ division, compute $2$ bits of quotient per iteration minimally redundant symmetric digit set \\

$\frac{r_{i+1}}{b}=4\cdot \frac{r_i}{b}-q_i$, basis $=4$. \\

I dont understand the things with pencil? \\

(1) If $|r_i/b|\le 1\implies |\frac{r_{i+1}}{b}|\le 2 \implies |\frac{r_{i+2}}{b}|\le 6$ \\
requires an extra bit for rem. per iteration, remainder becomes larger and larger \\
\implies connection of quotient via successive bit is generally impossible  \\

(2) If $|r_i/b|\le 3/4 \implies |\frac{r_{i+1}}{b}|\le 1 \implies$ case (1) \\

(3) If $|r_i/b|\le 1/2 \implies $  \\

\[
	4\cdot \frac{r_i}{b} \in
	\begin{cases}
	(\frac{3}{2},2] \implies q_{k-(i+1)}=2\\
	(\frac{1}{2},\frac{3}{2}) \implies q=1\\
	(-\frac{1}{2},\frac{1}{2}) \implies q=0\\
	(-\frac{3}{2},-\frac{1}{2}) \implies q=-1\\
	[-2,-\frac{3}{2}) \implies q=-2\\
	\end{cases}
\]

To ensure $|\frac{r_{i+1}}{b}|\le \frac{1}{2}$ \\
$\implies$ case (2) to severe. \\

Here the drawing is missing. \\
\\

\item SRT-8-Division, i.e. $\beta=8$, (?) $\to$ todo{missing} $\alpha=4, $ i.e. $S=\{-4,-3,..,3,4\}$ \\
$\implies$ best choice: $|\frac{r_i}{b}|\le \frac{4}{7} \implies |\frac{r_{i+1}}{b}|\le |\frac{32}{7}-4|=\frac{4}{7}$


\item P-D-Diagrams (partial remainder-divisor) \\
\todo{Here the graphs are missing}
\item Speeding up division vie SRT
	\begin{itemize}
		\item increase radix (use groups of digits!)
		\item compute only approximation $r_i$ or $r_i$, to be used in table look up process
		\item represent $r_i$ as $r_{i,c}$ and $r_{i,s}$ and (Here the thing under $r_{i,c}$ is missing)
		\item use CSSA to compute new partial remainder 
		$r_i=\beta\cdot r_{i-1} -q_{n-i}\cdot b$ \todo{Here the thing under $\beta\cdot r_{i-1}$ is missing}
		\item use short CPA (eg CLA) ro compute approx. remainder $r_i=r_{i,c}+r_{i,s}$
	\end{itemize}
\end{itemize}

\section*{3.2.2020}
\begin{itemize}
    \item SRT-4-Division: \\
    Assume $r=5$ bits, $b=4$ bits are used for quotient digit selection. \\
    In general we could compute the largest diameter of any uncertainly interval and compare it with the length of the overlap regions ($1/12$ in our case). \\
   
    $r: 6$ bits (including sign), $|r|\in [\frac{k}{32},\frac{k+1}{32})$, for $k=0,1,..,32$, $(0,_ _ _ _ _), 1/32 $ is min\\
   
    $b: 4$ bits (normalized), $b\in [\frac{m-1}{16},\frac{m}{16})$, for $m=9,..,16$ $(0,1, _ _ _ _), 1/16 $\\
   
    $\implies$ $|\frac{r}{b}|\in (\frac{k}{2m},\frac{k+1}{2(m-1)})=:I_{k,m}$, $\forall k\in\{0,..,31\}$, $\forall m\in\{9,..,16\}$ \\
   
    diam$(I_{k,m})=\frac{k+1}{2(m-1)}-\frac{k}{2m}=\frac{1}{2}(\frac{m(k+1)-(m-1)k}{m(m-1)})=\frac{m+k}{2m(m-1)}$ \\
   
    Maximal diameter attained for $m=9$, $k=31$ \\
    $\max diam(I_{k,m})=diam(I_{31,9})=\frac{9+31}{2\cdot 9\cdot 8}=\frac{40}{144}=\frac{5}{18}$ \\
   
    each overlap region has length $\frac{1}{12}$ \\
    the one with largest values $[1/3,5/12]$ \\
   
    Since $b$ is normalised: $\frac{1}{2}\le \frac{1}{1}=1$ \\
    Also, $|\frac{r}{b}|\le \frac{2}{3}$
    \begin{itemize}
        \item if $b=\frac{1}{2} \implies |r|=\frac{1}{3}$
        \item in any case $(forall b\in [\frac{1}{2},1))\implies |r|\le \frac{2}{3}$
    \end{itemize}
   
    If $b=\frac{1}{2} \implies m=9 \implies |r|\le \frac{1}{7}<\frac{11}{32} \implies k=10 \from$ largest possible $k$ \\
    $\implies$ $\max \diam (I_{k,m})=\diam(I_{10,9})=\frac{19}{144}>\frac{1}{12}$, but far from any overlap region \\
    Overlap region $[\frac{1}{3},\frac{5}{12}]:$ assume: $|\frac{r}{b}|\le \frac{5}{12}$ \\
    $\implies$ $|r|\le \frac{5}{12}\cdot b\in [\frac{5}{24},\frac{5}{12})$ \\
    Smallest $b(=\frac{1}{2}), m=9 \implies |r|\le \frac{5}{24}<\frac{7}{32} \implies$ largest $k=6$ \\
    $\diam(I_{6,9})=\frac{15}{144}=\frac{5}{48}(>\frac{1}{12})$ \\
   
    Uncertainty interval $(I_{6,9})$ \\
    $\frac{6}{32}\cdot \frac{16}{9}=\frac{1}{3}<\frac{r}{b}>\frac{7}{16}=\frac{7}{32}\cdot \frac{1}{1}(>\frac{5}{12}!)$ \\
    For $k=6, m=10 (b=0.1001) (I_{6,10})$ \\
    $\frac{6}{32}\cdot \frac{16}{10}=\frac{3}{10}<\frac{r}{b}<\frac{7}{18}=\frac{7}{32}\cdot \frac{7}{32}\cdot \frac{16}{9}(<\frac{5}{12}!)$ \\
   
    $\implies$ try $k=7: \frac{7}{31}\le |r| <\frac{1}{4}, b=\frac{12}{5}\cdot |r| \implies \frac{21}{40}\le b<\frac{3}{5}$ \\
    $\implies$ $m=9$ or $10$ \\
   
    for $k=7,m=9, (I_{7,9}): \frac{7}{32}\cdot \frac{16}{9}=\frac{7}{18}<\frac{r}{b}<\frac{1}{2}=\frac{1}{4}\cdot \frac{2}{1}$ \\
    $\diam(I_{7,9})=\frac{16}{144}=\frac{1}{9}>\frac{1}{12}$, but OK!\\
    \\
    for $k=7,m=10, (I_{7,10}): \frac{7}{32}\cdot \frac{8}{5}=\frac{7}{20}<\frac{r}{b}<\frac{4}{9}=\frac{1}{4}\cdot \frac{16}{9}>\frac{5}{12}$ \\
    $\diam(I_{7,10})=\frac{17}{180}>\frac{1}{12}$, but OK!\\
    \\
    for $k=7,m=11, (I_{7,11}): \frac{7}{22}<\frac{r}{b}<\frac{2}{5}<\frac{5}{12}$ \\
    $\diam(I_{7,11})=\frac{18}{220}<\frac{1}{12}$, so, always OK!\\
    \\
    for $m\ge 11 \implies \diam(I_{7,m})<\frac{1}{12}$ always \\
   
    Other cases are less critical! \\
    For $6$ bits of $r$ and $3$ bits of $b$ (leading bit always $=1$), we need a table with $2^6\times 2^3=64\times 8=512$ entries (of (?))\\
   
    \item example \\
    $r=+0.0011$ \\
    $b=+0.1001$ \\
    $\implies$ $0.0011\le r< 0.001 \iff \frac{3}{16}\le r< \frac{1}{4}$ \\
    $\implies$ $0.1001\le b< 0.101 \iff \frac{9}{16}\le b< \frac{5}{8}$ \\
   
    $\implies \frac{3}{16}\cdot \frac{8}{5}=\frac{3}{10}< \fracr{}{b}<\frac{4}{9}=\frac{1}{4}\cdot \frac{16}{9}$ \\
   
    \todo{Here the arrows are missing}
    \\
    $\implies $ overlap region $[\frac{1}{3}, \frac{5}{12}]\subset [\frac{3}{10}, \frac{4}{9}]$ \\
   
    We cannot decide wheather to use $q_{n-1}=+1$ or $+2$ \\
   
    Assume $r=6$ bit , $b=$ bit \\
    \todo{Here the arrows are missing}
    \begin{enumerate}
        \item $r=0.00110: \frac{3}{16}<r<\frac{7}{12}$ \\
        $\frac{3}{16}\cdot \frac{8}{5}=\frac{3}{10}<\frac{r}{b}<\frac{7}{18}=\frac{7}{32}]cdot \frac{16}{9}$
        \item $r=0.00111: \frac{7}{32}\le r<\frac{1}{4}$ \\
        $\frac{7}{32}\cdot \frac{5}{8}=\frac{7}{10}<\frac{r}{b}<\frac{4}{9}=\frac{1}{4}\cdot \frac{16}{9}$ \todo{Here the arrows are missing} 
    \end{enumerate}
    \item radix $\beta$ SRT-division \\
    Given $\beta\in N$ (even radix, usually a power of $2$) and given $s\in N$ (largest absolute value of any digit) and digit set $S:=\{-s,-s+1,..,s-1,s\}$ (symmetric (?)) \\
   
    Assume $\beta/2\le s\le \beta-1$ \\
   
    If we use $|\frac{r_i}{b}|\le x$, then $|\frac{r_{i+1}}{b}|\le \beta \cdot |\frac{r_i}{b}|-s\le \beta \cdot x-s=x$ \\
    \implies $s=(\beta-1)\cdot x \implies x=\frac{s}{\beta-1}$ (constant bounded) \\
   
    \todo{Here the arrows are missing}.
    This is also our measure of redundency. \\
    It is contained in $(\frac{1}{2},1]$ \\
    for minimal $s=\frac{1}{2}: |\frac{r_i}{b}|\le \frac{\beta}{2(\beta-1)} \implies $ small overlap! \\
    $\beta=4: |\frac{r_i}{b}|\le \frac{4}{2\cdot 3}=\frac{2}{3}$ \\
    $\beta=8: |\frac{r_i}{b}|\le \frac{4}{7}$
    for maximal $s=\beta-1: |\frac{r_i}{b}|\le 1\implies$ large overlap!
\end{itemize}

\section*{04.02.2020} \\
Interval $[\frac{-s}{\beta-1}, \frac{+s}{\beta-1}], $length $l=\frac{2s}{\beta-1}$ \\
can be exactly covered with $2\beta$ digits $\implies$ every digit covers a length $\frac{l}{\beta}=\frac{2s}{\beta(\beta-1)}=L$ \\
With $(2s+1)=d$ digits a length of $(2s+1)$ can be covered, i.e. a length of $(2s+1-\beta)\cdot L=(2s+1)\cdot L-l$ remains for overlap area, which is divided in $2s$ areas. \\

$(2s+1)\cdot L=(2s+1)\cdot L-l=d\cdot L-l=d\cdot \frac{l}{\beta}-l=\frac{d-\beta}{\beta}\cdot l=\frac{2s+1-\beta}{\beta}\cdot \frac{2s}{\beta-1}$ \\

With that every overlapping has an area of $\frac{2s+1\beta}{\beta(\beta-1)}$ \\
minimal redundance: $s=\frac{\beta}{2}, d=\beta+1$ \\
$|\frac{r_i}{b}|\le \frac{\beta}{2(\beta-1)}\implies l=\frac{\beta}{\beta-1}$, i.e. can be covered with exactly $\beta$ digits. \\

Every digit covers $L=\frac{1}{\beta-1}$ \\
An additional digit can be divided on (?) \\

Here the tableau is missing. \\
\\
Maximal redundance: $s=\beta-1, d=(2\cdot \beta)-1$\\
$|\frac{r_i}{\beta}|\le 1 \implies I=[-1,+1], l=2$ \\
exactly $\beta$ digits cover $I$ $\implies$ $L=\frac{l}{\beta}=\frac{2}{\beta}$ \\

With $\beta-1$ extra digits one can distribute $\frac{2(\beta-1)}{\beta}$on $2(\beta-1)=d-1$ $\implies$ every overlapping area has length of $\frac{1}{\beta}$.
\todo{Here the tableau is missing}
\begin{itemize}
    \item division through iterative methods:
    \begin{itemize}
        \item Building the inverse $\frac{1}{b}$ with $\frac{1}{2}\le b<1 $(normalised mantissa)
        \item We are searching for: $\frac{1}{b}\in (1,2]$
        \item $f(x)=\frac{1}{x}-b=0$ has a solution $x=\frac{1}{b}$ \\
        Over $x$ there needs to be a dash .
       
        \item Display of $b=[b_{n-1},b_{n-2},...,b_1,b_0]_2$ \\
        We change in $W(b)=\sum_{i=0}^{n-1} b_i\cdot 2^{i-(n-1)} \in [0,2-2^{n-1}]$ \\
        $0=f(x)=f(x)+(x-x)\cdot f^\prime (x)$ (here the dashes are missing. ) \\
       
        Fixed-point equality (Newton):
        $x=x-\frac{f(x))}{f^\prime(x)}\implies$ Newton-iteration :\\
        $x_{i+1}=x_i-\frac{f(x_i))}{f^\prime(x_i)}=x_i-\frac{\frac{1}{x_i}-b}{-\frac{1}{b^2}}=x_i+x_i^2(\frac{1}{x_i}-b)$\\
        $=2x_i-b\cdot x_i^2=(2-bx_i)\cdot x_i\implies $ e multiplications + 1 addition/subtraction. \\
       
        In every iteration the result is calculated with $3-$times the length (given $x_i$), always truncated, so that one stays at the same side of the solution $\to$ $x_{i+1}>x_i$. \\
       
        - error: $\varepsilon_i=\frac{1}{b}-x_i$ \\
        $\varepsilon_{i+1}=\frac{1}{b}-x_{i+1}=\frac{1}{b}-2x_i+bx_i^2=b(\frac{1}{b}-x_i)^2=b\cdot \varepsilon_i^2<\varepsilon_i^2$ \\
        $\implies$ quadratic convergence. \\
        \\
        - For example, for $n=64: 8$ bit inverse from the tableau, then $3$ Newton -iterations $\implies$ $\varepsilon<(((2^{-8})^2)^2)^2=2^{-64}$.
    \end{itemize}
\end{itemize}

\section*{WALTER STUFF}
Time delays of the logic gates and electronic circuits \\
tu:= logic gate time unit\\
$0$ tu for NOT ($\not$, -)\\
$1$ tu for AND ($\cdot$), OR (+), NAND(?), NOR(?) \todo{lots of signs}\\
$2$ tu for XOR(?), HA (half adder) \\
$4$ tu for FA(full adder), but only 3tu for carry! \\
$3$ tu per bit position in carry-ripple +1 tu (to finish final sum bit) \\
$1$ fatu:= 4tu (or 3 tu) = 1 full-adder time unit computing $S_i$ takes $4$ tu\\

$c_{i+1}$ takes $3$ tu: $c_{i+1}=a\cdot (b+c)+b\cdot c$ \\
Here the drawing is missing. \\

In 2`s complement addition \\

Overflow can be detected by remembering? the carry into the leading (sign) bit part? and company? it to the final carry-out overflow occurs $\iff c_n\not = c_{n-1}$ \\

Example: $n=4:$ \\
$0110=+\varepsilon $
$+0110=+\varepsilon$

Here I dont know how to write the line, so its better that you write the things. \\

Computed result is: \\
too small by $2^n$ (positive overflow), too large by $2^n$ (negative overflow) (?Here I dont know if the power is $n$ or $h$?) \\
When wrap-around occurs! \\

$c_{i+1}=G_{0,i}+P_{0,i}\cdot c_0$ \\

$G_{0,i}=q_i+p_iq_{i-1}+p_2p_{i-1}q_{i-2}+..+p_io_{i-1}...p_1p_0$ \\
$P_{0,i}=p_ip_{i-1}...p_1p_0$, where $p_i=a_i+b_i \forall i$ \\

$n=4: (i=3): $ \\
Here the drawing is missing. \\

Roughly speaking, the propagate bit $P_{i,k}$ of a group of $n=k-i+1$ bits can be computer in the same time as it takes one FA to produce s and c ((?) $4$ tu) for all $n\le 8$. \\

Carry-Skip Adder: \\
Build only carry-ripple cicuit? (for groups of bits) with extra hardware to compute propagat bits $P_{i,k}$ for these groups. \\

In order to compute the generate bits $G_{i,k}$ for these groups, we use the C-Ripple circuit(?) where $c_{ik}=?$ is always assumed ($c_{ik}=? \implies c_{out}=G_{i,k}$). \\
Carry-Skip-Adders: irregular (?)\todo{missing} \\
use $k$ groups of lengths $l_k,l_{k-1},..,l_2,l_1$ \\
Example: $n=16:$ \\

Here the drawing is missing. \\

minimal $_{n-16, irr.(?)}=7$ fatu \\

Optimal "regular" design for $n=16:$ \\
$l_{opt}=\sqrt{\frac{n}{2}}=sqrt{8} \implies $ either $l=2$ or $3$ \\
$k_{opt}=\sqrt{2n}=sqrt{32} \implies $ either $k=5$ or $6$  \\
\\
$k=5$ groups: + drawing \\
$T=10 fatu$ \\
\\
$k=6$ groups: + drawing \\
$T=8$ fatu$=2\cdot l +k-2=2\cdot 2+6-2$ \\
\\

Optimal design of irregular comp-skip adder: \\
Here the tableau is missing. \\

\\
Carry-Select adders: \\
regular design: $k$ groups of $l$ bits each $[n=l\cdot k]$ \\
$T=l+(k-1)$, the words under the equality are missing. \\

irregular design: $k$ groups of lengths $l_kl_{k-1}...l_1;$ \\
should use: $[l_1+k-2][l_1+k-3]...[l_1+1][l_1][l_1]$ \\

$T=l_1+k-1$ where $k$ is less than in the regular design! \\
$T=O(\sqrt{n})$\\

Here the tableau is missing. \\

Shifting: $n-$ bit $#s$: \\
$LShift_1(a)$ + drawing
$LShift_k(a)$, $1\le k<n$ + drawing \\
$RShift_k(a)$ + drawing \\

arithmetic right shift: duplicates leading (sign) bit \\
$ARShift_1(a)$ +drawing \\
\\
$\sum_{i=1}^{n-1} a_i \cdot 2^{i-1}-a_{n-i}2^{n-1}=\sum_{i=1}^{n-2} a_i \cdot 2^{i-1}-a_{n-i}2^{n-(?)}=\sum_{i=1}^{n-2} a_i \cdot 2^{i-1}-a_{n-i}2^{n-2}=2$`s complement ..(?)\todo{missing} \\

\\
$1111= -1$ \\
$\to$ $1111=-1=\frac{-1}{2}$ \todo{Here the sign is missing} \\
$1110= -2$ \\
$\to$ $1111=-1=\frac{-2}{2}$ \todo{Here the sign is missing} \\
$1101= -3$ \\
$\to$ $1110=-2=\frac{-3}{2}$ \todo{Here the sign is missing} \\
$1100= -4$ \\
$\to$ $1110=-2=\frac{-4}{2}$ \todo{Here the sign is missing}
Multiplication of $2 n-$ bit (?)\todo{missing}: \\
$a*b$\to$ r$ via repeated summation of partial products $a_i\cdot b:$ \\
$r;+\sum_{i=0}^{n-1} a_i\cdot 2^i\cdot b$ , here the drawing is missing. \\

Double-length register $R=P|A$ (here the box arounf $R$ is missing + drawing. \\

Algorithm: initially, set $B:=b, A:=a $ and $P:=0$ (including $c:=0$) \\
$P_0 n $steps: \\
\begin{enumerate}
    \item $P:=P+a_0\cdot B; ARShift_1(R)$
    \item $P:=P+a_1\cdot B; ARShift_1(R)$
    \item
    \item $P:=P+a_n\cdot B; ARShift_1(R)$
\end{enumerate}
(*) Here i dont know how to write the 3 dots in the enumeration. \\

After $n$ steps, $R=P|A$ (here the box is missing) contains the full double-length product $r=a*b [2n$ bits$]$. P contains approx. product and A (?) exact corresponding error [$\to$ floating -point] \\
\implies $T(n)=O_{CLA}(n\cdot \log n)$ or $O(n\cdot \sqrt{n})$ (The words under are missing.
\\
\\

Possible arguments:
\begin{enumerate}
    \item if we have groups of $k$ (?) in a, then immediatelly do $ARShift_k(R)$
    \item if we have any group of $k$ ibes ub am then $a= ... 01111 ...10....$ (The bracket under is missing) \\
    $\sum_{i=j}^l 2^i=2^{l+1}-a^j=2^{j+k}+-2^j$ \\
    \implies replace $k$ additions by $1+$ and $1-$.
\end{enumerate}

Booth recoding: recode the multiplier a\\
In the $i-$th step:
\begin{enumerate}
    \item $a_i=0$ and $a_{i-1}=0$: do nothing (only (?))
    \item $a_i=0$ and $a_{i-1}=1$: $P:=P+B$
    \item $a_i=1$ and $a_{i-1}=0$: $P:=P-B$
    \item $a_i=1$ and $a_{i-1}=1$: do nothing (only (?))
\end{enumerate}

For the first step, set $a_{-1}=0$. \\
Booth recoding is like adding $B\cdot (a_{i-1}-a_i)\cdot (?)$ in the $i-$th step. \\
\implies Less work only if $k\ge 3$, but that is not the main goal!\\
\bigskip

\section*{NEWPAGE}

$\sum_{i=0}^{n-1} B(a_{i-1}-a_i)\cdot a^i=B[(a_{n-2}-a_{n-1})2^{n-1}+(a_{n-3}-a_{n-2})2^{n-2}+ (a_{n-4}-a_{n-3})2^{n-3}+ ... + (a_{0}-a_{1})2^{1}+(a_{-1}-a_{0})2^{0}$ \\
(Here you just need to do it in scales.. \\

$=B\cdot [a_{-1}+a_0\cdot 2^0+a_1\cdot 2^1+a_2\cdot 2^2+...+a_{n-2}\cdot 2^{n-2}-a_{n-i}\cdot 2^{n-1}=B*A$, where $i$ is a sign bit.\\
(Here the bracket under is missing) \\

Booth recoding might lead to less or the same or more effort to compute the product:  \\
Example: $a=[01010101]_2$ with (?) algorithm $4+$ \\
Booth: $a=[11111111]_2$ with Booth: $4+ & 4-$. \\
$1:=-1$
(Here the dashes above the ones are missing. \\
\implies General 2`s complement multiplicated with a and b of arbitrary sign!! \\

Need:
\begin{itemize}
    \item $ARShift_1$
    \item store the most recently shifted -out bit of a [$\to$ $a_{-1}$]
    \item use 2`s complement adder [for $P+B & P-B$]
    \item ignore (throw away) $c_{out}$[eliminate extra]
\end{itemize}

Example: $n=4:$ multiply $(-6)\cdot (-5)$, where $a:=-6, b:=-5$ \\
$b=-5=1011_2$\\
$a=-6=1010_2$\\
$-b=+5=0101_2$ \\
Here the tableau is missing. \\
\bigskip

\section*{NEWPAGE}
Booth recoding uses signes bits, i.e. the digit set $D:=\{-1,0,+1\}$. \\
To reduce the # of iterations, i.e. higher radix (basis), e.g. $2^2=4$ or $2^3=8$. \\

Minimal Symmetric digit sets are: \\
$S=\{-2^{k-1},-2^{k-1}+1,..,0,..,+2^{k-1}\}$ when using radix $2^k$, i.e. $k-$bit groups. \\
This reduces the # of iterations to $\frac{n}{k}$. Here the sign is missing. \\

Example: If we use $k=1$ bits at a time, we will need $\frac{n}{2}$ (the sign is missing) iterations using $S=\{-2,-1,0,1,2\}$. This radix $4$ Booth recoding also requires last shifted -out bit: \\
(Here the tableau is missing. \\

Radix $4=2^2$ only requires $LShift_1$ (for $+-2B$) (The signs are supposed to be one over another), and `s complement of $B$ ($B$ (with dash) for negative (?)). \\

Table of multiples of $B$:\\
Here the tableau is missing. \\

Precomputing multiples of $B$ typically requires only $2$ (or maybe $3$) operations ($LShifts & $ addiion). Larger radixes are not very useful ([normally $2^2$ or (?)]). \bigskip

\section*{NEWPAGE}

How to add many $(k), n-$digit #s: +drawing: \\

Using $n$ FAs, we can add $2($ or $3) n-$digits #s $D_1=[d_{n-1}^1d_{n-2}^1...d_{1}^1d_{0}^1]_2$ and $D_2=[d_{n-1}^2d_{n-2}^2...d_{1}^2d_{0}^2]_2$ in the first FA-cycle $(T=1$ fatu$)$ and then $1$ additional $D_i=[d_{n-1}^id_{n-2}^i...d_{1}^id_{0}^i]_2$ in every FA-cycle until the last # $D_k$ has been added (after $(k-1)$ or $(k-2)$ fatu).  \\

To finish all carry propagation requires an extra $(n-1)$ fatu, so we need a total of $T(n,k)=n+k-2$ fatu, instead of (or $n+k-3$) \\
$O(n\cdot k)$ fatu if carries are (?) every time! \\

Here the 2 drawings is missing. \\
\bigskip

\section*{NEWPAGE}
Wallace Tree: + drawing \\

Wallace Tree: (radix $2$) \\
$T(n):=T_{Wallace}(n)+T_{CLA}(2)=O(\log_{\frac{3}{2}} n+\log_2 n)=O(\log n)$ + drawing
\bigskip

\section*{NEWPAGE}
Multiplication ($n$ bit \times $n$ bit)\\
Iterative:\\
with little HW: \\
\begin{itemize}
    \item $1 CRippleA: T=O(n^2)$
    \item  $1 CSkipA|CSelectA: T=O(n\cdot \sqrt{n})$
    \item $1$ CLA (binary tree): $T=O(n\cdot \log n)$
    \item $1$ CSA (carry (?) adder): $T\approx (n-2)+(2n)\approx 2n=O(n)$\\
    Here the words under the approximation are mising.
\end{itemize}

With substantially more HW: use $(n-2)$ CSA plus $1$ CLA. \\
$\to$ for single multiplication: $(n-2)$ CSA + $1$ CLA \\
(one-dimensional layout) $T_1=O(n+\log n)=O(n)$\\
\\
$\to$ pipelined:  $T_p=\frac{T_1}{\text{# pipeline stages}}=\frac{O(n)}{O(n)}=O(1)$, (Here the bracket under is missing) \\

Wallace tree layout: \\
$\to$ (?) : $-(n-2)$ CSA $+1$ CLA:  $T_1\spprox \log_{\frac{3}{2}} n+2\cdot \log_{2} n=O(\log n)$ \\
$\to$ pipelined: $T_p\frac{T_1}{O(\log n)}=O(1)$ \\
Speedup \approx # of pipeline stages: $O(\log n)$. \bigskip

\section*{NEWPAGE}
Division: $q:=(?)(a/b)$ with remainder $r$ where we assume $a\ge 0 $ and $b>0$. Result must satusfy $a=q\cdot b+r$. Iterative division produces $n$ partial remainders starting with $r_0:=a$ and resulting in $q:=[q_{n-1}q_{n-2}...q_{1}q_{0}]$ and "find" remainder $r:=r_n$ in $n$ steps (iterations). + drawing. \\

Restoring division: \\
Initialisation: $a$\to$ A, b$\to$ B$ possible precompute (?) $0$\to$ P$ \\

For $i=1,2,..,n:$ \\
\begin{enumerate}
    \item $LShift_1(R) $[all $2n+1$ bits]
    \item $P:=P-B$
    \item If $P<0$, set (?) position of $A$ to (?), otherwise to $1$ [i.e. $q_{n-i}:=p_n$] (?) (Over $p$ a dash is missing).
    \item If $P<0$, restore old $P:=P+B$.
\end{enumerate}
After $n$ steps, A contains q and P contains (?). \\

Nonresting Division: samo init-phase. \\
For $i=1,2,..,n:$ \\
If $P<0$:
\begin{enumerate}
    \item $LShift(R)$ [$2n+1$ bits]
    \item $P:=P+B$
\end{enumerate}
Else ($P\ge 0$):
\begin{enumerate}
    \item $LShift(R)$
    \item $P:=P-B$
\end{enumerate}
EndIf.\\
(Here the 1 and 2 have to be with signs + and -).\\

3. Set quotient digit
$$
q_{n-i}:=
\begin{cases}
0, & \text{for } 0>P\\
1,  & \text{for } P\ge 0
\end{cases}
$$
After $n$ steps, \\
4. If $(P<0)$, restore temainder by $P:P+B$, (?) making $P\ge 0$. \bigskip

\section*{NEWPAGE} \\

Why are restoring and (?) division equivalent: \\
In step $k$, $q_{n-k}$ is based on sign of $2r_{k-1}-2^n\cdot b$ in restoring division. \\
I. If $2r_{k-1}-2^n\cdot b \ge 0$, both algorithms do the same: new partiel remainder $r_k:=2r_{k-1}-2^n\cdot b$ and $q_{n-k}:=1$. \\
II. If $2r_{k-1}-2^n\cdot b < 0$, restoring div. sets $r_k:=2r_{k-1}$ and in step $k+1$, $q_{n-(k+1)}$ is based on sign of $2\cdot r_k-2^n\cdot b=4\cdot r_{k-1}-2^n\cdot b$, whereas in nonrestoring division, we keep the negative partial remainder $r_k=2r_{k-1}-2^n\cdot b[<0]$ and in step $(k+1, q_{n-(k+1)}$ is based on sign of $2\cdot r_k+2^n\cdot b=2\cdot (2r_{k-1}-2^n\cdot b)+1^n\cdot b=4\cdot r_{k-1}-2^n\cdot b$, so both algorithms do the same thing, i.e. produce the same quotient digits, but possibly a different (?) of partial rem.(?). If after $n$ steps (in nonrestoring division) $r_n<0$, then restore it to $r_n+2^n\cdot [\ge 0]$. \\

Example: $n=4:$ restoring division: + tableau. \\
\bigskip

\section*{NEWPAGE} \\
drawing \\

$q= 01000110=70$ \\
$r= 00000001=1$ \\
$a=211=0.11010011\cdot 2^8$ \\
$b=3=0.00000011\cdot 2^8$ \\
Maybe you could do these so the digits fall in the same column?\\

\\

Redundant number representation\\
radix(=base): $\beta\ge 2, \beta\in N$. \\

A number representing system is redundant if the digit set $S$ contains more than $\beta$ digits. We assume that (?) digit set $S$ contains (?) and its digits are (?) (no holes): \\

$S=\{d,d+1,..,d\}$ where $d\le 0\le (?)$ \\

In practise we only consider digit sets $S$ with $d\ge 1-\beta$ and $d\le \beta-1$ (here a dash over d is missing). \\

A symmetric digit set satisfies $d=-d$ (dashes are missing). \\

A minimally readundant digit set S contains exactly $\beta+1$ digits. \\

A minimally redundant symmetric digit set (?): \\
- for $\beta$ even is $S=\{-\frac{\beta}{2},-\frac{\beta}{2}+1,...,+\frac{\beta}{2}\}$ \\
- for $\beta$ odd is $S=\{-\frac{\beta+1}{2},...,+\frac{\beta+1}{2}\}$ \\

Example: "signed bits": $S=\{-1,0,1\}$\\
minimum (?) summetric radix $4$: $S=\{-2,-1,0,1,2\}$. \bigskip

\section*{NEWPAGE}
SRT Division: (?) independently\\
by D. (?) \\

SRT is a class of division algorithms characterized by: \\
\begin{itemize}
    \item divisor $b$ is initially "normalized" (by $LShift_k$)
    \item requirement for redundant symmetric digit set for quotient digits $q_{n-i} [q_{n-1}q_{n-2}...q_{0}]$
    \item quotient digit selecton based on only a few leading digits of partial remainder $r_{i-1}$ and of the divisor $b$. \\
    \implies use of a ($2-$ dimension) lookup table is possible!\\
    \item partial remainders $(r_i)$ are represented in a redundant number system, e.g. as $2 n-$ bit #s (useful for CSA). \\
\end{itemize}

Simple radix $2$ SRT division: $\frac{a}{b}-(?)$ \\
Assume $b\not = 0$ \\
Think of $a$ and $b$ as floating-point numbers with the binary point just to the left of (?) P|A (box is missing) and B. \\

1. Determine number of leading zeros $k$ in B ($n$ bit) and $LShift_k(B)$, $LShift_k(R)$, [R=P|A].\\

Note: $0\le k\le n-1$ (since $b\not =0$) \implies leading bit of P in (?) \implies $|r_0|\le \frac{1}{2}$. \\

2. For $i=1,,.,n$:\\
\begin{itemize}
    \item If leading $3$ bits of P are all equal (or (?)), set $q_{n-i}:=0$ and $LShift_1(R)$.
    \item If leading $3$ bits of P are not all equal and $P<0$, set $q_{n-i}:=-1=T$ and $LShift_1(R)$ and $P:=P+B$.
    \item If leading $3$ bits of P are not all equal and $P\ge 0$, set $q_{n-i}:=+1$ and $LShift_1(R)$ and $P:=P-B$.
\end{itemize}

3. If final remainder $P<0$, restore it: $P:=P+B$ and correct the quotient $q:=q-1$. (here the bracket underneath is missing). \\
[substraction of $1$ in last position $a_0$]. \\

4. $r_n$ is obtained $ARShift_k(P)$ (?). \bigskip

\section*{NEWPAGE} \\
Analysis of algorithms: \\
Via initial $LShift_k(B)$, the value stored in B (as floating point number) has $|B|\ge \frac{1}{2}$ and $|B|<1$. \\

Initial remainder $r_0:=LShift_k(R)$ satisfies $-\frac{1}{2}\le r_o\le \frac{1}{2}$. \\
(Here the thing above R is missing). \\

In steps: \\
Wuotient-digit selection:
$$
X(m,n)=
\begin{cases}
1, & \text{if } \frac{1}{4}\le r+{i-1}<\frac{1}{2}\\
0, & \text{if } -\frac{1}{4}\le r_{i-1}<\frac{1}{4}\\
-1, & \text{if } [-\frac{1}{2}\le] r_{i-1}<-\frac{1}{4}
\end{cases}
$$

New remainder $r_i:=2\cdot r_{i-1}-q_{n-i}\cdot b$ remains bounded: \\
(The arrow under 2 is missing). \\

$-\frac{1}{2}\le r_i< \frac{1}{2}$ \\
because: If $q_{n-i}=1(r_{i-1}\ge+\frac{1}{4})\implies$ \\
$r_i\le 2\cdot [\frac{1}{4},\frac{1}{2})-[\frac{1}{2},1)=[-\frac{1}{2},\frac{1}{2})$ \\
$q_{n-i}=0:$ obvious! \\
If $q_{n-i}=-1(r_{i-1}<-\frac{1}{4})\implies r_i\in[-\frac{1}{2},\frac{1}{2})$ \\

So SRT div. keeps $|\text{remainder}|\le \frac{1}{2}$, but only does work with $r\ge \frac{1}{4}$ or $r<-\frac{1}{4}$. \\
(The bracket under is missing). \\

\implies P only required $n$ bits so eliminate extra leading bit in P (and B). \\
(?) steps 2.a) and 2.b) only (?) bits are tested. \\

P-D-diagrams (partial -remainder division) + drawings. \bigskip

\section*{NEWPAGE} \\
SRT $4$ division: computer $2$ bits of quotient per iteration minimally redundant symmetric digit set $S=\{-2,-1,0,1,2\}$. \\

$\frac{r_{i+1}}{b}=4\cdot \frac{r_i}{b}-q_i$. (Here radix 4 is missing). \\

1. If $|r_i/b|\le 1 \implies |\frac{r_{i+1}}{b}|\le 2\implies|\frac{r_{i+2}}{b}| \le 4$\\
requires an extra bit for rem. per iteration, remainder becomes larger & larger \implies correction of quotient via successive bits is generally impossible!\\
2. If $|r_i/b|\le \frac{3}{4}\implies |\frac{r_{i+1}}{b}|\le 1\implies$ case 1!\\
3. If $|r_i/b|\le \frac{1}{2} \implies $?\\
\implies $$
4\cdot \frac{r_i}{b} \in
\begin{cases}
(\frac{3}{2},2], & \text{\implies } q_{n-(i+1)}=2\\
(\frac{1}{3},\frac{3}{2}], & \text{\implies } q=1\\
[-\frac{1}{2},\frac{1}{2}], & \text{\implies } q=0\\
[-\frac{3}{2},-\frac{1}{2}), & \text{\implies } q=-1\\
[-2,-\frac{3}{2}), & \text{\implies } q=-2
\end{cases}
$$
to ensure $|\frac{r_{i+1}}{b}|\le \frac{1}{2}$ also, etc. \\

(?) No overlap of these ranges \implies requires exact division $\frac{r_i}{b}$, \implies no fast quotient selection, no table (?).\\

4. If $|\frac{r_{i}}{b}|\le \frac{2}{3} \implies |\frac{r_{i+1}}{b}|\le |\frac{2}{3}-2|\le (?)$ \\

\implies best choice with maximal ovelap. + drawing \\

$4$ overlap regions of length $\frac{1}{12}$ each! \\
Assume that $5$ bits of $r$ and $4$ bis of $b$ are used for quotient digit selection: \\
\bigskip

\section*{NEWPAGE}\\
In general, we could compute the largest diameter of any uncertainty interval and compare it with the length of the overlap regions (\frac{1}{12} in our case). \\

$r: 6$ bits (including sign) \\
$|r|\in [\frac{k}{32},\frac{k+1}{32})$, for $k=0,1,..,32$ \smallskip

$b: 4$ bits (normalised) \\
$b\in [\frac{m-1}{16},\frac{m}{16})$, for $m=9,..,16$ \\
(This is supposed to be in a table).

$|\frac{r}{b}|\in (\frac{k}{2m},\frac{k+1}{2(m-1)}):=I_{k,m}$ $\forall k=0,..,31, \forall m=9,..,16$. \\

$\diam(I_{k,m})=\frac{k+1}{2(m-1)}-\frac{k}{2m}=\frac{1}{2}(\frac{m\cdot (k+1)-(m-1)\cdot k}{m(m-1)})$ \\
\implies $\diam (I_{k,m})=\frac{m+k}{2m(m-1)}$. \\

Maximum diameter attained for $m=9$ and $k=31$. \\

$\max \diam(I_{k,m})=\diam(I_{1,9})=\frac{9+31}{2\cdot 9\cdot 8}=\frac{40}{144}=\frac{5}{18}$.  (Here m,k under max is missing). \\

Each overlap region has length $\frac{1}{12}$; the one with largest absolute values $[\frac{1}{3},\frac{5}{12}]$. \\

Since $b$ is normalized, $\frac{1}{2}\le b<1$. \\
Also $|\frac{r}{b}|\le \frac{2}{3}$, so if $b=\frac{1}{2}$, then $|r|\le \frac{1}{3}$ and in any case $[\forall b\in [\frac{1}{2},1)]$, $|r|\le \frac{2}{3}$. \\

If $b=\frac{1}{2}$, then $m=9 \implies |r|\le \frac{1}{3}<\frac{11}{32}$ \implies largest possible $k=10$. \\

So if $b=\frac{1}{2}$, $\max \diam (I_{k,m})=\diam(I_{10,9})=\frac{19}{144}$, which is $>\frac{1}{12}$, but far from any overlap region.  \\

Overlap region $[\frac{1}{3}, \frac{5}{12}]: $ assume $|\frac{r}{b}|\le \frac{5}{12} \implies |r|\le \frac{5}{12}\cdot b\in [\frac{5}{24}, \frac{5}{12})$. \\

smallest $b(=\frac{1}{2}), m=9\implies |r|\le \frac{5}{24}<\frac{7}{32}\implies$ largest $k=6$: \\

$\diam(I_{b,9})=\frac{15}{144}=\frac{5}{48}[>\frac{1}{12}]$ \\

Uncertainty interval\\
$I_{6,9}: \frac{6}{32}\cdot \frac{16}{9}=\frac{1}{3}<\frac{r}{b}<\frac{7}{16}=\frac{7}{32}\cdot \frac{2}{1}>\frac{5}{12}!$ \\
Here the brackets under are missing . \\

for $k=6, m=10 b=0.1001..._2$: \\
$I_{6,10}: \frac{6}{32}\cdot \frac{16}{10}=\frac{3}{10}<\frac{r}{b}<\frac{7}{18}=\frac{7}{32}\cdot \frac{16}{9}<\frac{5}{12}$.\bigskip

\section*{NEWPAGE}
\begin{itemize}
    \item $k=7: \frac{7}{32}\le |r|<\frac{1}{4}, b=\frac{12}{5}\cdot |r| \implies \frac{21}{40}\le b<\frac{3}{5} \implies m=9 $ or $10$. \item $m=9: I_{7,9}: \frac{7}{32}\cdot \frac{16}{9}=\frac{7}{18}<\frac{r}{b}<\frac{1}{2}=\frac{1}{4}\cdot \frac{2}{1}$ \\
    $\diam(I_{7,9})=\frac{16}{144}=\frac{1}{9}>\frac{1}{12}$, but OK! (arrow missing).
   
    \item $m=10: I_{7,10}: \frac{7}{32}\cdot \frac{8}{5}=\frac{7}{20}<\frac{r}{b}<\frac{4}{9}=\frac{1}{4}\cdot \frac{16}{9}>\frac{5}{12}$ \\
    $\diam(I_{7,10})=\frac{17}{180}>\frac{1}{12}$, but OK! (arrow missing).
   
    \item $m=11: I_{7,11}: \frac{7}{32}\cdot \frac{16}{11}=\frac{7}{22}<\frac{r}{b}<\frac{2}{5}=\frac{1}{4}\cdot \frac{8}{5}<\frac{5}{12}$ \\
    $\diam(I_{7,11})=\frac{18}{220}<\frac{1}{12}$, so always OK! (arrow missing).
    \end{itemize}
    for longer $m\ge 11$, $\diam(I_{7,m})<\frac{1}{12}$ always! $=\frac{7+m}{2m(m-1)}$. \\
    etc. other (?) less critical! \\
    For $6$ bits of $r$ and $3$ bits of $b$ (leading bit always $=1$) we need a lookup table with $2^6\times 2^3=64\times 8=512$ (?) (at most). +table $\to$do{missing}
   
    \section*{NEWPAGE} \\
    $\to$do{Just 2 tables.}
    \bigskip
   
    \section*{NEWPAGE} \\
    $\to$do{graph +} \\
   
    Radix $\beta$ SRT division:\\
    Given $\beta$ (even radix usually a power of $2$) and given s (largest absolute value of (?) digit) and digit set $S:=\{-s,-s+1,..,s-1,(?)\}$ [symmetric & redundant], we assume that $\frac{\beta}{2}\le \beta-1$. \\
   
    If we use $|\frac{r_i}{b}|\le x$, then $|\frac{r_{i+1}}{b}|\leqls
     \beta \cdot |\frac{r_i}{b}|-s\le \beta\cdot x-s = (?) $ (Here the arrows are missing). \\
     
     \implies $s=(\beta-1)\cdot x\implies x=\frac{s}{\beta-1}$ const. (?)\\
     This is also our measure of redundance. \\
     It is contained in $(\frac{1}{2},1]$. \\
     For minimal $s=\frac{1}{2}: |\frac{r_i}{b}|\le \frac{\beta}{2(\beta-1)}\implies (?)$ \\
     
     $\beta=4: |\frac{r_i}{b}|\le \frac{4}{2\cdot 3}=\frac{2}{3}$;\\
     $\beta=8: |\frac{r_i}{b}|\le $(?);\\
     
     For maximal $s=\beta-1: |\frac{r_i}{b}|\le 1\implies$ large (?).\bigskip
     
     \section*{NEWPAGE}\\
     
     Counterexample: $r=+0.00011 ..._2$ \\
     $b=+0.1001..._2$ \\
     $\frac{3}{16}\le r<\frac{1}{4}$ \\
     \implies $0.00011\le r<0.001 \iff \frac{3}{32}\le r<\frac{1}{8}$ (here the line over the 0 is missing). \\
     
     $0.1001\le b<0.101 \iff \frac{9}{16}\le b<\frac{5}{8}$ \\
     
     $\frac{3}{18}\cdot \frac{8}{5}=\frac{3}{10}<\frac{r}{b}<\frac{4}{9}=\frac{1}{4}\cdot \frac{16}{9}=0.3<\frac{1}{3}$\\
     
     $0.444...>\frac{5}{12}=0.41666...$ \\
     
     \implies overlap region $[\frac{1}{3},\frac{5}{12}]\subset [\frac{3}{10},\frac{4}{9}]$ \\
     \implies we cannot decide wheather to use $q_{n-1}=+1$ or $+2$!??? \\
     would have required lookup table of $32\times 8$ entries.  \\
     
     Assume we use $b$ bits of $r$ (and still $4$ of $b$): \\
     \implies for case above we now have $2$ cases:
     \begin{enumerate}
         \item $r=0.00110..._2$: $\frac{3}{16}\le r<\frac{7}{32} $ \\
         $\frac{3}{16}\cdot \frac{8}{5}=\frac{3}{10}<\frac{r}{b}<\frac{7}{18}=\frac{7}{32}\cdot \frac{16}{9}$\\
         $\frac{9}{10}<\frac{1}{3}$, $0.3888..<\frac{5}{12}=0.416$ (dash over 6). \\
         
         \item $r=0.00111..._2$: $\frac{7}{32}\le r<\frac{1}{4} $ \\
         $\frac{7}{32}\cdot \frac{8}{5}=\frac{7}{20}<\frac{r}{b}<\frac{4}{9}=\frac{1}{(?)}\cdot \frac{16}{9}$\\
         $\frac{1}{2}<0.35$, $\frac{5}{(?)}=\frac{4}{(?)}$ \\
         
         \end{enumerate}
\chapter{Summer term 2020}
\section{numerical quadrature Quadratur and integration}
\subsection{Interpolation quadrature formulas}
	\begin{*notation}
		We will will make a few convinient notations
		\begin{itemize}
			\item $\powerset_n=$ set of polynomials of degree $\le n$
			\item $[n]$ for the set $\set{0,1,\dots,n}$
			\item the natural numbers $\N$ always include the 0
			\item We will use a few abbreviations: interpolation \text polynomial with ``Ipolpoly'', Newton-Cotes formula as ``NCF'' \todo[inline]{more please add here!}
		\end{itemize}
	\end{*notation}
We have: $(n+1)$ \emph{base points} (Stützstellen) $(x_i,y_i)$, $i \in \N_0$ with abscizzse canonical order on $(x_i,y_i)$ base points. Values are represented by function $f\colon y_i=f(x_i)$.\\
We need a polynomial $p(x) \in \powerset_n \with p(x_i) = y_i$ for all $i \in \N_0$.
\begin{*definition}[\person{Lagrange}-\person{Newton} Interpolation polynomial]
	There $\exists!$ solution $p \in \powerset_n$, for $j \in [n], l_j \in \powerset_n$ and we define $l_j\colon \R \to \R$
	\begin{align*}
	l_j(x_j) &= \prod_{\stackrel{i=0}{i\neq j}}^n \frac{(x-x_i)}{x_j-x_i} = \frac{(x-x_0)(x-x_1)\cdots(x-x_{j-1}(x-x_{j+1}))\cdots (x-x_n)}{(x_j-x_0)(x_j-x_1)\cdots(x_j-x_{j-1}(x_j-x_{j+1}))\cdots (x_j-x_n)}\\
	l_j(x_i) &= \delta_{ij} \quad \forall j \in [n]\\
	p(x) &:= \sum_{j=0}^n y_j \cdot l_j(x) \in \powerset_n \nd p(x_i) = y_i l_i(x_i) = y_i
	\end{align*}
	is the \begriff{interpolation polynomial}
\end{*definition}
Assumption: $p,q \in \powerset_n$ and $p(x_i) = q(x_i)= y_i \forall i \in [n]$, then we have
\begin{align*}
	r = p-q \in \powerset_n \nd r(x_i) = p(x_i) = p(x_i)-q(x_i) \quad \forall i \in [n]
\end{align*}
and this repsentation is unique. (Why?)
\begin{*definition}[weight function]
	$f\in \CC[a,b]$ Integrand, $\omega \in \CC[a,b]$ weight function with $\omega(x) \ge 0$, $0 < \int_a^b \omega(x) \d x < \infty$ häufig $\omega(x) \equiv 1$. If $f$ has $n+1$ base points with $a \le x_0 <\dots < x_n \le b$. We know, that we can set
	\begin{align*}
	I(p_n) = \int_a^b p_n(x)\cdot \omega(x)\d x \approx I(f)
	\end{align*}
	as approximation calculated with $p_n \in \powerset_n$.
\end{*definition}
		The Ipolpoly for the stützstellen $x_j, f(x_j)$, $j  \in [n]$
		\begin{align*}
			p_n(x) = \sum_{j=0}^n f(x_j)\cdot l_j(x)
		\end{align*}
\begin{*definition}[quadrature formula]
	The quadrature formula is defined by
	\begin{align*}
	Q_n(f) &:= I(p_n) = \int_a^b\brackets{\sum_{j=0}^n f(x_j)\cdot l_j(x)}\omega(x)\d x\\
	&= \sum_{j=0}^n\brackets{f(x_j)\int_a^b l_j(x)\omega(x)\d x} = \sum_{j=0}^n \omega_j \cdot f(x_j)
	\end{align*}
	with \emph{weights}
	\begin{align*}
	\ww_j := \int_a^b l_j(x)\cdot \ww(x)\d x
	\end{align*}
\end{*definition}
		$Q_n(f)$ is the weighted sum of the function values for the base point value with fixed weights $\ww_j$. The weights only dependon the base points, the intervall bounds $a,b$ and the weight function $\ww$.
\begin{*definition}[Error]
	The errors for a quadrature formula is defined by $f(x) = p_n (x) + e_n(x)$ with $p_n, e_n \in \powerset_n$, $e \equiv 0$, if $f \in \powerset_n$. If $f \in \CC^{n+1}[a,b]$, then
	\begin{align*}
		e_n(x) &= \frac{f^{n+1}(\xi)}{(n+1)!}W_n(x) \with W_n(x) = \prod_{i\in [n]} (x-x_i)\quad \xi \in [a,b]
		\intertext{and}
		I(f) &= \int_a^b f(x)\ww(x)\d x = Q_n(f) + E_n(f) \with\\
		Q_n(f) &= \int_a^b p_n(x)\ww(x)\d x \nd E_n(f) = \int_a^b e_n(x)\ww(x) \d x = \int_a^b \frac{f^{n+1}(\xi)}{(n+1)!}W_n(x)
	\end{align*}
\end{*definition}
\begin{*definition}[Degree of accuracy]
	A quadrature formula $Q(f)$ has a \begriff{Degree of accuracy} $k$ if\begin{align*}
	I(f) = Q(f) \quad \forall f \in \powerset_n \nd \exists g \in \powerset_{k+1}\colon E(g) \neq 0
	\end{align*}
	Equivalent is: With $E(f) := I(f) - Q(f)$ holds
	\begin{align*}
	E(1) = E(x) = E(x^2) = \dots = ...
	\end{align*}
\end{*definition}
\subparagraph{Newton-Cotes formulas (overview)}
		\todo[inline]{add table here ...}
\begin{*remark}
	For $n \ge 8$ we get negative weights, numerical instability (Auslöschung). The Newton-Cotes formulas will be used in general with orders $n=1,2$ (\person{Simpson}!) or order 4. So we use also \emph{summed variants} (split interval and use different orders). Open NCF use midpoints with sub intervals (von $[a,b]$) $\to$ $n$ base points.
\end{*remark}
%\begin{*example}[\person{Runge}]
%		$f:\R\to\R$, $f(x)=\frac{1}{1+25x^2}$ \\
	equidistant base points $x_0,...,x_n$, $p\in\Pi_n$ as Ipolpoly
	\begin{center}
		\begin{tabular}{l|p{8cm}}
			\textbf{base points} & \textbf{interpolated polynomial} \\
			\hline
			2 & $1-\frac{25x^2}{26}$ \\
			\hline
			4 & $3,31565x^4 - 4,27719x^2 + 1$ \\
			\hline
			8 & $53,6893x^8 - 102,815x^6 + 61,3672x^4 - 13,203x^2 + 1$ \\
			\hline
			16 & $15403,1x^{16} - 49713,5x^{14} + 63743,8x^{12} - 41870x^{10} + 15206x^8 - 3100,35x^6 + 351,984x^4 - 22,7759x^2 + 1$
		\end{tabular}
	\end{center}
	\begin{center}\begin{tikzpicture}
		\begin{axis}[
		xmin=-1, xmax=1, xlabel=$x$,
		ymin=-1.5, ymax=3, ylabel=$y$,
		samples=1200,
		axis y line=middle,
		axis x line=middle,
		width=0.9\textwidth,
		height=0.5\textheight,
		restrict x to domain=-1:1,
		restrict y to domain=-1.5:2
		]
		\addplot+[mark=none, line width=1mm] {1/(1+25*x^2)};
		\addlegendentry{Runge-function}
		\addplot+[mark=none] {1-(25*x^2)/(26)};
		\addlegendentry{interpolation 2 base points}
		\addplot+[mark=none, color=darkgreen] {3.31565*x^4 - 4.27719*x^2 + 1};
		\addlegendentry{interpolation 4 base points}
		\addplot+[mark=none, color=black] {53.6893*x^8 - 102.815*x^6 + 61.3672*x^4 - 13.203*x^2 + 1};
		\addlegendentry{interpolation 8 base points}
		\addplot+[mark=none, color=lime] {15403.1*x^16 - 49713.5*x^14 + 63743.8*x^12 - 41870*x^10 + 15206*x^8 - 3100.35*x^6 + 351.984*x^4 - 22.7759*x^2 + 1};
		\addlegendentry{interpolation 16 base points}
		\end{axis}
		\end{tikzpicture}
	\end{center} % tikz picture does compile very long!
%\end{*example}
\begin{*remark}
	\person{Tchebychev} has improve this, compare Tschebychev polynomials.
	There exists a MATLAB package named \texttt{chebfun} by \person{Nick Trefethen}. \footnote[1]{\url{https://www.chebfun.org/}}
\end{*remark}
How do we calculate this now?
\subparagraph{Calculation of the factors of the NCF-remainders}
\begin{*example}
	We will take a look at two ``rules'' to understand the \textit{modus operandi}
	\begin{itemize}
		\item We have $I = Q+E$ with
		\begin{align*}
			Q &= (b-a)\sum_{i=0}^n \ww_i \cdot f(x_i)\\
			E &= c(b-a)^{k+2}\cdot f^{k+1}(\xi) \with k = 2 \floor{n/2} +1
		\end{align*}
		for Trapoid rule $(n=k=1)$:
		\begin{align*}
			Q &= 1/2 (f(a) + f(b)(b-1)\\
			E &= c(b-a)^3\cdot f''(\xi)\\
			Q(f) &= I(f)\forall f \in \powerset_1, Q(f) + E(f) \forall f \in \powerset_2, \text{ since } f'' \text{const}\\
			\intertext{set $f = x^2/2 \nd f' = 1$}
			\int_a^b x^2/2 &= \dots = 1/6(b-a)(b^2 +ab + a^2)\\
			Q &= 1/4(b-a)(b^2 + a^2)\\
			E &= c(b-a)(b-a^2) = c(b-a)(b^2 - 2)
		\end{align*}
		\item Simpson rule $(n=2, k = 3)$ \todo[inline]{get from numerics 1}
	\end{itemize}	
\end{*example}
\subsection{Extrapolation methods (\person{Richardson} Extrapolation $\to$ \person{Romberg}-integration)}
\subsection{\person{Gauss}-method}
\end{enumerate}

\part*{Anhang}
\addcontentsline{toc}{part}{Anhang}
\appendix

\nocite{*}
\bibliography{literatur}
\bibliographystyle{acm}

%\printglossary[type=\acronymtype]

\printindex \end{document}