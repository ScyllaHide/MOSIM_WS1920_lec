\documentclass[ngerman,a4paper,order=firstname]{mathscript}
\usepackage{mathoperators}
\usepackage{bbm} 							% hbb one


% % % local commands
\DeclareMathOperator{\Ad}{Ad}				% Adjoint
\DeclareMathOperator{\PSL}{PSL} 			% projective linear group 
\newcommand{\with}{\text{ with }}
\newcommand{\nd}{\text{ and }}
\renewcommand{\rhd}{\triangleright}
\renewcommand{\lhd}{\triangleleft} 			% normal subgroups
\DeclareMathOperator{\Set}{Set}				% Category of sets
\DeclareMathOperator{\Vect}{Vect}			% Category of vector spaces
\DeclareMathOperator{\Grp}{Grp}				% Category of groups
\DeclareMathOperator{\Mod}{Mod}				% Cat of moduls
\DeclareMathOperator{\Ann}{Ann}				% annihilator
%\DeclareMathOperator{\diam}{diam}
\newcommand{\Circlearrowleft}{\rotatebox{180}{$\circlearrowright$}}
\DeclareMathOperator{\Cl}{Cl}				% conjugation class of something.

% MOSIM
% % % rounding symbols
\newcommand{\rndup}{\Delta}
\newcommand{\rnddown}{\nabla}
\newcommand{\rndinterval}{\diamondsuit}
% % % rounding symbols for aritmetic, which can be filled with a math symbol!!!
\newcommand{\rndupin}[1]{%
	\ooalign{\Large $\Delta$\cr\hss\raisebox{0.4ex}{\scriptsize $#1$}\hss}}%
\newcommand{\rnddownin}[1]{%
	\ooalign{\Large $\nabla$\cr\hss\raisebox{0.9ex}{\scriptsize $#1$}\hss}}%
\newcommand{\rndintervalin}[1]{%
	\ooalign{\Large $\rndinterval$\cr\hss\raisebox{0.45 ex}{\scriptsize $#1$}\hss}}%
\newcommand{\compop}[1]{%
	\ooalign{\Normal $O$\cr\hss\raisebox{0.3 ex}{\scriptsize $#1$}\hss}}% % computer operation to fill with something

% leq, geq
\renewcommand{\le}{\leqslant}
\renewcommand{\ge}{\geqslant}

%FPN
\newcommand{\emin}{\underline{e}}
\newcommand{\emax}{\overline{e}}
\newcommand{\maxval}[1]{\overline{#1}}
\newcommand{\minval}[1]{\underline{#1}}
\DeclareMathOperator{\denorm}{denorm}
\DeclareMathOperator{\ulp}{ulp}
\DeclareMathOperator{\pred}{pred}
\DeclareMathOperator{\success}{succ}
\DeclareMathOperator{\digits}{digits}
%\DeclareMathOperator{\mid}{mid}
\newcommand{\one}{\mathbbm 1}				% interval 1
\newcommand{\zero}{\mathbbm 1}				% interval 0

% ceil and floor brackets
\DeclarePairedDelimiter\ceil{\lceil}{\rceil}
\DeclarePairedDelimiter\floor{\lfloor}{\rfloor}

% logical values
\newcommand{\true}{\text{true}}
\newcommand{\false}{\text{false}}

% get this stupid arrows:
%\usepackage{mathabx,graphicx}  % ---> add to mathoperators
%\def\Circlearrowleft{\ensuremath{%
%		\rotatebox[origin=c]{180}{$\circlearrowleft$}}}
%\def\Circlearrowright{\ensuremath{%
%		\rotatebox[origin=c]{180}{$\circlearrowright$}}}
%\def\CircleArrowleft{\ensuremath{%
%		\reflectbox{\rotatebox[origin=c]{180}{$\circlearrowleft$}}}}
%\def\CircleArrowright{\ensuremath{%
%		\reflectbox{\rotatebox[origin=c]{180}{$\circlearrowright$}}}}
%\begin{document}
%	\Huge
%	$\circlearrowleft \circlearrowright $
%	
%	$\Circlearrowleft \Circlearrowright $
%	
%	$\CircleArrowleft \CircleArrowright $

% % % local packages
\usepackage{braids}

\newlist{remarkenum}{enumerate}{1}
\setlist[remarkenum]{label=(\alph*),ref=\theremark~(\alph*)}
\crefalias{remarkenumi}{remark}

\newlist{propenum}{enumerate}{1}
\setlist[propenum]{label=(\alph*),ref=\theproposition~(\alph*)}
\crefalias{propenumi}{proposition}

\newlist{expenum}{enumerate}{1}
\setlist[expenum]{label=(\alph*),ref=\theexample~(\alph*)}
\crefalias{expenumi}{example}

\newlist{lemmaenum}{enumerate}{1}
\setlist[lemmaenum]{label=(\alph*),ref=\thelemma~(\alph*)}
\crefalias{lemmaenumi}{lemma}

\newlist{defenum}{enumerate}{1}
\setlist[defenum]{label=(\roman*),ref=\thedefinition~(\roman*)}
\crefalias{defenumi}{definition}

\title{\textbf{MOSIM WS 19/20}}
\author{Dozent: Prof. Dr. \person{Wolfgang V. Walter}}

\begin{document}
\pagenumbering{roman}
\pagestyle{plain}

\maketitle

\hypertarget{tocpage}{}
\tableofcontents
\bookmark[dest=tocpage,level=1]{Inhaltsverzeichnis}

\pagebreak
\pagenumbering{arabic}
\pagestyle{fancy}

\chapter*{Vorwort}
\input{./TeX_files/Vorwort}
\chapter{Introduction}
% !TeX spellcheck = en_US
\textbf{14.10.2019}
\begin{itemize}
	\item Example:
	\begin{align*}
		e^{-20}=1-20+\frac{400}{2}-\frac{8000}{6}+..+-.. \frac{x^n}{n!}
	\end{align*}
	is problematic when you perform addition!
	\item Here comes the tableau
	\item IEEE 
	\begin{itemize}
		\item Standards: example: 754 (1985,2008,2018)
		\item Konverenz ARITH
		\item Institute of Electrical and Electronics Engineers
		\item Standard 1788 - for intervals/ interval arithmetic
	\end{itemize}
	\item Scientific Computation (step by step) %TODO make a node diagram later!
	\begin{enumerate}
		\item definition of problem
		\item \emph{simplification}
		\item physical problem
		\item \emph{model error}
		\item mathematical modeling
		\item \emph{approximation error}
		\item mathematical approximation
		\item \emph{rounding error}
		\item computation
		\item \emph{error analysis}
	\end{enumerate}
	So we have 4 possible sources of errors!
	\item numerical differentiation (derivatives)
	\begin{align*}
		f'(x)\approx \frac{f(x+h)-f(x-h)}{2h} \quad \text{ for small $h$}, h > 0
	\end{align*}
	With smaller $h$ we get better results, \emph{but} do not make $h$ to small. Then it is going to be even worse, because we get rounding - off error.
	\bigskip
	\textbf{15.10.2019}\\
	\item Scientific software in trouble\\
	\begin{tabularx}{\textwidth}{|X|X|X|}
		\hline
		\textbf{mathematics/brain}   & \textbf{app/software}    & \textbf{Hardware/Comp.} \\
		real, complex numbers & floating point numbers & FPN \\
		dense continuum & discrete grid & less fixed NFs\\
		strong structure & weak algebraic structure & arithemtic structure\\
		order relation (OR) $\le$ & OR $\le$ & directed rounding \\
		OR $\subseteq$ & OR $\subseteq$ & directed rounding \\
		intervals, sets & guaranteed inclusions & directed rounding\\
		function & function value inclusions & directed rounding\\
		math. notation & math. notation & machine code\\
		math. objects & data abstraction, OOP & memory management \\
		math. operations & operator overload & memory management\\
		sequential methods & parallelization & multicore/ M-processor\\
		\hline
	\end{tabularx}
	$\to$ \person{Willian Kahan} (``directed roundings'')
	\item Number formats (NFs)
	\begin{enumerate}
		\item single 32 bit
		\item double 64 bit
		\item extended 80 bit 
		\item quadruple 128 bit
	\end{enumerate}
	\item Rounding 
	\begin{center}
		\tikzset{every picture/.style={line width=0.75pt}} %set default line width to 0.75pt        
		
		\begin{tikzpicture}[x=0.75pt,y=0.75pt,yscale=-1,xscale=1]
		%uncomment if require: \path (0,310); %set diagram left start at 0, and has height of 310
		
		%Straight Lines [id:da8610701323480399] 
		\draw    (46,41) -- (122.83,40.67) ;
		%Straight Lines [id:da4813505677920028] 
		\draw    (55.83,31.67) -- (56,50) ;
		%Straight Lines [id:da6888001065658323] 
		\draw    (109.83,31.67) -- (110,50) ;
		%Straight Lines [id:da482468059869271] 
		\draw    (86.83,32.67) -- (87,51) ;
		
		% Text Node
		\draw (87,57) node   [align=left] {x};
		% Text Node
		\draw (110,57) node   [align=left] {$\rndup$x};
		% Text Node
		\draw (54,57) node   [align=left] {$\rnddown$x};
		
		\end{tikzpicture}
	\end{center}
	Interval $X=(x,x) , \rndinterval X=[\rnddown x, \rndup x]$, but then we have errors in PC ($10^{20}+13)-10^{20} \to 10^{20}-10^{20} \to 0$ instead of 13 (Here rounding - off error is missing)
	\item Interval mathematics\\
	$\to$ \emph{Slogan}: Interval-algorithm always possible close and mathematical correct enclosures of solutions.\\
	If this is \emph{not} possible, errors must be reported!
	\begin{itemize}
		\item control rounding errors
		\item enclosure approximation errors
		\item enclosure of derivatives and remainders
		\item guaranteed enclosures of the solution (sets) 
		\item proof of existence of a solution in calculated interval
		\item If necessary, proof of  uniqueness of the solution
		\item control of step size, order, accuracy of results
		\item sharp termination condition and reliable error detection
		\item no fantasy solutions or numerical artifacts
	\end{itemize}
	Tools:
	\begin{itemize}
		\item automatic differentation
		\item LGS, NLGS
		\item Fixed-point theorems (e.g. \person{Brouwer})
		\item integration ,..
	\end{itemize}
	\item classic \person{Newton} method
	\newline $x_0=$ start value \smallskip
	\newline
	\begin{align*}
		x_k=x_{k-1}-\frac{f(x_{k-1})}{f'(x_{k-1})} \quad \with x_0 = \text{ initial value} 
	\end{align*}
	\item Interval \person{Newton} method
	\begin{align*}
		x_k=(mid(x_{k-1})-\frac{f(mid(x_{k-1})}{f'(x_{k-1})})\cap x_{k-1} \quad \with x_0 = \text{ initial interval} 
	\end{align*}
	(has quadratic convergence)
	\newline CISC $\to$ RISC $\to$ better pipelines (RISC is today in every processor!)
	\begin{align*}
		S_0=1+16^N-16^N+16^{N-1}\pm \dots +16^{-N}-16^{-N}+16^N-16^{N-1}+16^{N-1}-16^{N-1}
	\end{align*}
	calculation with 4 pipelines 
	% TODO (Here the drawing is missing)
	we have here obliteration problems!
	\item \person{Knuth} - computer science pioneer, \person{Turing}, \person{Wilkinson} 
	\item Requirements for programming systems
	\begin{itemize}
		\item mathematical notation (simple program structure, clear, reusable, modulary, overloading, simple formulas)
		\item Data abstraction (new data objects/ operations from existing operators combining
		\item automatic memory management (dynamic objects variable size)
		\item accuracy and inclusion
	\end{itemize}
\end{itemize}
\chapter{Floating point numbers}
% !TeX spellcheck = en_US
\textbf{28.10.2019}\\
A floating point number is 
\begin{align*}
	x = \begin{cases}
	= (-1)^{S_x}\cdot m_xb^{e_x}=(S_x,m_x,e_x) &\quad \\
	0 \quad
	\end{cases}
\end{align*}
with sign bit (Vorzeichen-bit) $S_x\in \{0,1\}$, mantissa $m_x=0.m_1m_2..m_l$ and exponent $e_x\in \{\emin,e+1,\dots,\emax\} \with \emin \approx -\emax$. The mantissa digits are $m_i\in S-\{0,1,..,b\}$ with $b:=b-1$ and mantissa length $l\in N^+$ ($b =$ base, $\emin =$ emin, $\emax =$ emax and $\N^+ = \N\setminus \set{0,1}$).
\begin{itemize}
    \item  Floating point format $R=R(b,l,\emin,\emax,\denorm)$
    with
    \begin{align*}
    	\denorm = \begin{cases}
    		\true &\quad\text{if denormalized FPN is allowed}\\
    		\false &\quad\\
    	\end{cases}
    \end{align*}
  \item normalized FPN $x\neq 1$: has as first mantissa digit $m_1\neq 0$ $(b=2 \implies m_1=1!)$
  \item unnormalized FPN $x\neq0$ has $m_1=0$
  \item Normalization of a FPN $x \neq 0$: mantissa digits to push for $k$ digits to the left ($k =$  number of leading 0-digits) and subtracting $k$ from $e_x$. $e_{x_neu}=e_x-k$.
  \newline Normalization is only possible if $e_{xneu} = e_x-k \geqslant e$
  \item denormalized FPN: $e_x = e$ and mantissa not normalized or can not be normalized
  \item number line is symmetrical to zero! Here its supposed to be the line.
  \item \begriff{\person{Wilkinson}-epsilon}
  \begin{align*}
  	\epsilon:=b^{1-l}=\frac{1}{b^{l-1}}
  \end{align*}
  is the biggest relative gap between two neighbouring normalized FPN $\sim$ the biggest relative error in solving 
  %TODO add example!
  \item \begriff{unit in last place} ulp:
  \begin{align*}
    \ulp(x) := \begin{cases}
    	b^{e_x-l} &\quad\text{if x is normal}\\
    	b^{e-l} &\quad\text{if x is denormalized or }0
    \end{cases}
  \end{align*} 
 ``unit in the last place'' (at the mantissa of $x$. 
  \begin{align*}
  	x=0.m_1 m_2...m_l\cdot b^{e_x} \with m_l = b^{e_x - l}
  \end{align*}
  \item \begriff{successor} ($k \in \Z$)
  \begin{align*}
  	\succ(x) := \begin{cases}
  		x+\ulp(x) &\quad \text{if } x\neq -b^k\\
  		x+(\frac{\ulp(x)}{b}) &\quad \text{if } x=-b^k
  	\end{cases}
  \end{align*}
  \item \begriff{predecessor} ($k\in \Z$)
  \begin{align*}
  	\pred(x) := \begin{cases}
  		x-\ulp(x)& \quad\text{if } x\neqb^k\\
  		x-(\frac{\ulp(x)}{b})& \quad\text{if } x=b^k	
  	\end{cases}
  \end{align*}
  \begin{tabularx}{\textwidth}{|X|X|}
  	\hline
  	\textbf{$x$}   & $\digits_b(x)$\\
  	0 & 1\\
  	$b^0 = 1$ & \vdots \\
  	$\vdots$ & $\vdots$ \\
  	$b-1$ & 1\\
  	$b^1 = [10]_b$ & 2\\
  	$\vdots$ & $\vdots$\\
  	$b^2 -1$ & 2\\
  	$b^2 = [100]_b$ & 3
  	\hline
  \end{tabularx}
	where we have base $b \in \N\setminus \set{0,1}$ and $\digits_b(x) = \floor{ \log_b(x)} +1$
\newline Interval arithmetics ( Wrap-around) \from in two-part complement
\newline exact result: $z^*=x+-* y$, $x,y\in I_n$ (Integer with n bits inclusive sign)
\newline Here please make the signs one above the other
\newline generated result: $z:=((z^*+z^{n-1})mod2^n)-2^{n-1}\in I_n$
\newline for normal mantissa $m_x$ it holds: $\frac{1}{b}\leqslant m_x<1$
\newline for normal mantissa $m_x$ it holds: $0\leqslant m_x<\frac{1}{b}$ \smallskip

\begin{itemize}
    \item 2 equivalent display possibilities for FPN $x\not = 0:$
    \newline $x=(-1)^{S_x}\cdot m_x\cdot b^{e_x}=(-1)^{S_x}\cdot M_x\cdot b^{e_x-l}$; $M_x=m_x\cdot b^l \in N$
    \newline Here please make $N$ as the $N$ from the natural numbers
    \newline $b^{l-1}\leqslant M_x<b^l$
    \newline Optionally one could also add $M_x$
    \newline \implies $|M_x|<b^l$, $M_x\in Z$
    \newline \implies Every FPN $x$ is an integer multiple of its ulp(x)`s.
    \item ulp(1)=$\epsilon$, since $1=0,10...0\cdot b^1$ and $\epsilon=b^{1-l}=$ulp(1) 
    \newline Here just the bracket is missing. Compare the actual photo.
    \newline $x\not =0, x\in R$ \from Roster: 
    \begin{itemize}
        \item succ($x$)$\leqslant x(1+\epsilon)=x(1+$ulp($1$)$=x\cdot $(succ($1$))
        \item succ($x$)$\geqslant x(1-\epsilon)=x(1-$ulp($1$)$=x\cdot $(pred($1$))
    \end{itemize}
    \item relative distance between 2 neighbouring FPN:
    \newline Here comes the graph
    \item Rounding:
    \newline Rounding is a mapping $o:R\xrightarrow{}R $(?) with the following properties 
    \newline (R1) $ox=x, \forall x\in R$ (projection)
    \newline (R2) $x,y \in R: x\leqslant y \implies ox\leqslant oy$ (monotonicity)
    \newline Here again comes the graph
    \newline An antisymmetric rounding i guess additionally satisfies:
    \newline (R3) $o(-x)=-ox, \forall x\in R$
    \newline Here we just need to put R as in: real numbers
    \item Established rounding possibilities are:
    \begin{itemize}
        \item \square x to nearest floating point number 
        \newline \implies error$\leqslant \frac{1}{2}$ ulp
        \newline stochastically calculated: "round to even" = to next FPN with (?) mantissa, i.e. end digit = 0 (binary)
        \newline IF the value which is to be rounded lays exactly in the middle between 2 FPN (also satisfies R3): \square (-x)=-\square x
        \item I dont know how to write this symbol
        \newline truncation, rounding \to $0$
        \newline $|x|\leqslant |x|$
        \newline maximal error <1 ulp
        \newline satisfies (R3)
        \item I dont know how to write this symbol
        \newline augmentation(= the, in absolute value, biggest FPN)
        \newline |x|\leqslant |x|
        \newline maximal error <1 ulp
        \newline satisfies (R3)
        \newline \triangle x: up(ward(s)), rounding \to $+\infty$
        \newline $x\leqslant\triangle x$
        \newline maximum error <1 ulp
        \newline not antisymmetric
        \newline \nablax: down(ward(s)), rounding \to $\-infty$
        \newline $\nabla x\leqslant x$
        \newline maximum error $< 1$ ulp
        \newline not antisymmetric
        \item instead of (R3) the antisymmetry holds: 
        \newline $\nabla (-x)=-\triangle x$
        \newline or $\triangle(-x)=-\nabla x$ (equivalent)
        \newline Intervals:
        \newline IR:$=\{x=[x,x]|x\leqslant x; x,x \in R\}$, i.e. set of the bounded, closed, real intervals
        \item Floating- point intervals:
        \newline IR$=\{x=[x,x], x\leqslant x; x,x\in R\}$ \to intervals with floating point boundaries
        \item Interval rounding (?)the diamont sign $:R\xrightarrow{}R$, with $X=[\nabla x, \triangle x]\subseteq X=[x,x]$
        \newline (?) has the property
        \begin{enumerate}
            \item (R1) (?)$X=X$, $\forall X\in R$
            \item (R2) $X,Y\in R: X\subseteq Y \implies (?)X\leqslant (?)Y$ (Inclusions theory)
            \item (R3) $(?)(-X)=-(?)X$, since $\triangle(-x)=-\nabla(x)$, $\nabla(-x)=-\triangle x \in R$
            \newline Here the things written with pencil is missing
            \newline $X+Y=[x+y,x+y]$
            \newline $X-Y=[x-y,x-y]$
            \newline I dont know how to write the brackets
        \end{enumerate}
    \end{itemize}
\end{itemize} \bigskip

\part*{Anhang}
\addcontentsline{toc}{part}{Anhang}
\appendix

\nocite{*}
\bibliography{literatur}
\bibliographystyle{acm}

%\printglossary[type=\acronymtype]

\printindex \end{document}
