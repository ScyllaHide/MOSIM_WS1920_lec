% !TeX spellcheck = en_US
  \begin{landscape}
  	\newcolumntype{Y}{>{\centering\arraybackslash}X} % centering for X-cols
%    \thispagestyle{empty} % probably to not use the predefined page style.
    \begin{tabularx}{\linewidth}{|c|c|Y|c|Y|Y|} %{\textwidth}{@{}llYlYl@{}} this part was for centering a few cols, but wasnt working
      \hline
      method & $n$ & weights & $\sum$ & error $E$ & degree of accuracy \\
      \hline
      Trapez & 1 & 1,1 & 2 & $-1/12 h_1^3 f''(\xi)$ & 1\\
      \person{Simpson} (\person{Kepler}) & 2 & 1,4,1 & 6 & $-1/30 h_3^5 f^(4)(\xi)$ & 3\\
      3/8 rule & 3 & 1,3,3,1 & 8 & $-1/80 h_3^5 f^(4)(\xi)$ & 3\\
      \person{Milne} & 4 & 7,32,12,32,7 & 90 & $-8/945 h_4^7 f^(6)(\xi)$ & 5\\
      \person{Milne} & 5 & 19,75,51,51,75,19 & 288 & $-275/12096 h_5^7 f^(6)(\xi)$ & 5\\
      \person{Weedle} & 6 & 41,216,27,27,216,41 & 288 & $-9/1400 h_5^9 f^(8)(\xi)$ & 7\\ \hline
    \end{tabularx}
	\begin{*remark}
		For $n \ge 8$ we get negative weights, numerical instability (loss of significance). The Newton-Cotes formulas will be used in general with orders $n=1,2$ (\person{Simpson}!) or order 4. So we use also \emph{summed variants} (split interval and use different orders). Open NCF use midpoints with sub intervals (von $[a,b]$) $\to$ $n$ base points.
	\end{*remark}
	\begin{*example}[\person{Runge}]
			$f:\R\to\R$, $f(x)=\frac{1}{1+25x^2}$ \\
	equidistant base points $x_0,...,x_n$, $p\in\Pi_n$ as Ipolpoly
	\begin{center}
		\begin{tabular}{l|p{8cm}}
			\textbf{base points} & \textbf{interpolated polynomial} \\
			\hline
			2 & $1-\frac{25x^2}{26}$ \\
			\hline
			4 & $3,31565x^4 - 4,27719x^2 + 1$ \\
			\hline
			8 & $53,6893x^8 - 102,815x^6 + 61,3672x^4 - 13,203x^2 + 1$ \\
			\hline
			16 & $15403,1x^{16} - 49713,5x^{14} + 63743,8x^{12} - 41870x^{10} + 15206x^8 - 3100,35x^6 + 351,984x^4 - 22,7759x^2 + 1$
		\end{tabular}
	\end{center}
	\begin{center}\begin{tikzpicture}
		\begin{axis}[
		xmin=-1, xmax=1, xlabel=$x$,
		ymin=-1.5, ymax=3, ylabel=$y$,
		samples=1200,
		axis y line=middle,
		axis x line=middle,
		width=0.9\textwidth,
		height=0.5\textheight,
		restrict x to domain=-1:1,
		restrict y to domain=-1.5:2
		]
		\addplot+[mark=none, line width=1mm] {1/(1+25*x^2)};
		\addlegendentry{Runge-function}
		\addplot+[mark=none] {1-(25*x^2)/(26)};
		\addlegendentry{interpolation 2 base points}
		\addplot+[mark=none, color=darkgreen] {3.31565*x^4 - 4.27719*x^2 + 1};
		\addlegendentry{interpolation 4 base points}
		\addplot+[mark=none, color=black] {53.6893*x^8 - 102.815*x^6 + 61.3672*x^4 - 13.203*x^2 + 1};
		\addlegendentry{interpolation 8 base points}
		\addplot+[mark=none, color=lime] {15403.1*x^16 - 49713.5*x^14 + 63743.8*x^12 - 41870*x^10 + 15206*x^8 - 3100.35*x^6 + 351.984*x^4 - 22.7759*x^2 + 1};
		\addlegendentry{interpolation 16 base points}
		\end{axis}
		\end{tikzpicture}
	\end{center} % tikz picture does compile very long!
	\end{*example}
  \end{landscape}